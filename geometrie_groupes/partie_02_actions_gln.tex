\section{Actions de $\mathbf{\GL(n,\K)}$ sur $\mathbf{\M(n,\K)}$}
\vspace{0.5em}

\subsection[Action par multiplication]{Action par multiplication à gauche et à
droite}
\vspace{0.5em}

\subsubsection{Action par multiplication à gauche}
\vspace{0.5em}

\index{pivot de Gauss}\index{groupe classique!$\GL_n(\K)$}
\index{groupe classique!$\M_n(\K)$}
On va montrer qu'étudier l'orbite de l'action par multiplication à gauche
revient à effectuer le pivot de Gauss.
\begin{displaymath}
 \begin{array}{rrcl}
     f :&   \GL_n(\K) \times \M_n(\K) & \longrightarrow & \M_n \\
        &(g,M)& \longmapsto & gM
       \end{array}
\end{displaymath}

On définit l'ensemble $\{E_{ij}\}$ des matrices élémentaires de $\M_n(\K)$ par
$(E_{ij})_{kl} = \delta_{ik} \delta_{jl} $.

\begin{defi}[Matrices de transvection et de dilatation]
\index{matrice!de transvection}\index{matrice!de dilatation}
\index{transvection|see{matrice de transvection}}
\index{dilatation|see{matrice de dilatation}}

 Soient $i\neq j$ et $\lambda \in \K$.
\begin{itemize}
 \item $T_{ij}(\lambda) = I_n + \lambda E_{ij}$ est une matrice de transvection,
 \item $D_i(\lambda) = I_n + (\lambda -1)E_{ii}$ est une matrice de dilatation
($\lambda \neq 0$).
\end{itemize}

Ce sont en particulier des matrices inversibles.
\end{defi}

\begin{example}[Remarque]\ 
\begin{itemize}
 \item L'action de la transvection $T_{ij}$ sur une matrice $M$ revient à
ajouter $\lambda$ fois la $j^{\text{ème}}$ ligne à la $i^{\text{ème}}$,
 \item L'action de la dilatation $D_i$ sur une matrice $M$ revient à multiplier
la $i^{\text{ème}}$ par $\lambda \neq 0$,
 \item $D_j(-1) \circ T_{ij}(1) \circ T_{ji}(-1) \circ T_{ij}(-1)$ permet
d'inverser les lignes $i$ et $j$. 
\end{itemize}
\end{example}


\begin{defi}[Forme echelonnée réduite]

Un matrice est dite sous forme échelonnée réduite si elle est de la forme
suivante :
 \begin{displaymath}\begin{pmatrix}
0\ \cdots\              0  &1      &*\cdots *&0&*\cdots *&0&*\cdots *\\
\ \vdots\ \cdots\ \vdots   &0      &0\cdots 0&1&*\cdots *&0&*\cdots *\\
\ \vdots\ \cdots\ \vdots   &\vdots &\vdots\ \cdots\ \vdots&0&0\cdots 0&1&*\cdots
*\\
\ \vdots\ \cdots\ \vdots   &\vdots &\vdots\ \cdots\ \vdots&\vdots&\vdots\
\cdots\  \vdots&0&0\cdots 0\\
   \end{pmatrix} \end{displaymath}
\end{defi}

\begin{lemm}
 
D'après les résultats sur le pivot de Gauss, dans chaque orbite de $\M_n(\K)$
sous l'action de $\GL_n(\K)$ il y a une matrice sous forme échelonnée réduite.

De plus cette matrice est unique dans chaque orbite.
\end{lemm}

\begin{proof}
 
On commence par remarquer que les matrices d'une même orbite ont les mêmes
noyaux.
En effet, si $M\in \M_n(\K)$ et $g\in \GL_n(\K)$ alors :
\begin{displaymath} x \in \Ker(gM) \Longleftrightarrow gM(x) = 0
\Longleftrightarrow M(x) = 0
\Longleftrightarrow x\in \Ker M\end{displaymath}

On motre ensuite qu'il n'existe qu'une forme échelonnée réduite pour un certain
noyau.
\end{proof}

\begin{coro}
 
$M,M' \in \M_n(\K)$ sont dans la même orbite si et seulement si $\Ker M = \Ker
M'$.
\end{coro}

\begin{coro}
 
L'ensemble des orbites et en bijection avec l'ensemble des sous-espaces
vectoriels de $\K^n$.
\end{coro}


\vspace{0.5em}
\subsubsection{Action de $\mathbf{\GL_n(\K)\times\GL_n(\K)}$ sur
$\mathbf{\M_n(\K)}$}
\vspace{0.5em}

\begin{displaymath}  \begin{array}{rrcl}
     f :&   (\GL_n(\K)\times\GL_n(\K))\times\M_n(\K)&\longrightarrow&\M_n(\K) \\
        &((g,h),M) &\longmapsto&gMh^{-1} \\
       \end{array}\end{displaymath}

On a vu que pour $M\in \M_n(\K)$, il existe $g\in \GL_n(\K)$ tel que $gM$ soit
echelonnée réduite.

On remarque que $T_{ij}^{-1}(\lambda) = T_{ij}(-\lambda)$ et que
$D_i^{-1}(\lambda) = D_i(1/\lambda)$. De plus l'action par transvection et
dilatation à droite revient à effectuer les opérations élémentaires sur les
colonnes.

Ainsi, par multiplication à droite, à partir d'une matrice échelonnée réduite,
on obtient une matrice de la forme suivante :
\begin{displaymath}\begin{pmatrix} I_r & 0 \\ 0 & 0
\end{pmatrix}\end{displaymath}

\begin{theo}

Les orbites de $\M_n(\K)$ sous l'action de $\GL_n(\K)\times\GL_n(\K)$ est en
bijection avec $\{1,\cdots,n\}$
\end{theo}

\begin{example}[Remarque]
 Tout ceci est un exemple d'une théorie plus générale sur les groupes de Lie.
\end{example}


\subsection{Action par conjugaison}
\vspace{0.5em}

On considère l'action liée au changement de base :
\begin{displaymath} \begin{array}{rrcl}
   f:&   \GL_n(\K)\times \M_n(\K)&\longrightarrow&\M_n(\K) \\
      &     (g,M)&\longmapsto&gMg^{-1} \\
       \end{array}\end{displaymath}

On peut voir cette action comme la restriction à la diagonale de l'action de
$\GL_n(\K)\times\GL_n(\K)$. En particulier, le rang est encore un invariant.
\begin{displaymath} \GL_n(\K) \simeq \{(g,g) \tq g \in \GL_n(\K)\}
\hookrightarrow\GL_n(\K)\times\GL_n(\K)\end{displaymath}

\begin{lemm}
 
Le polynôme caractéristique est un invariant de l'action par conjugaison.
\end{lemm}
\begin{proof} Si $M \in \M_n(\K)$ son polynôme caractéristique est $\chi_M(X) =
\det M-XI_n$. Si $g\in\GL_n(\K)$ alors :
\begin{displaymath}\chi_{gMg^{-1}}(X) = \det (gMg^{-1} - XI_n) = \det
\left(g\left(M-XI_n\right)g^{-1}\right) = \det g \cdotp \chi_M(X) \cdotp \det
g^{-1} = \chi_M(X)\end{displaymath}
\end{proof}

\begin{lemm}
 
Le polynôme minimal est un invariant de l'action par conjugaison.
\end{lemm}

\begin{proof}
On rappelle que le  polynôme minimal $P_M$ de $M\in \M_n(\K)$ est le polynôme
unitaire de degré minimal tel que $P_M(M)=0$.

On remarque que pour $(g,M) \in \GL_n(\K) \times \M_n(\K)$ on a :
\begin{displaymath}(gMg^{-1})^2 = gMg^{-1}gMg^{-1} = gM^2g^{-1} \quad \text{
d'où }\quad \forall
k \in \N, (gMg^{-1})^k = gM^kg^{-1} \end{displaymath}

Ainsi pour tout polynôme $P$ on a $P(gMg^{-1}) = gP(M)g^{-1}$ et en particulier
$P_M = P_{gMg^{-1}}$.
\end{proof}


\begin{example}[Remarque]
Sur $\C$ tout polynôme est scindé, donc via le lemme des noyaux, toute matrice
est trigonalisable.
\end{example}

\begin{theo}[Réduction de Jordan]

L'ensemble des orbites de $\M_n(\C) / \GL_n(\C)$ est en bijection avec
l'ensemble des matrices de la forme suivante (modulo permutation des blocs) :
\begin{displaymath}\begin{pmatrix} J_{\lambda_1} & & 0 \\ & \ddots & \\ 0 & &
J_{\lambda_k} 
\end{pmatrix} \quad \text{ avec } \quad J_{\lambda_i} =\begin{pmatrix}
\lambda_i & 1      &        & 0    \\
          & \ddots & \ddots &      \\
          &        & \ddots & 1    \\
       0   &        &        & \lambda_i\\  \end{pmatrix}\end{displaymath}
\end{theo}

\begin{prop}
 Les orbites de $\GL_2(\R)$ agissant sur $\M_2(\R)$ par conjugaison sont
représentés par les matrices :
\begin{align}
&   \{\lambda \Id \tq \lambda \in \R\} \\
& \left\lbrace\begin{pmatrix} \lambda & 0 \\ 0&  \mu \end{pmatrix} \Btq
\lambda,\mu \in \R, \lambda \neq\mu \right\rbrace \\
& \left\lbrace\begin{pmatrix} \lambda & 1 \\ 0&  \lambda \end{pmatrix} \Btq
\lambda\in \R \right\rbrace \\
& \left\lbrace\lambda R_\theta\tq \lambda\in \R, \theta \in \R \- \pi\Z
\right\rbrace \text{ où } R_\theta = \begin{pmatrix}
\cos \theta & \sin \theta \\ -\sin\theta & \cos\theta \end{pmatrix}
\end{align}

\end{prop}

