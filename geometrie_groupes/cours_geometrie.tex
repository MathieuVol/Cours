\documentclass[a4paper,10pt,makeidx]{article}

\usepackage[utf8x]{inputenc}
\usepackage{fullpage}
\usepackage[cut=false]{thmbox}
\usepackage{amsmath}
\usepackage{amssymb}
\usepackage{amsfonts}
\usepackage[french]{babel}
\usepackage{makeidx}
\usepackage[pdftex,colorlinks,linkcolor=black]{hyperref}

\newtheorem[style=S]{defi}{Définition}[section]
\newtheorem[style=M]{prop}{Proposition}[section]
\newtheorem[style=M]{coro}[prop]{Corollaire}
\newtheorem[style=M]{lemm}[prop]{Lemme}
\newtheorem[style=L]{theo}[prop]{Théorème}
\newtheorem[style=S]{defiprop}[defi]{Définition - Proposition}
\newtheorem[style=S]{defitheo}[defi]{Définition - Théorème}


\author{Notes de Cours}
\title{Géométrie et groupes classiques}
\date{2011}




% OPERATEURS :
% ------------

% Boules
\DeclareMathOperator{\Boule}{B}
\DeclareMathOperator{\adh}{adh}

% Image / Noyau / ...
\DeclareMathOperator{\Ima}{Im\,}
\DeclareMathOperator{\Ker}{Ker\,}
\DeclareMathOperator{\Nil}{Nil\,}

% Fonctions classiques
\DeclareMathOperator{\Id}{Id}
\DeclareMathOperator{\e}{e}
\DeclareMathOperator{\Frac}{Frac}
\DeclareMathOperator{\Log}{Log}
\DeclareMathOperator{\Exp}{Exp}
\DeclareMathOperator*{\Sup}{Sup}
\DeclareMathOperator*{\Inf}{Inf}
\DeclareMathOperator*{\Max}{Max}
\DeclareMathOperator*{\Min}{Min}
\DeclareMathOperator{\Zen}{Z} % Zentrum (centre)
\DeclareMathOperator{\Cen}{C} % Centralisateur
\DeclareMathOperator{\Com}{C} % Combinatoire
\DeclareMathOperator{\Nor}{N} % Normalisateur
\DeclareMathOperator{\Rad}{Rad} % Radical (de Jacobson)
\DeclareMathOperator{\pgcd}{pgcd}
\DeclareMathOperator{\ppcm}{ppcm}
\DeclareMathOperator{\Spectre}{Sp}
\DeclareMathOperator{\Aut}{Aut}


% Fonction classiques (Actions)
\DeclareMathOperator{\Stab}{Stab\,}

% Fonctions classiques (Matrices)
\DeclareMathOperator{\Mat}{Mat\,}
\DeclareMathOperator{\Tr}{Tr\,}
\DeclareMathOperator{\Diag}{Diag\,}

\makeindex

\begin{document}

% Topologie 

\newcommand{\XO}{$(X,\mathcal{O})$}
\newcommand{\XOX}{$(X,\mathcal{O}_X)$}
\newcommand{\YOY}{$(Y,\mathcal{O}_Y)$}

% ``Tel que`` dans une definition ensembliste {x \tq p(x)}

\newcommand{\tq}{ \: | \: }
\newcommand{\btq}{ \: \big| \: }
\newcommand{\Btq}{ \: \Big| \: }
\renewcommand{\|}{|\negthinspace\hspace{0.07em}|} % norme ||.||
\newcommand{\f}{\|_{{}_F}} % norme ||.||
\renewcommand{\*}{^{*}\hspace{-0.1em}} % conjuguée
\renewcommand{\t}{{}^t\!} % transposée

% Ensembles classiques de nombres (mathbb)

\newcommand{\Z}{\mathbb{Z}}
\newcommand{\N}{\mathbb{N}}
\newcommand{\R}{\mathbb{R}}
\newcommand{\C}{\mathbb{C}}
\newcommand{\K}{\mathbb{K}}
\newcommand{\Prem}{\mathbb{P}}
\newcommand{\Q}{\mathbb{Q}}
\newcommand{\UU}{\mathbb{U}}

% Ensembles classiques (Matrices)
% --------- ---------- ----------

% hermitienne (positive, def positive)

\renewcommand{\H}{\mathcal{H}}
\newcommand{\Hp}{\mathcal{H}^{+}}
\newcommand{\Hpp}{\mathcal{H}^{++}}

% symétrique (positive, def positive)

\renewcommand{\S}{\mathcal{S}}
\newcommand{\Sp}{\mathcal{S}^{+}}
\newcommand{\Spp}{\mathcal{S}^{++}}

% sous-groupes

\newcommand{\GL}{\mathcal{GL}}
\newcommand{\SO}{\mathcal{SO}}
\newcommand{\U}{\mathcal{U}}
\newcommand{\SU}{\S\U}
\newcommand{\Spl}{\mathcal{S}\textnormal{p}}
\newcommand{\USp}{\mathcal{US}\textnormal{p}}
\newcommand{\M}{\mathcal{M}}
\renewcommand{\O}{\mathcal{O}}
\newcommand{\SL}{\mathcal{SL}}

% Autres
% ------

% < et > triangle

\newcommand{\<}{\vartriangleleft}
\renewcommand{\>}{\vartriangleright}

% Divise : ''a | b``
\newcommand{\divise}{\: |\: }

\newcommand{\x}{^{\hspace{-0.1em}\times}}
\renewcommand{\-}{\!\smallsetminus\!{}}


%%%%%%%%%%%%%%%%%%%%%%%%%%%%%%%%%%%%%%%%%%%%%%%%%%%%%%%%%%%%%%%%%%%%%%%%%%%%%%%%
%% DEBUT DOCUMENT %%%%%%%%%%%%%%%%%%%%%%%%%%%%%%%%%%%%%%%%%%%%%%%%%%%%%%%%%%%%%%


% Page de titre

\maketitle
\tableofcontents

% Sections à inclure (dans différents fichiers LaTeX)

\pagebreak
\section{Actions de groupes, groupes topologiques}
\vspace{0.5em}

\subsection{Actions de groupes}
\vspace{0.5em}


\begin{defi}[Action d'un groupe $G$ sur un ensemble $X$]
\index{action de groupe}

 On dit que $G$ agit sur $X$ s'il existe une fonction
\begin{displaymath}
 \begin{array}{rrcl}
          \cdotp :& G\times X& \longrightarrow & X \\
           & (g,x)    & \longmapsto     & g\cdotp x
          \end{array}
\end{displaymath}

telle que :
\begin{enumerate}
 \item $\forall g,h \in G,\ \forall x \in X,\ g\cdotp(h\cdotp x) = (gh)\cdotp x$
 \item $\forall x \in X,\ e\cdotp x = x$
\end{enumerate}
\end{defi}

\begin{example}[Remarque]
 Il existe toujours une action triviale telle que $\forall g\in G,\ \forall x
\in X,\ g\cdotp x = x$.
\end{example}

\begin{prop}[Autre point de vue]

$G$ agit sur $X$ si et seulement si il existe un homomorphisme :
\begin{displaymath}
 \varphi : G \longrightarrow \S_X
\end{displaymath}

Un unique action $\cdotp$ est associé à un tel homomorphisme $\varphi$ :
\begin{displaymath}
\begin{array}{rrcl}
          \cdotp :& G\times X& \longrightarrow & X \\
           & (g,x)    & \longmapsto     & g\cdotp x = \varphi(g)(x)
          \end{array}
\end{displaymath}

Et réciproquement, un unique homomorphisme $\varphi$ est associé à une action
$\cdotp$ :
\begin{displaymath}
\begin{array}{rrcl}
          \varphi :& G & \longrightarrow & \S_X \\
                   & g & \longmapsto     & f : x\longmapsto f(x) = g\cdotp x
          \end{array}
\end{displaymath}
\end{prop}

\begin{defi}[Orbite]\index{orbite}

Soit $x\in X$. On définit l'orbite de $x$ sous l'action de $G$ par :
$G.x := \{g\cdotp x\tq g\in G\}$
\end{defi}

\begin{example}[Remarque]
 L'appartenance à une même orbite définit une relation d'équivalence sur $X$.
En particulier les orbites constituent une partition de $X$.
\end{example}

\begin{defi}[Stabilisateur]\index{stabilisateur}
 
Soit $x\in X$. Le stabilisateur de $x$ dans $G$ est le sous-groupe $\Stab_G(x)
= G_x = \{g\in G \tq g\cdotp x = x\}$.

Si $Y\subset X$, $\Stab_G(Y)= \{g\in G \tq g\cdotp Y \subset Y$ et
$g^{-1}\cdotp Y \subset Y \}
= \Stab_G(Y)= \{g\in G \tq g\cdotp Y = Y\}$
\end{defi}

\begin{defi}[Action transitive]\index{action de groupe!transitive}

 On dit que $G$ agit transitivement sur $X$ s'il n'y a qu'une seule orbite :
\[\forall x,y \in X,\ \exists g\in
G,\ g\cdotp x = y\]
\end{defi}

\begin{defi}[Action fidèle]\index{action de groupe!fidèle}

 On dit que $G$ agit fidèlement sur $X$ si :
\[\forall g\in G,\ \left[ (\forall x \in X,\ g\cdotp x = x) \Longrightarrow (g =
e) \right] \]
\end{defi}


\begin{example}[Remarques] \
 \begin{itemize}
  \item Avec le second point de vue, une action est fidèle si et seulement si
$\Ker\varphi = \{e\}$, ie : $\varphi$ injective.
 \item Si $G$ n'agit pas fidèlement, on peut quotienter par $\Ker\varphi$
pour obtenir une action fidèle.
 \end{itemize}
\end{example}

\begin{example}[Exemples]\ 
\begin{enumerate}
  \item $\GL_n(\R)$ agit transitivement sur $\R^n\-\{0\}$. En
particulier, il n'y a qu'une seule orbite qui est $\R^n\-\{0\}$.
  \item On rappelle que $\O(n) := \{M\in \GL_n (\R) \tq \t MM = \Id_n\}$. Les
orbites de $\O(n)$ sont les sphères centrées en l'origine.
 \end{enumerate}
\end{example}

\begin{prop}[Stabilisateur - Orbite]

 L'application suivante est une bijection :
\[  \begin{array}{rrcl} \varphi :&
              G/(\Stab_G(x)) &\longrightarrow& \omega (x) \\
              & \overline{g}   &\longmapsto    &g\cdotp x
             \end{array} \]
\end{prop}

\begin{proof}\ 

 On commence par montrer que $\varphi$ est bien définie.
Soient $g,g' \in G$ tels que $g^{-1}g' \in \Stab_G (x)$. Alors $g\Stab_G(x) =
g'\Stab_G(x)$ donc $(g^{-1}g')\cdotp x = x $ et $ g' \cdotp x = g
\cdotp x$. Ainsi les éléments d'une même classe modulo le stabilisateur
agissent de la même manière sur $X$.

La surjectivité étant évidente, il reste à montrer l'injectivité. Soient $g,g'
\in G$ tels que $g\cdotp x = g' \cdotp x$. Alors $x = (g^{-1}g')\cdotp x$ et
$(g^{-1}g') \in \Stab_G(x)$. Ainsi si deux éléments agissent de la même manière
sur $X$, alors ils sont dans la même classe d'équivalence.
\end{proof}

\subsection{Groupes topologiques}
\vspace{0.5em}

\begin{defi}[Groupe topologique]\index{groupe topologique}

Un groupe $G$ est dit topologique, s'il est muni d'une topologie pour laquelle
la multiplication et l'inversion continues.
\end{defi}

\begin{example}[Exemple]
 $(\Z,+)$ pour la topologie discrète (ie : les fermés sont les ensembles finis
de points) est un groupe topologique.
\end{example}

\begin{prop}
\index{groupe topologique!sous-groupe}
\index{groupe topologique!produit}
 \begin{enumerate}
  \item Un sous-groupe d'un groupe topologique est un groupe topologique pour
la topologie induite.
  \item Un produit de groupes topologiques est un groupe topologique pour la
topologie produit.
 \end{enumerate}
\end{prop}

\begin{proof}\ 
 
\begin{enumerate}
 \item Soit $H < G$. Un ouvert $\Omega$ de $H$ est de la forme $\Omega'\cap H$
avec $\Omega'$ ouvert de $G$. Ainsi :
\[\mu_H^{-1}(\Omega) = \mu_H^{-1}(\Omega'\cap H) = \mu_H^{-1}(\Omega')\cap
\mu_H^{-1}(H) = \underbrace{\underbrace{\mu^{-1}(\Omega')}_{\text{ouvert de
}G \times G}\cap(H\times H)}_{\text{ est un ouvert de } H\times H}\]
Et de même :
\[\iota_H^{-1}(\Omega) = \iota_H^{-1}(\Omega'\cap H) =
\iota_H^{-1}(\Omega')\cap \iota_H^{-1}(H) = \underbrace{
\underbrace{\iota^{-1}(\Omega')}_{\text{ouvert de } G}\cap H}_{\text{ouvert de
}H}\]
\item Par définition de la topologie produit.
\end{enumerate}
\end{proof}

\begin{defi}[morphisme de groupes topologiques]
\index{morphisme!de groupe topologique}
\index{morphisme!de groupe topologique!isomorphisme}
\index{morphisme!de groupe topologique!automorphisme}

Un morphisme de groupes topologiques est un morphisme de groupes qui est
continu.

On définit de même un isomorphisme de groupes topologiques s'il est
bicontinu, et un automorphisme de groupe topologique si c'est un isomorphisme
bicontinu d'un groupe topologique dans lui même.
\end{defi}

\begin{prop}

Soit $G$ un groupe topologique.
\begin{enumerate}
 \item Pour $x\in G$ la multiplication à gauche (resp. à droite) par $x$ notée
$l_x : g\mapsto xg$ (resp. $r_x : g\mapsto gx$) est un homéomorphisme mais pas
un morphisme de groupe.
 \item De même l'inversion $\iota : g\mapsto g^{-1}$ est un homéomorphisme mais
pas un morphisme de groupe.
 \item Pour $x\in X$ la conjugaison par $x$ noté $\gamma_x : g\mapsto xgx^{-1}$
est un isomorphisme de groupe topologique.

Pour avoir une action à gauche il
faut $xgx^{-1}$ et pour avoir une action à droite il faut $x^{-1}gx$.
\end{enumerate}
\end{prop}

\begin{defi}[Composante du neutre]
\index{composante connexe!du neutre}
\index{composante connexe}

Soit $G$ un groue topologique. On définit la composante du neutre $G_e$ comme
la composante connexe de $G$ contenant $e$.
\end{defi}

\begin{prop}
 \begin{enumerate}
  \item $G_e$ est un sous-groupe fermé et distingué de $G$,
  \item $\forall x \in G$ la composante connexe de $G$ contenant $x$ est $G_ex =
xG_e$,
  \item Tout sous-groupe $H < G$ qui est ouvert, est également fermé. En
particulier, $H$ est la réunion de composantes connexes de $G$ et $H \supset
G_e$.
 \end{enumerate}
\end{prop}

\begin{proof} \
 \begin{enumerate}
  \item \begin{itemize}
    \item [(fermé)] Toute composante connexe est fermée et ouverte.
    \item [(sous-groupe)] La continuité de la multiplication nous assure
 que $\mu(G_e\times G_e)$ est connexe, de plus
$\mu(G_e\times G_e) \owns e = \mu(e,e)$ et $G_e = \mu(\{e\}\times G_e)\subset
\mu(G_e\times G_e)$ donc $\mu(G_e\times
G_e) = G_e$.\\
La continuité de l'inversion assure de même que
$\iota(G_e) = G_e$.
    \item [(distingué)] Soit $x\in G$ alors $xG_ex^{-1} =
\gamma_x(G_e)\owns 1$. 
\end{itemize}
\item $G_e = l_{x^{-1}}(xG_e) \subset l_{x^{-1}}(\Cen_x) \subset G_e$.
\item TODO
 \end{enumerate}
\end{proof}

\begin{lemm}
 
Si $Y\subset X$ est fermé et ouvert, alors c'est une union de composantes
connexes de $X$.
\end{lemm}

\begin{prop}
 
Soit $G$ un groupe topologique connexe et $V$ un voisinage ouvert de $\{e\}$
dans $G$. Alors $V$ engendre $G$ :
\begin{displaymath}
 G = \bigcup_n \left(V\cup \iota(V)\right)^n
\end{displaymath}
\end{prop}

\begin{theo}[Connexité de $\GL_n$]
\index{groupe classique!$\GL_n(\C)$}
 
$\GL_n(\C)$ est un groupe topologique connexe.
\end{theo}

\begin{proof}\
\index{norme de Frobénius}
 
$\M_n(\C)$ est muni de la topologie issue de la norme de Frobénius :
\begin{displaymath}
 \|A\f = \left(\sum_{j,j} |a_{i,j}|^2 \right)^{\frac{1}{2}} =
\left(\Tr\left(AA\*\right)\right)^{\frac{1}{2}}
\end{displaymath}

On montre ensuite la connexité par arcs en considérant la connexité par arc du
complémentaire des zéros du déterminant de $zA + (1-z)\Id$.
\end{proof}

\begin{prop}
\index{norme de Frobénius}

La norme de Frobénius est une norme d'algèbre : $\|AB\f \leq \|A\f \|B\f$.
\end{prop}

\begin{defi}[Produit semi-direct]
\index{groupe topologique!produit!semi-direct}
\index{groupe topologique!produit!direct}
 
Soient $N,H$ des groupes et $\theta : H \rightarrow \Aut N$ un morphisme de
groupe. On définit :

\begin{displaymath}
 N\rtimes_\theta H \text{ avec la loi : }
(n,h)(n',h') \mapsto (n\theta(h)(n'), hh')
\end{displaymath}

Si $\theta = \Id$ on dit que le produit est direct.
\end{defi}

\begin{prop}
\index{groupe classique!$\GL_n(\R)$}

$\GL_n(\R)$ n'est pas connexe.
\end{prop}

\begin{proof}
 
En effet $\R^*$ est l'image non connexe de $\det : \GL_n(\R) \rightarrow \R$ qui
est continu.
\end{proof}

\begin{example}[Remarque]
On verra plus loin que $\GL_n(\R)$ possède exactement deux composantes connexes
: $\GL_n^+(\R)$ qui est aussi un sous-groupe distingué et $\GL^-_n(\R)$.
\end{example}

\subsection{Action de groupes topologiques}
\vspace{0.5em}

\begin{defi}[Action de groupes topologiques]\index{action de groupe topologique}

Soient $G$ un groupe topologique et $X$ un espace topologique.

On dit que $G$ agit topologiquement sur $X$ si l'action : $G\times X\rightarrow
X$ est continue.
\end{defi}

\begin{prop}
\index{orbite!connexe}
\index{orbite!compacte}
 
Si $G$ agit topologiquement sur $X$ alors :
\begin{enumerate}
 \item Si $G$ est connexe alors les orbites sont connexes.
 \item Si $G$ est compact alors les orbites sont compactes.
\end{enumerate}
\end{prop}

\begin{proof}
Une orbite est une image par une application continue.
\end{proof}

\begin{prop}
 
Si $G$ agit topologiquement sur $X$ et $x \in X$ alors $\adh (G\cdotp x)$ est
une union d'orbites (ie : $\adh (G\cdotp x)$ est stable sous l'action de $G$).
\end{prop}

\begin{proof}
 On prend une suite $(x_n)_n$ de $G\cdotp x$ qui converge vers $y\in
\adh(G\cdotp x)$. Alors par continuité de l'action, la suite des images est
aussi une suite de $G\cdotp x$ qui converge vers $g\cdotp y \in \adh (G\cdotp
x)$.
\end{proof}

\begin{coro}
 
On peut construire une relation d'ordre partiel $\preccurlyeq$ sur les orbites
de $G$ dans $X$.
\begin{displaymath}
 \Omega \preccurlyeq \Omega' \Longleftrightarrow \Omega \subset \adh(\Omega')
\end{displaymath}
\end{coro}

\begin{defi}\index{orbite}
 
En considérant la relation d'équivalence $\sim$ d'appartenance à la même
orbite, on définit l'espace quotient $X/G$ des orbites.
\end{defi}

\begin{prop}
 
L'ensemble $X/G$ peut être muni d'une topologie naturelle rendant l'application
$\pi$ continue et ouverte.
\begin{displaymath} \begin{array}{rrcl}
 \pi :&X &\longrightarrow& X/G \\
      &x &\longmapsto    & G\cdotp x
\end{array} \end{displaymath}
\end{prop}

\begin{proof}
 On construit la topologie sur $X/G$ en posant que $U\subset X/G$ est ouvert si
$\pi^{-1}(U)$ est ouvert pour la topologie sur $X$. On vérifie ensuite qu'on a
bien une topologie rendant $\pi$kjii ouverte (elle est continue par
construction).
\end{proof}

\begin{prop}
 $f : X/G \rightarrow Y$ est continue si et suelement si $f\circ
\pi:X\rightarrow Y$ est continue.
\end{prop}

\begin{proof}
 C'est une conséquence de la définition de la topologie définie sur $X/G$.
\end{proof}

\begin{defi}[Espace homogène]\index{espace homogène}
 
Un espace homogène est un espace topologique sur lequel un groupe topologique
agit topologiquement et transitivement.

En particulier il n'y a qu'une seule orbite et les stabilisateurs sont tous
conjugués.
\end{defi}

\begin{theo}
 Si $X$ est un espace topologique localement compact à base dénombrable, alors
pour tout $x \in X$ :
\begin{displaymath}
 X \text{ est isomorphe à } G/\Stab_G (x)
\end{displaymath}

\end{theo}


























\pagebreak
\section{Actions de $\mathbf{\GL(n,\K)}$ sur $\mathbf{\M(n,\K)}$}
\vspace{0.5em}

\subsection[Action par multiplication]{Action par multiplication à gauche et à
droite}
\vspace{0.5em}

\subsubsection{Action par multiplication à gauche}
\vspace{0.5em}

\index{pivot de Gauss}\index{groupe classique!$\GL_n(\K)$}
\index{groupe classique!$\M_n(\K)$}
On va montrer qu'étudier l'orbite de l'action par multiplication à gauche
revient à effectuer le pivot de Gauss.
\begin{displaymath}
 \begin{array}{rrcl}
     f :&   \GL_n(\K) \times \M_n(\K) & \longrightarrow & \M_n \\
        &(g,M)& \longmapsto & gM
       \end{array}
\end{displaymath}

On définit l'ensemble $\{E_{ij}\}$ des matrices élémentaires de $\M_n(\K)$ par
$(E_{ij})_{kl} = \delta_{ik} \delta_{jl} $.

\begin{defi}[Matrices de transvection et de dilatation]
\index{matrice!de transvection}\index{matrice!de dilatation}
\index{transvection|see{matrice de transvection}}
\index{dilatation|see{matrice de dilatation}}

 Soient $i\neq j$ et $\lambda \in \K$.
\begin{itemize}
 \item $T_{ij}(\lambda) = I_n + \lambda E_{ij}$ est une matrice de transvection,
 \item $D_i(\lambda) = I_n + (\lambda -1)E_{ii}$ est une matrice de dilatation
($\lambda \neq 0$).
\end{itemize}

Ce sont en particulier des matrices inversibles.
\end{defi}

\begin{example}[Remarque]\ 
\begin{itemize}
 \item L'action de la transvection $T_{ij}$ sur une matrice $M$ revient à
ajouter $\lambda$ fois la $j^{\text{ème}}$ ligne à la $i^{\text{ème}}$,
 \item L'action de la dilatation $D_i$ sur une matrice $M$ revient à multiplier
la $i^{\text{ème}}$ par $\lambda \neq 0$,
 \item $D_j(-1) \circ T_{ij}(1) \circ T_{ji}(-1) \circ T_{ij}(-1)$ permet
d'inverser les lignes $i$ et $j$. 
\end{itemize}
\end{example}


\begin{defi}[Forme echelonnée réduite]

Un matrice est dite sous forme échelonnée réduite si elle est de la forme
suivante :
 \begin{displaymath}\begin{pmatrix}
0\ \cdots\              0  &1      &*\cdots *&0&*\cdots *&0&*\cdots *\\
\ \vdots\ \cdots\ \vdots   &0      &0\cdots 0&1&*\cdots *&0&*\cdots *\\
\ \vdots\ \cdots\ \vdots   &\vdots &\vdots\ \cdots\ \vdots&0&0\cdots 0&1&*\cdots
*\\
\ \vdots\ \cdots\ \vdots   &\vdots &\vdots\ \cdots\ \vdots&\vdots&\vdots\
\cdots\  \vdots&0&0\cdots 0\\
   \end{pmatrix} \end{displaymath}
\end{defi}

\begin{lemm}
 
D'après les résultats sur le pivot de Gauss, dans chaque orbite de $\M_n(\K)$
sous l'action de $\GL_n(\K)$ il y a une matrice sous forme échelonnée réduite.

De plus cette matrice est unique dans chaque orbite.
\end{lemm}

\begin{proof}
 
On commence par remarquer que les matrices d'une même orbite ont les mêmes
noyaux.
En effet, si $M\in \M_n(\K)$ et $g\in \GL_n(\K)$ alors :
\begin{displaymath} x \in \Ker(gM) \Longleftrightarrow gM(x) = 0
\Longleftrightarrow M(x) = 0
\Longleftrightarrow x\in \Ker M\end{displaymath}

On motre ensuite qu'il n'existe qu'une forme échelonnée réduite pour un certain
noyau.
\end{proof}

\begin{coro}
 
$M,M' \in \M_n(\K)$ sont dans la même orbite si et seulement si $\Ker M = \Ker
M'$.
\end{coro}

\begin{coro}
 
L'ensemble des orbites et en bijection avec l'ensemble des sous-espaces
vectoriels de $\K^n$.
\end{coro}


\vspace{0.5em}
\subsubsection{Action de $\mathbf{\GL_n(\K)\times\GL_n(\K)}$ sur
$\mathbf{\M_n(\K)}$}
\vspace{0.5em}

\begin{displaymath}  \begin{array}{rrcl}
     f :&   (\GL_n(\K)\times\GL_n(\K))\times\M_n(\K)&\longrightarrow&\M_n(\K) \\
        &((g,h),M) &\longmapsto&gMh^{-1} \\
       \end{array}\end{displaymath}

On a vu que pour $M\in \M_n(\K)$, il existe $g\in \GL_n(\K)$ tel que $gM$ soit
echelonnée réduite.

On remarque que $T_{ij}^{-1}(\lambda) = T_{ij}(-\lambda)$ et que
$D_i^{-1}(\lambda) = D_i(1/\lambda)$. De plus l'action par transvection et
dilatation à droite revient à effectuer les opérations élémentaires sur les
colonnes.

Ainsi, par multiplication à droite, à partir d'une matrice échelonnée réduite,
on obtient une matrice de la forme suivante :
\begin{displaymath}\begin{pmatrix} I_r & 0 \\ 0 & 0
\end{pmatrix}\end{displaymath}

\begin{theo}

Les orbites de $\M_n(\K)$ sous l'action de $\GL_n(\K)\times\GL_n(\K)$ est en
bijection avec $\{1,\cdots,n\}$
\end{theo}

\begin{example}[Remarque]
 Tout ceci est un exemple d'une théorie plus générale sur les groupes de Lie.
\end{example}


\subsection{Action par conjugaison}
\vspace{0.5em}

On considère l'action liée au changement de base :
\begin{displaymath} \begin{array}{rrcl}
   f:&   \GL_n(\K)\times \M_n(\K)&\longrightarrow&\M_n(\K) \\
      &     (g,M)&\longmapsto&gMg^{-1} \\
       \end{array}\end{displaymath}

On peut voir cette action comme la restriction à la diagonale de l'action de
$\GL_n(\K)\times\GL_n(\K)$. En particulier, le rang est encore un invariant.
\begin{displaymath} \GL_n(\K) \simeq \{(g,g) \tq g \in \GL_n(\K)\}
\hookrightarrow\GL_n(\K)\times\GL_n(\K)\end{displaymath}

\begin{lemm}
 
Le polynôme caractéristique est un invariant de l'action par conjugaison.
\end{lemm}
\begin{proof} Si $M \in \M_n(\K)$ son polynôme caractéristique est $\chi_M(X) =
\det M-XI_n$. Si $g\in\GL_n(\K)$ alors :
\begin{displaymath}\chi_{gMg^{-1}}(X) = \det (gMg^{-1} - XI_n) = \det
\left(g\left(M-XI_n\right)g^{-1}\right) = \det g \cdotp \chi_M(X) \cdotp \det
g^{-1} = \chi_M(X)\end{displaymath}
\end{proof}

\begin{lemm}
 
Le polynôme minimal est un invariant de l'action par conjugaison.
\end{lemm}

\begin{proof}
On rappelle que le  polynôme minimal $P_M$ de $M\in \M_n(\K)$ est le polynôme
unitaire de degré minimal tel que $P_M(M)=0$.

On remarque que pour $(g,M) \in \GL_n(\K) \times \M_n(\K)$ on a :
\begin{displaymath}(gMg^{-1})^2 = gMg^{-1}gMg^{-1} = gM^2g^{-1} \quad \text{
d'où }\quad \forall
k \in \N, (gMg^{-1})^k = gM^kg^{-1} \end{displaymath}

Ainsi pour tout polynôme $P$ on a $P(gMg^{-1}) = gP(M)g^{-1}$ et en particulier
$P_M = P_{gMg^{-1}}$.
\end{proof}


\begin{example}[Remarque]
Sur $\C$ tout polynôme est scindé, donc via le lemme des noyaux, toute matrice
est trigonalisable.
\end{example}

\begin{theo}[Réduction de Jordan]

L'ensemble des orbites de $\M_n(\C) / \GL_n(\C)$ est en bijection avec
l'ensemble des matrices de la forme suivante (modulo permutation des blocs) :
\begin{displaymath}\begin{pmatrix} J_{\lambda_1} & & 0 \\ & \ddots & \\ 0 & &
J_{\lambda_k} 
\end{pmatrix} \quad \text{ avec } \quad J_{\lambda_i} =\begin{pmatrix}
\lambda_i & 1      &        & 0    \\
          & \ddots & \ddots &      \\
          &        & \ddots & 1    \\
       0   &        &        & \lambda_i\\  \end{pmatrix}\end{displaymath}
\end{theo}

\begin{prop}
 Les orbites de $\GL_2(\R)$ agissant sur $\M_2(\R)$ par conjugaison sont
représentés par les matrices :
\begin{align}
&   \{\lambda \Id \tq \lambda \in \R\} \\
& \left\lbrace\begin{pmatrix} \lambda & 0 \\ 0&  \mu \end{pmatrix} \Btq
\lambda,\mu \in \R, \lambda \neq\mu \right\rbrace \\
& \left\lbrace\begin{pmatrix} \lambda & 1 \\ 0&  \lambda \end{pmatrix} \Btq
\lambda\in \R \right\rbrace \\
& \left\lbrace\lambda R_\theta\tq \lambda\in \R, \theta \in \R \- \pi\Z
\right\rbrace \text{ où } R_\theta = \begin{pmatrix}
\cos \theta & \sin \theta \\ -\sin\theta & \cos\theta \end{pmatrix}
\end{align}

\end{prop}



\pagebreak
\section{Etude des sous-groupes classiques}
\vspace{0.5em}

On applle groupe classique tout sous-groupe de $\GL_n$ fermé et défini en
famille.

\subsection[$\mathbf{\SL_n(\K)}$]{$\mathbf{\SL_n(\K)}$ avec $\mathbf{\K = \R
\text{ ou }\C}$}
\vspace{0.5em}

\begin{defi}[$\SL_n(\K)$]
 \begin{displaymath} \SL_n(\K) := \{M\in \GL_n(\K) \tq \det M = 1\}
\end{displaymath}

En particulier on a $\SL_n(\R) \subset \GL_n^+(\R)$.
\end{defi}

On définit l'homéomorphisme suivant :
\begin{displaymath}  \begin{array}{rrcl}
    \sigma :&\GL_n(\K) &\longrightarrow&\SL_n(\K) \times \K\* \\
       &     g &\longmapsto    & \left(g \begin{pmatrix}
                                 (\det g)^{-1} &    &        &  \\
                                               & 1  &        &  \\
                                              &    & \ddots &  \\
                                               &    &        & 1
                                \end{pmatrix},\ \det g\right)
   \end{array}\end{displaymath}


\begin{prop}
 \begin{enumerate}
  \item $\SL_n(\C)$ est connexe par arc.
  \item Si $\GL_n^+(\R)$ est connexe par arc, alors $\SL_n(\R)$ l'est aussi.
 \end{enumerate}
\end{prop}

\begin{proof}
On définit la projection : $p : \SL_n(\C)\times \C\* \owns (A,t) \longmapsto A
\in \SL_n(\C)$.

Ainsi on a $p\circ\sigma(\GL_n(\C)) = \SL_n(\C)$.

De même la restriction $\sigma' := \sigma\big|_{\GL_n^+(\R)}$ est aussi un
homéomorphisme et $p\circ \sigma'(\GL_n^+(\R)) = \SL_n(\R)$.
\end{proof}

\subsection{Groupes associés à une forme sesquilinéaire}
\vspace{0.5em}

\begin{defi}[Forme sesquilinéaire]

Soit $E$ un $\C$-espace vectoriel de dimension finie $n$.
Un application $<\cdotp,\cdotp>\ : E\times E \longrightarrow E$ est une forme
sesquilinéaire si :
\begin{enumerate}
 \item $<x,\cdotp>\ : E \longrightarrow E$ est linéaire,
  \item $\forall x,y,z \in E,\ <x+y, z>\ =\ <x,z>\ +\ <y,z>$,
   \item $\forall \lambda \in \C,\ \forall x,y\in E, \ <\lambda x,y>\ =
\overline{\lambda} \ <x,y>$
\end{enumerate}
\end{defi}

\begin{defiprop}[Groupe des invariants $\mathcal{U}(E)$]
 
\begin{displaymath} \mathcal{U}(E) := \{f : E \longrightarrow E\text{ linéaire
bijective } \tq
<f(x),f(y)> \ = \ <x,y>\ \} \end{displaymath}

$\mathcal{U}(E)$ est le sous-groupe des invariants de $<\cdotp,\cdotp>$ dans
$\GL(E)$. C'est un sous-groupe fermé de $\GL(E)$ isomorphe à un sous-groupe de
$\GL_n(\C)$.

\end{defiprop}

\begin{proof}
Si $f,g \in \mathcal{U}(E)$ on vérifie facilement que $f\circ g$, $f^{-1}$, et
$\Id_E$ sont également dans $\mathcal{U}(E)$.

En fixant une base $\mathcal{B} = (e_1, \cdots, e_n)$ on peut identifier
$\GL(E)$ à $\GL_n(\C)$ par $\Mat_{\mathcal{B}} : \GL(E)\longrightarrow
\GL_n(\C)$. Soit alors $\Omega = (<e_i,e_j>) \in \M_n(\C)$, on a $<x,y>\ =
X\*\Omega Y$. Si $M = \Mat_\mathcal{B}(f)$ alors :
\begin{displaymath}\begin{array}{rl}
                    & f\in \mathcal{U}(E) \\
\Longleftrightarrow & X\*\Omega Y = (MX)\*\Omega(MY) \\
\Longleftrightarrow & X\*\Omega Y = X\*M\* \Omega MY \\
\Longleftrightarrow & \Omega = M\*\Omega M \\
  \end{array}
\end{displaymath}

La dernière implication est bien vérifiée car si $A=(a_{kl}), X=e_i, Y=e_j$
alors $X\*AY = a_{ij}$. Ainsi :
\begin{displaymath}\mathcal{U}(E) \simeq \{M\in \GL_n(\C) \tq M\*\Omega M -
\Omega = 0\} \end{displaymath}

C'est en particulier l'image réciproque du fermé $\{0\}$ par un application
continue, c'est donc un sous-groupe fermé.
\end{proof}

\begin{defiprop}[Groupe unitaire $\U(n)$]

Si la forme $<\cdotp,\cdotp>$ est hermitienne (ie : $\forall x,y \in E,\
<y,x>\ =\ \overline{<x,y>}$), on note alors $\U(n,\C)$ ou $\U(n)$.
\begin{displaymath} \U(n) = \{ M \in \GL_n(\C) \tq M\*M = \Id\}
\end{displaymath}
\end{defiprop}

\begin{prop}
 
$\U(n)$ est compact et connexe par arcs.
\end{prop}

\begin{proof}\ 

On a vu que $\U(n)$ est fermé. De plus $\forall M\in\U(n), \|M\f^2 = \Tr
(M\*M) = n$. Ainsi $\U(n)$ est fermé borné, donc compact.

On montre la connexité par arcs en utilisant la théorie de la réduction des
endomorphismes. Pour $M\in\U(n)$, il existe une base orthonormée telle que $M$
est diagonale. On peut alors construire le chemin continu.
\begin{displaymath}\exists P\in\U(n), P\*MP =
\begin{pmatrix}         \lambda_1 & &0\\
                          & \ddots & \\
                      0 & &\lambda_n \end{pmatrix}\end{displaymath}
\end{proof}

\begin{defi}[Groupe spécial unitaire]
 
\begin{displaymath} \SU(n) := \U(n) \cap \SL_n(\C) \end{displaymath} 
\end{defi}


\begin{example}[Remarque]
 En remarquant que $(M\*M=\Id) \Longrightarrow (|\det M| = 1)$, on peut définir
l'homéomorphisme :
\begin{displaymath} \sigma\big|_{\U(n)} : \U(n) \longrightarrow \SU(n)\times
\S^1 \end{displaymath}

Ainsi $\SU(n)$ est également compact et connexe par arcs.
\end{example}


\vspace{0.5em}
\subsection{Groupes associés à une forme bilinéaire symétrique}
\vspace{0.5em}

\begin{defi}[Forme bilinéaire symétrique - antisymétrique]
 
Si $E$ est un $\K$-espace vectoriel de dimension fini, et $<\cdotp,\cdotp>$ un
forme bilinéaire, elle est dite :
\begin{itemize}
 \item Symétrique si : $\forall x,y \in E,\ <y,x>\ =\ <x,y>$
 \item Antisymétrique si : $\forall x,y \in E,\ <y,x>\ = -<x,y>$
\end{itemize}
\end{defi}

Comme dans le paragraphe précédant, en remplaçant $M\*$ par $^tM$ on montre
que $\O(E)$, le sous-groupe de $\GL(E)$ qui laisse une forme bilinéaire
invariante, est isomorphe à un sous-groupe fermé de $\GL_n(\K)$.

On pose $\Omega = (<e_i,e_j>) \in \M_n(\K)$. En particulier si la forme est
symétrique, $\Omega$ est symétrique.

\begin{prop}[$\O(n)$]
\begin{displaymath}\O(n) \simeq \{ M \in \GL_n(\K) \tq \ ^tM\Omega M - \Omega =
0 \} \end{displaymath}
\end{prop}

\begin{defi}[Groupe orthogonal et groupe spécial orthogonal complexe]
 
Si $\K = \C$ et que la forme n'est pas dégénérée, on peut supposer $\Omega =
\Id$. On définit le groupe orthogonal complexe :
\begin{displaymath}\O(n,\C) = \{ M \in \M_n(\C) \tq \ ^tM M = \Id
\}\end{displaymath}

En remarquant que $(^tMM=\Id) \Longrightarrow (|\det M| = 1)$, on définit
également le groupe spécial orthogonal complexe :

\begin{displaymath}\SO(n,\C) = \O(n,\C) \cap \SL(n,\C) \end{displaymath}
\end{defi}

\begin{defi}[Groupe orthogonal et groupe spécial orthogonal réel]
 
Si $\K = \R$ on ne peut plus supposer que $\Omega = \Id$. On définit le groupe
orthogonal réel :
\begin{displaymath}\O(n) := \O(n,\R) = \{ M \in \M_n(\R) \tq \ ^tM M = \Id
\}\end{displaymath}

On définit de même le groupe spécial orthogonal réel :

\begin{displaymath}\SO(n) := \SO(n,\R) = \O^+(\R) = \O(n) \cap \SL(n,\R)
\end{displaymath}
\end{defi}

\begin{example}[Remarque]
 $\O(n) = \O^+(\R) \sqcup \O^-(\R)$ (union disjointe de fermés de $\O(n)$).
\end{example}

\begin{prop}
 \begin{enumerate}
  \item $\O(n)$ est compact,
  \item $\SO(n)$ est connexe par arcs,
  \item $\O^+(\R) \sqcup \O^-(\R)$ est la décomposition en composantes connexes.
 \end{enumerate}
\end{prop}

\begin{proof}\

\begin{enumerate}
 \item $\O(n) = \U(n) \cap \M_n(\R)$, en particulier, c'est un sous-groupe
fermé du compact $\U(n)$, c'est donc un compact.
 \item \begin{displaymath} \ ^tPMP = \begin{pmatrix}
                     I_p &      &             &        & \\
                         & -I_q &             &     0   & \\
                         &      & R_{\theta_1}&        & \\
                         &    0 &             & \ddots & \\
                         &      &             &        & R_{\theta_i} \\
                    \end{pmatrix} = \begin{pmatrix}
                     I_p &             &        & \\
                         &             &     0   & \\
                         & R_{\theta_1}&        & \\
                         &         0    & \ddots & \\
                         &             &        & R_{\theta_r} \\
                    \end{pmatrix} \end{displaymath}
 $M \in \SO(n)$ donc $M$ est diagonalisable par blocs de taille au plus $2$.
De plus $(\det M = 1) \Rightarrow (q\text{ est pair})$, et on peut remplacer
$I_2$ par $R_{\theta_\pi}$.

On peut construire un chemin continu $\tilde{\gamma} : [0,1] \owns t
\longmapsto R_{t\theta}$ tel que $\tilde{\gamma}(0) = \Id$ et
$\tilde{\gamma}(1) = R_\theta$. On en déduit un chemin continu $\gamma$ de $\Id$
à $M$.

\end{enumerate}
\end{proof}


\vspace{0.5em}
\subsection{Groupes associés à une forme bilinéaire antisymétrique}
\vspace{0.5em}

On suppose que la forme est non-dégénérée, ce qui implique que $\dim E = 2n$.
Alors dans une certaine base on a :
\begin{displaymath}\Omega = \begin{pmatrix}
            0 & \Id_n \\ -\Id_n & 0
           \end{pmatrix}
 \quad \text{ ou bien } \quad \begin{pmatrix}
                               &     &  &  &     & -1 \\
                               &    0 &  &  & \cdotp   &    \\
                               &     &  &-1&     &     \\
                               &     &1  &  &     &  \\
                               &\cdotp&  &  &  0  &    \\
                              1&     &  &&     &     \\
                              \end{pmatrix} \end{displaymath}


\begin{defi}[Groupe symplectique]
 
On appelle groupe symplectique :
\begin{displaymath} \Spl_{2n} := \{ M\in \M_n(\K) \tq ^tM\Omega M - \Omega = 0\}
\end{displaymath}
On a toujours $(\M \in \Spl_{2n}) \Rightarrow (|\det M| = 1)$, on définit ainsi
le
groupe spécial symplectique :
\begin{displaymath} \USp_{2n} := \Spl_{2n} \cap
\mathcal{U}(2n)\end{displaymath} 
\end{defi}

\begin{prop}
\begin{enumerate}
 \item $\USp_{2n}$ est compact,
 \item $\USp_{2n} = \left\lbrace z \in \mathcal{U}(2n)\ \Big|\ \exists
A,B\in\M_n{\C}, z = \begin{pmatrix}A&-B\\B&A\end{pmatrix}\right\rbrace.$

\end{enumerate}
\end{prop}

\begin{proof}\ 

\begin{enumerate}
 \item C'est un fermé de $\U(n)$, donc un compact.
 \item
\begin{displaymath}\begin{array}{ccccc} \begin{array}{rl}
                    & z\in \mathcal{U}(2n) \\
\Leftrightarrow & z\*z = \Id \\
\Leftrightarrow & z\* = z^{-1} \\
\Leftrightarrow & ^tz = \overline{z^{-1}}
   \end{array} &  \text{ et }  & 
\begin{array}{rl}
                    & z \in \Spl(2n) \\
\Leftrightarrow & ^tz\Omega z = \Omega \\
                    & z \in \USp(2n) \\
\Leftrightarrow & \overline{z^{-1}}\Omega z = \Omega \\
\Leftrightarrow & z^{-1}\Omega \overline{z} = \Omega \\
\Leftrightarrow & \Omega \overline{z} = z \Omega \\
\end{array}&  \text{ donc }  & 
\end{array} \end{displaymath}
\begin{displaymath}
\begin{array}{rl}
                    & z=\begin{pmatrix}   A&B\\
C&D\end{pmatrix}\in \USp(2n) \\
\Leftrightarrow & \begin{pmatrix}0&I_n\\-I_n&0 \end{pmatrix}
\begin{pmatrix}\overline{A}&\overline{B}\\ \overline{C}&\overline{D}
\end{pmatrix} = \begin{pmatrix}A&B\\C&D \end{pmatrix}
\begin{pmatrix}0&I_n\\-I_n&0 \end{pmatrix}  \\
\Leftrightarrow & \begin{pmatrix}\overline{C}&\overline{D}\\
-\overline{A}&-\overline{B}
\end{pmatrix} = \begin{pmatrix}A&B\\C&D \end{pmatrix} \\
\Leftrightarrow& C = -\overline{B} \text{ et } D = \overline{A} \\
\end{array} \end{displaymath}
\end{enumerate}

\end{proof}


\vspace{0.5em}
\subsection{Exemples en petites dimensions}
\vspace{0.5em}

\begin{itemize}
 \item $\O(1) = \{\pm 1\}$ et $\O^+(1) = \{1\}$,
 \item $\U(1) = \S^1$ et $\S\U(1) = \{1\}$,
 \item $\SO(2) \simeq \U(1)$
\end{itemize}









\pagebreak

\section{Décomposition polaire et applications}
\vspace{0.5em}

\begin{defi}[Matrice hermitienne]
\index{matrice!hermitienne}
\index{matrice!positive}
\index{matrice!positive!définie positive}

Soit $M\in \M_n{\C}$. On dit que $M$ est hermitienne si $M\*=M$ et on note
$\H_n$ leur ensemble.

On dit que $M\in\H_n$ est positive (resp. définie positive) si pour tout
vecteur colonne $X$ on a $X\*MX \geq 0$ (resp. $X\*MX > 0$) et on note $\Hp$
(resp. $\Hpp$) leur ensemble.
\end{defi}

\begin{example}[Remarque]
 Cette définition a bien un sens car $M\in\H_n \Rightarrow X\*MX \in \R$.
En effet :
\begin{align*}
 & \overline{X\*MX} & &= \t X\overline{M}\overline{X} & & =
\t (\t X\overline{M}\overline{X}) & & \text{(car de taille }(1,1) \text{)} \\
 &                & & = X\*MX & & =X\*M\*X & & \text{(car } M\* = M \text{)}\\
\end{align*}
\end{example}

\begin{defi}[Matrice symétrique]
\index{matrice!symétrique}
 
On note $\S_n$ l'ensemble des matrices symétriques (réelles). On a :
\begin{align*}
\S_n &= \H_n \cap \M_n(\R) \\
\Sp_n &= \Hp_n \cap \M_n(\R) \\
\Spp_n &= \Hpp_n \cap \M_n(\R) 
\end{align*}
\end{defi}

\begin{theo}[Décomposition polaire 1]
\label{theo_decomposition_polaire_1}
\index{décomposition polaire!premier théorème}
\begin{enumerate}
 \item Pour toute matrice $A\in \GL_n(\C)$, il existe une unique matrice
unitaire $U\in\U(n)$ et une unique matrice hermitienne définie positive
$P\in\Hpp_n$ telles que : $A = UP$.
 \item Pour toute matrice $A\in\GL_n(\R)$, il existe une unique matrice
orthogonale $U\in\O(n)$ et une unique matrice symétrique définie positive
$P\in\Spp(n)$ telles que : $A=UP$.
\end{enumerate}
\end{theo}

\begin{lemm}[Racine carrée]
\label{lemm_racine_carree}
\index{racine carrée}

Soit $M\in\Hp(n)$. Il existe une unique matrice $N\in\Hp(n)$ telle que
$N^2=M$.\\ De plus $M\in\Hpp(n) \Longleftrightarrow \N\in\Hpp(n)$.

On a un énoncé équivalent avec les matrices symétriques.
\end{lemm}

\begin{proof}[du lemme \ref{lemm_racine_carree}]\ 

\begin{description}
 \item [(existence)]
Soit $M\in\Hp(n)$, il existe $(d_1,\cdots,d_n)\in(\R^+)^n$ et $P\in\U(n)$
telles que $M=P^{-1}\Diag(d_1,\cdots,d_n)P$. Les $d_i$ sont bien des réels
positifs car si $X\neq 0$ tel que $MX=d_iX$ alors $X\*MX=d_iX\*X=d_i\|X\|^2\geq
0$ car $M\in\Hp(n)$.

\item [(unicité)]
Si $N^2=M$ alors $NM=MN=N^3$, or si deux matrices commutent, elles sont
diagonalisables dans une même base.
\begin{displaymath}
 N = P^{-1}\begin{pmatrix}
    \lambda_1 & & 0 \\  &\ddots& \\  0 & & \lambda_n
           \end{pmatrix} P \quad \text{ et } \quad 
 M = P^{-1}\begin{pmatrix}
    d_1 & & 0 \\  &\ddots& \\  0 & & d_n\\
           \end{pmatrix} P \quad 
\end{displaymath}

Donc $N^2=M\Longleftrightarrow \forall i,\lambda_i^2=d_i$. Comme $N\in\Hp(n)$
les $\lambda_i$ sont positifs donc $d_i = \sqrt{\lambda_i}$ d'où l'unicité.

\item [(cas réel)] La preuve est la même
\end{description}
\end{proof}

\begin{proof}[du théorème \ref{theo_decomposition_polaire_1}]\

\begin{enumerate}
 \item \begin{description}
        \item[(existence)] Soit $A\in\GL_n(\C)$. On remarque que
$(A\*A)\*=A\*A$,
ie $A\*A\in \H_n$. De plus pour tout vecteur colonne $X$ non nul, on a $X\*A\*AX
= (AX)\*(AX) = \|AX\|^2 > 0$ (car $A$ invesible). Ainsi $A\*A\in\Hpp(n)$.
D'après le lemme, il existe $P\in\Hpp(n)$ telle que $P^2=A\*A$. Il reste à
vérifier que $U := AP^{-1} \in\U(n)$.

En remarquant que $(AP^{-1})\*(AP^{-1}) = (P^{-1})\*A\*AP^{-1} =
(P\*)^{-1}A\*AP^{-1}$.\\ On a $U\*U = P^{-1}A\*AP = P^{-1}P^2P^{-1} = \Id$ donc
$U\in\U(n)$.
\item[(unicité)] Si $A=UP$ avec $U\in\U(n)$ et $P\in \Hpp(n)$ alors :\\
$A\*A=(UP)\*(UP) = P\*U\*UP = P\*P = P^2$. L'unicité de la racine de $A\*A$
implique l'unicité de $P$ qui implique celle de $U$.
       \end{description}
\item \begin{description}
       \item [(cas réel)] La preuve est la même
      \end{description}
\end{enumerate}
\end{proof}

\begin{example}[Exemple]
 En dimension 1, cette décomposition est exactement la forme dite polaire d'un
nombre complexe.
\end{example}

\begin{defi}[Exponentielle]
\index{exponentielle}
\begin{displaymath}
\begin{array}{rrcl} \Exp : &          \M_n(\R\text{ (resp. }\C\text{)})
&\longrightarrow& \M_n(\R\text{ (resp. }\C\text{)}) \\
   & A &\longmapsto& \displaystyle\sum_{k=0}^{\infty} \dfrac{A^k}{k!}   
\end{array}
\end{displaymath}
\end{defi}

\begin{prop}[Continuité de l'exponentielle]
 
$\Exp : \M_n(\C) \longrightarrow \M_n(\C)$ est une application continue (en
particulier, elle est bien définie).
\end{prop}

\begin{proof} \
 
$\M_n(\C)$ est une algèbre de Banach (un espace vectoriel normé complet, et une
algèbre telle que $\|AB\| \leq \|A\|.\|B\|$ ce qui est le cas par exemple avec
la norme de Frobénius).

La série de fonctions $\sum \frac{A^n}{n!}$ est normalement convergente sur
toute boule de $\M_n(\C)$. En effet :
\begin{displaymath}
\forall r > 0,\ \forall A \in \Boule(0,r)
\subset \M_n(\C),\ \sum \dfrac{\|A^k\|}{k!} \leq  \sum\dfrac{\|A\|^k}{k!} =
\exp({\|A\|}) \leq \exp({r}) 
\end{displaymath}

En particulier, sur chaque boule, la série $\sum \dfrac{A^k}{k!}$ est
uniformément convergente, ainsi $\forall A \in\M_n(\C),\ \Exp(A)$ est bien
définie, et $\Exp$ est limite uniforme de fonctions continues donc $\Exp$ est
continue.
\end{proof} 

\begin{prop}
\begin{enumerate}
 \item $\Exp(0) = \Id$.
 \item Si $A,B\in\M_n(\C)$ commutent, alors $\Exp(A+B) = \Exp(A)\Exp(B)$.
 \item Si $A\in\M_n(\C)$, alors $\Exp(A) \in\GL_n(\C)$ et $\Exp(A)^{-1} =
\Exp(-A)$.
\end{enumerate}
\end{prop}

\begin{proof}\

\begin{enumerate}
 \item $\Exp(0) = O^0 + \frac{1}{2}O^1 + \cdots = \Id$ par convention.
 \item Comme
$\displaystyle\sum_{m,p\in\N}\dfrac{\|A^mB^p\|}{m!p!}$ converge, la
famille $\left( \dfrac{A^m}{m!}\times\dfrac{B^p}{p!} \right)_{m,p\in \N}$ est
absolument sommable : \\
\begin{align}
 \sum_{m,p\in\N}\dfrac{A^mB^p}{m!p!} & & &= \sum_{m\in\N}\dfrac{A^m}{m!} \times
\sum_{p\in\N}\dfrac{B^p}{p!} & & = \Exp(A)\Exp(B)\\
 \sum_{m,p\in\N}\dfrac{A^mB^p}{m!p!} & & &=
\sum_{k\in\N} \sum_{p+q=k}\dfrac{A^p}{p!}\dfrac{B^q}{q!}
= \sum_{k\in\N} \left( \dfrac{1}{k!} \right) \underbrace{
\sum_{p+q=k}\dfrac{(p+q)!}{p!q!}A^pB^q}_{=(A+B)^k \text{ (Newton)}}
& &= \Exp(A+B)
\end{align}
\item $A$ et $(-A)$ commutent dont $\Exp(A)\Exp(-A) = \Exp(A-A) = \Exp(0) =
\Id$.
\end{enumerate}

\end{proof}

\begin{prop}
\index{matrice!anti-symétrique}
\index{matrice!anti-hermitienne}
\index{matrice!nilpotente}
\index{matrice!unipotente}
 \begin{enumerate}
  \item $\Exp(A\*) = \Exp(A)\*$ et $\Exp(\t A) = \t \Exp(A)$.
  \item $\forall P\in\GL_n(\C),\ \Exp(PAP^{-1}) = P\Exp(A)P^{-1}$.
  \item Si $A$ est symétrique (resp. hermitienne), alors $\Exp(A)$ est
symétrique (resp. hermitienne) définie positive.
  \item Si $A$ est anti-symétrique (resp. anti-hermitienne), alors $\Exp(A)$ est
orthogonale (resp. unitaire).
  \item Si $A$ est diagonale (resp. triangulaire), alors $\Exp(A)$ est
diagonale (resp. triangulaire).
  \item Si $A$ est nilpotente alors $\Exp(A)$ est unipotente.
  \item $\lambda\in\C\*$ est valeur propre de $A$ si et seulement si
$\e^\lambda$ est valeur propre de $\exp(A)$. De plus les multiplicités sont les
mêmes.
 \end{enumerate}
\end{prop}

\begin{example}[Rappels] \

$A$ est anti-symétrique (resp. anti-hermitienne) si $\t A = -A$ (resp. $A\* =
-A$).

$A$ est nilpotente (resp. unipotente) si il existe $m\in\N$ tel que $A^m = 0$
(resp. $(A-\Id)^m =0$).
\end{example}

\begin{proof}\

 \begin{enumerate}
  \item (exo)
  \item (exo)
  \item Si $\t A = A$ alors $A$ est diagonalisable. Donc il exsite $P\in\GL_n$
tel que $A = P\Diag(\lambda_1, \cdots, \lambda_n) P^{-1}$. Ainsi $\Exp(A) =
P\Diag(\exp(\lambda_1), \cdots, \exp(\lambda_n))P^{-1}$ qui est définie
positive.
\item Si $\t A = A$ alors $\Exp(\t A)\Exp(A) = \Exp(-A +A) = \Id$. Le
raisonnement est identique sur $\C$ pour la conjugaison.
 \item $A$ triangulaire $\Rightarrow \forall m\in\N,\ A^m$ triangulaire
$\Rightarrow \Exp(A)$ triangulaire.
 \item Si $A^m = 0$, $\Exp(A) = \sum_{p=0}^{m-1}\frac{A^p}{p!} = \Id +
A\sum_{p=1}^{m-1}\frac{A^p}{p!}$ donc $(\Exp(A)-\Id)^m =
A^m(\sum_{p=1}^{m-1}\frac{A^p}{p!})^m = 0$.
 \item[...] Remarque : les matrices strictement triangulaires supérieures sont
nilpotentes.
 \item Si $AX = \lambda $ alors $\Exp(A)X = \sum_{m\in\N}\frac{A^mX}{m!}
 = \sum_{m\in\N}\frac{\lambda^m X}{m!} = \exp(\lambda) X$. Il reste à montrer
que les multiplicités sont conservées.\\
On peut se ramener à l'étude des blocs de Jordan car $\Exp(PMP^{-1}) =
P\Exp(M)P^{-1}$. Soit alors $J_\lambda = \lambda \Id + K_n$ un bloc de Jordan.
on va montrer que la seule valeur propre de $\Exp(J_\lambda)$ est
$\exp(\lambda)$.
\begin{displaymath}
K_n^m = \begin{pmatrix}
 0      & \cdots  & 0      & 1      & 0      & \cdots &  0       \\
 \vdots & \ddots  &        & \ddots & \ddots & \ddots &  \vdots  \\
 \vdots &         & \ddots &        & \ddots & \ddots &  0       \\
 \vdots &         &        & \ddots &        & \ddots &  1       \\
 \vdots &         &        &        & \ddots &        &  0       \\
 \vdots &         &        &        &        & \ddots &  \vdots  \\
 0      & \cdots  & \cdots & \cdots & \cdots & \cdots &  0       \\
 \end{pmatrix}
   \Exp(K_n) =  \begin{pmatrix}
 1      & 1       & 1/2    & 1/6    & \cdots & 1/(n-1)! \\
 0      & \ddots  & \ddots & \ddots & \ddots & \vdots\hfill \\
 \vdots &         & \ddots & \ddots & \ddots & 1/6 \hfill  \\
 \vdots &         &        & \ddots & \ddots & 1/2 \hfill \\
 \vdots &         &        &        & \ddots & 1 \hfill \\
 0      & \cdots  & \cdots & \cdots & 0      & 1 \hfill \\

 \end{pmatrix}
\end{displaymath}
Comme $\Id$ et $K_n$ commutent, on a $\Exp(J_\lambda) =
\exp(\lambda)\Id\Exp(K_n)$.
 \end{enumerate}
\end{proof}

\begin{prop}\index{exponentielle}\index{logarithme}
 
$\Exp : \H_n \rightarrow \Hpp_n$ est un homémorphisme (de même pour  
$\Exp : \S_n \rightarrow \Spp_n$).
\end{prop}

\begin{proof}
 On a déjà vu que $\Exp(\H_n) \subset \Hpp_n$.
 
 \begin{description}
  \item [(injectivité)] On a déjà vu que $\lambda_i$ est une valeur propre de
$M$ de multiplicité $n_i$ si et seulement si $\exp(\lambda_i)$ est valeur propre
de $\Exp(M)$ de même multiplicité $n_i$. De plus $x$ est vecteur propre de $M$
si et seulement s'il est vecteur propre de $\Exp(M)$.\\
Dans notre cas $M\in\H_n$ est diagonalisable et $\Exp(M)\in\Hpp_n$ l'est aussi
dans une même base. Soit alors $\Exp(M) = \Diag(\mu_1,\cdots,\mu_n)$, tous
les $\mu_i$ étant positifs on peut écrire $\mu_i = \exp{\lambda_i}$ de manière
unique (car $\exp : \R \rightarrow \R\*_+$ est bijective). Ce qui prouve
l'unicité de $M = \Diag(\lambda_1,\cdots,\lambda_n)$.

\item [(surjectivité)] Soit $M \in\Hpp_n$. Il existe $U\in\U(n)$ telle que
$U\*MU=\Diag(\lambda_1,\cdots,\lambda_n)$. Comme $\lambda_i > 0$ on peut
considérer $\ln(\lambda_i)$. On pose $N = \Diag(\ln(\lambda_n),
\cdots, \ln(\lambda_1))$.\\
Donc $M = U\Diag(\exp(\ln(\lambda_n)), \cdots, \exp(\ln(\lambda_1)))U\*
 = U\Exp(N)U\* = \Exp(UNU\*) \in\Ima(\Exp)$.

\item [(bicontinuité)] $\Exp$ étant bijective (et continue) on peut
définir sa réciproque $\Log : \Hpp_n \rightarrow \H_n$. Il reste donc à montrer
que $\Log$ est continue.
\begin{itemize}
 \item[(choix de la norme)] On considère la norme suivante sur $\M_n(\C)$ :
$\|M\|
= \Sup_{\|x\| = 1} \|Mx\|$ avec $\|x\|^2=\sum(x_i^2)$.\\
En particulier, si $M\in\H_n$, $M$ est diagonalisable dans une base
orthonormale de vecteurs propres $(e_1,\cdots,e_n)$ donc :
\begin{displaymath}
 \|Mx\|^2 \leq \sum|\lambda_i|^2 |x_i|^2 \leq \sum (\Sup|\lambda_i|^2)x_i^2 \leq
(\Sup |\lambda_i|^2) \|x\|^2 \Longrightarrow \|M\|\leq
\Sup_{\lambda_i\in\Spectre(M)}|\lambda_i|
\end{displaymath}
En considérant $\lambda_j$ la valeur propre maximale et $e_j$ le vecteur propre
associé, on a :
\begin{displaymath}
 \|Me_j\| = |\lambda_j| \Longrightarrow \|M\|\geq
\Sup_{\lambda_i\in\Spectre(M)}|\lambda_i| \text { et donc : } \|M\| =
\Sup_{\lambda_i\in\Spectre(M)}|\lambda_i|
\end{displaymath}
\item[(définition d'une suite)] On considère une
suite $(H_p)_p \subset \Hpp_n$ convergente vers $H \in \Hpp_n$. Cette suite est
donc bornée et il existe $\rho_+$ tel que toutes les valeurs propres de
$H_p$ sont majorées par $\rho_+$. Comme l'inversion d'une matrice invesible est
un homéomorphisme, on applique le même raisonnement à la suite convergente
$(H_p^{-1})_p$ et on construit $\rho_-$ qui majore toutes les valeurs propres
de $H_p^{-1}$.\\ En remarquant que $\Spectre(H_p^{-1})=\{\lambda^{-1} \t q
\lambda\in\Spectre(H_p)$ on déduit que toute valeur propre de $H_p$ appartient
à l'intervalle $[1/\rho_-;\rho_+]$.
\item [(suite image)] On définit la suite des images $(\Log(H_p))_p$. Comme
$\Spectre(\Log(H_p)) = \ln \Spectre(H_p)$ on déduit $\|\Log(H_p)\|\leq \Max
\{\ln(1/\rho_-) , \ln(\rho_+)\}$.\\
En particulier la suite des images est bornée et admet une valeur d'adhérence
$M=\lim H_{p_k}$. On conclut en montrant l'unicité de cette valeur d'adhérence
:
\begin{displaymath}
 \Exp(M) = \Exp (\lim \Log(H_{p_k})) = \lim \Exp(\Log(H_{p_k}))
 = \lim H_{p_k} = \lim H_p = H
\end{displaymath}
\item [(remarque)] La valeur d'adhérence $M$ est bien dans $\H_n$ car c'est un
fermé de $\M_n(\C)$.
\end{itemize}
 \end{description}
\end{proof}

\begin{theo}[Décomposition polaire 2]
\index{décomposition polaire!second théorème}

Les applications suivantes sont des homéomorphismes :
\begin{align*}
 \H_n\times\U(n) & \longrightarrow \GL_n(\C) &
 \S_n\times\O(n) & \longrightarrow \GL_n(\R) \\
 (H,U)          & \longmapsto \Exp(H)U      & 
 (S,O)          & \longmapsto \Exp(S)O      \\
\end{align*}
\end{theo}

\begin{example}[Application]\

$\H_n$ est un sous espace vectoriel de $\M_n(\C)\simeq\R^{2n^2}$. De plus
en considérant les choix de coefficients sa dimension est $n^2$, on obtient
le résultat suivant (même raisonnement pour $\S_n$) :
\begin{displaymath}
 \GL_n(\C)\simeq\U(n)\times\R^{n^2} \quad \text{ et } \quad
 \GL_n(\R)\simeq\O(n)\times\R^{\frac{n(n+1)}{2}}
\end{displaymath}

En particulier comme $\O(n) = \O^+(n) \sqcup \O^-(n)$ est la décomposition en
composantes connexes, on obtient que :
$\GL_n(\R) = \GL^+_n(\R) \sqcup \GL_n^-(\R)$ est aussi la décomposition en
composantes connexes.
\end{example}
































\pagebreak
\appendix
\printindex


%% FIN DOCUMENT %%%%%%%%%%%%%%%%%%%%%%%%%%%%%%%%%%%%%%%%%%%%%%%%%%%%%%%%%%%%%%
%%%%%%%%%%%%%%%%%%%%%%%%%%%%%%%%%%%%%%%%%%%%%%%%%%%%%%%%%%%%%%%%%%%%%%%%%%%%%%%%


\end{document}

