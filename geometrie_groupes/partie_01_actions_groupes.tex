\section{Actions de groupes, groupes topologiques}
\vspace{0.5em}

\subsection{Actions de groupes}
\vspace{0.5em}


\begin{defi}[Action d'un groupe $G$ sur un ensemble $X$]
\index{action de groupe}

 On dit que $G$ agit sur $X$ s'il existe une fonction
\begin{displaymath}
 \begin{array}{rrcl}
          \cdotp :& G\times X& \longrightarrow & X \\
           & (g,x)    & \longmapsto     & g\cdotp x
          \end{array}
\end{displaymath}

telle que :
\begin{enumerate}
 \item $\forall g,h \in G,\ \forall x \in X,\ g\cdotp(h\cdotp x) = (gh)\cdotp x$
 \item $\forall x \in X,\ e\cdotp x = x$
\end{enumerate}
\end{defi}

\begin{example}[Remarque]
 Il existe toujours une action triviale telle que $\forall g\in G,\ \forall x
\in X,\ g\cdotp x = x$.
\end{example}

\begin{prop}[Autre point de vue]

$G$ agit sur $X$ si et seulement si il existe un homomorphisme :
\begin{displaymath}
 \varphi : G \longrightarrow \S_X
\end{displaymath}

Un unique action $\cdotp$ est associé à un tel homomorphisme $\varphi$ :
\begin{displaymath}
\begin{array}{rrcl}
          \cdotp :& G\times X& \longrightarrow & X \\
           & (g,x)    & \longmapsto     & g\cdotp x = \varphi(g)(x)
          \end{array}
\end{displaymath}

Et réciproquement, un unique homomorphisme $\varphi$ est associé à une action
$\cdotp$ :
\begin{displaymath}
\begin{array}{rrcl}
          \varphi :& G & \longrightarrow & \S_X \\
                   & g & \longmapsto     & f : x\longmapsto f(x) = g\cdotp x
          \end{array}
\end{displaymath}
\end{prop}

\begin{defi}[Orbite]\index{orbite}

Soit $x\in X$. On définit l'orbite de $x$ sous l'action de $G$ par :
$G.x := \{g\cdotp x\tq g\in G\}$
\end{defi}

\begin{example}[Remarque]
 L'appartenance à une même orbite définit une relation d'équivalence sur $X$.
En particulier les orbites constituent une partition de $X$.
\end{example}

\begin{defi}[Stabilisateur]\index{stabilisateur}
 
Soit $x\in X$. Le stabilisateur de $x$ dans $G$ est le sous-groupe $\Stab_G(x)
= G_x = \{g\in G \tq g\cdotp x = x\}$.

Si $Y\subset X$, $\Stab_G(Y)= \{g\in G \tq g\cdotp Y \subset Y$ et
$g^{-1}\cdotp Y \subset Y \}
= \Stab_G(Y)= \{g\in G \tq g\cdotp Y = Y\}$
\end{defi}

\begin{defi}[Action transitive]\index{action de groupe!transitive}

 On dit que $G$ agit transitivement sur $X$ s'il n'y a qu'une seule orbite :
\[\forall x,y \in X,\ \exists g\in
G,\ g\cdotp x = y\]
\end{defi}

\begin{defi}[Action fidèle]\index{action de groupe!fidèle}

 On dit que $G$ agit fidèlement sur $X$ si :
\[\forall g\in G,\ \left[ (\forall x \in X,\ g\cdotp x = x) \Longrightarrow (g =
e) \right] \]
\end{defi}


\begin{example}[Remarques] \
 \begin{itemize}
  \item Avec le second point de vue, une action est fidèle si et seulement si
$\Ker\varphi = \{e\}$, ie : $\varphi$ injective.
 \item Si $G$ n'agit pas fidèlement, on peut quotienter par $\Ker\varphi$
pour obtenir une action fidèle.
 \end{itemize}
\end{example}

\begin{example}[Exemples]\ 
\begin{enumerate}
  \item $\GL_n(\R)$ agit transitivement sur $\R^n\-\{0\}$. En
particulier, il n'y a qu'une seule orbite qui est $\R^n\-\{0\}$.
  \item On rappelle que $\O(n) := \{M\in \GL_n (\R) \tq \t MM = \Id_n\}$. Les
orbites de $\O(n)$ sont les sphères centrées en l'origine.
 \end{enumerate}
\end{example}

\begin{prop}[Stabilisateur - Orbite]

 L'application suivante est une bijection :
\[  \begin{array}{rrcl} \varphi :&
              G/(\Stab_G(x)) &\longrightarrow& \omega (x) \\
              & \overline{g}   &\longmapsto    &g\cdotp x
             \end{array} \]
\end{prop}

\begin{proof}\ 

 On commence par montrer que $\varphi$ est bien définie.
Soient $g,g' \in G$ tels que $g^{-1}g' \in \Stab_G (x)$. Alors $g\Stab_G(x) =
g'\Stab_G(x)$ donc $(g^{-1}g')\cdotp x = x $ et $ g' \cdotp x = g
\cdotp x$. Ainsi les éléments d'une même classe modulo le stabilisateur
agissent de la même manière sur $X$.

La surjectivité étant évidente, il reste à montrer l'injectivité. Soient $g,g'
\in G$ tels que $g\cdotp x = g' \cdotp x$. Alors $x = (g^{-1}g')\cdotp x$ et
$(g^{-1}g') \in \Stab_G(x)$. Ainsi si deux éléments agissent de la même manière
sur $X$, alors ils sont dans la même classe d'équivalence.
\end{proof}

\subsection{Groupes topologiques}
\vspace{0.5em}

\begin{defi}[Groupe topologique]\index{groupe topologique}

Un groupe $G$ est dit topologique, s'il est muni d'une topologie pour laquelle
la multiplication et l'inversion continues.
\end{defi}

\begin{example}[Exemple]
 $(\Z,+)$ pour la topologie discrète (ie : les fermés sont les ensembles finis
de points) est un groupe topologique.
\end{example}

\begin{prop}
\index{groupe topologique!sous-groupe}
\index{groupe topologique!produit}
 \begin{enumerate}
  \item Un sous-groupe d'un groupe topologique est un groupe topologique pour
la topologie induite.
  \item Un produit de groupes topologiques est un groupe topologique pour la
topologie produit.
 \end{enumerate}
\end{prop}

\begin{proof}\ 
 
\begin{enumerate}
 \item Soit $H < G$. Un ouvert $\Omega$ de $H$ est de la forme $\Omega'\cap H$
avec $\Omega'$ ouvert de $G$. Ainsi :
\[\mu_H^{-1}(\Omega) = \mu_H^{-1}(\Omega'\cap H) = \mu_H^{-1}(\Omega')\cap
\mu_H^{-1}(H) = \underbrace{\underbrace{\mu^{-1}(\Omega')}_{\text{ouvert de
}G \times G}\cap(H\times H)}_{\text{ est un ouvert de } H\times H}\]
Et de même :
\[\iota_H^{-1}(\Omega) = \iota_H^{-1}(\Omega'\cap H) =
\iota_H^{-1}(\Omega')\cap \iota_H^{-1}(H) = \underbrace{
\underbrace{\iota^{-1}(\Omega')}_{\text{ouvert de } G}\cap H}_{\text{ouvert de
}H}\]
\item Par définition de la topologie produit.
\end{enumerate}
\end{proof}

\begin{defi}[morphisme de groupes topologiques]
\index{morphisme!de groupe topologique}
\index{morphisme!de groupe topologique!isomorphisme}
\index{morphisme!de groupe topologique!automorphisme}

Un morphisme de groupes topologiques est un morphisme de groupes qui est
continu.

On définit de même un isomorphisme de groupes topologiques s'il est
bicontinu, et un automorphisme de groupe topologique si c'est un isomorphisme
bicontinu d'un groupe topologique dans lui même.
\end{defi}

\begin{prop}

Soit $G$ un groupe topologique.
\begin{enumerate}
 \item Pour $x\in G$ la multiplication à gauche (resp. à droite) par $x$ notée
$l_x : g\mapsto xg$ (resp. $r_x : g\mapsto gx$) est un homéomorphisme mais pas
un morphisme de groupe.
 \item De même l'inversion $\iota : g\mapsto g^{-1}$ est un homéomorphisme mais
pas un morphisme de groupe.
 \item Pour $x\in X$ la conjugaison par $x$ noté $\gamma_x : g\mapsto xgx^{-1}$
est un isomorphisme de groupe topologique.

Pour avoir une action à gauche il
faut $xgx^{-1}$ et pour avoir une action à droite il faut $x^{-1}gx$.
\end{enumerate}
\end{prop}

\begin{defi}[Composante du neutre]
\index{composante connexe!du neutre}
\index{composante connexe}

Soit $G$ un groue topologique. On définit la composante du neutre $G_e$ comme
la composante connexe de $G$ contenant $e$.
\end{defi}

\begin{prop}
 \begin{enumerate}
  \item $G_e$ est un sous-groupe fermé et distingué de $G$,
  \item $\forall x \in G$ la composante connexe de $G$ contenant $x$ est $G_ex =
xG_e$,
  \item Tout sous-groupe $H < G$ qui est ouvert, est également fermé. En
particulier, $H$ est la réunion de composantes connexes de $G$ et $H \supset
G_e$.
 \end{enumerate}
\end{prop}

\begin{proof} \
 \begin{enumerate}
  \item \begin{itemize}
    \item [(fermé)] Toute composante connexe est fermée et ouverte.
    \item [(sous-groupe)] La continuité de la multiplication nous assure
 que $\mu(G_e\times G_e)$ est connexe, de plus
$\mu(G_e\times G_e) \owns e = \mu(e,e)$ et $G_e = \mu(\{e\}\times G_e)\subset
\mu(G_e\times G_e)$ donc $\mu(G_e\times
G_e) = G_e$.\\
La continuité de l'inversion assure de même que
$\iota(G_e) = G_e$.
    \item [(distingué)] Soit $x\in G$ alors $xG_ex^{-1} =
\gamma_x(G_e)\owns 1$. 
\end{itemize}
\item $G_e = l_{x^{-1}}(xG_e) \subset l_{x^{-1}}(\Cen_x) \subset G_e$.
\item TODO
 \end{enumerate}
\end{proof}

\begin{lemm}
 
Si $Y\subset X$ est fermé et ouvert, alors c'est une union de composantes
connexes de $X$.
\end{lemm}

\begin{prop}
 
Soit $G$ un groupe topologique connexe et $V$ un voisinage ouvert de $\{e\}$
dans $G$. Alors $V$ engendre $G$ :
\begin{displaymath}
 G = \bigcup_n \left(V\cup \iota(V)\right)^n
\end{displaymath}
\end{prop}

\begin{theo}[Connexité de $\GL_n$]
\index{groupe classique!$\GL_n(\C)$}
 
$\GL_n(\C)$ est un groupe topologique connexe.
\end{theo}

\begin{proof}\
\index{norme de Frobénius}
 
$\M_n(\C)$ est muni de la topologie issue de la norme de Frobénius :
\begin{displaymath}
 \|A\f = \left(\sum_{j,j} |a_{i,j}|^2 \right)^{\frac{1}{2}} =
\left(\Tr\left(AA\*\right)\right)^{\frac{1}{2}}
\end{displaymath}

On montre ensuite la connexité par arcs en considérant la connexité par arc du
complémentaire des zéros du déterminant de $zA + (1-z)\Id$.
\end{proof}

\begin{prop}
\index{norme de Frobénius}

La norme de Frobénius est une norme d'algèbre : $\|AB\f \leq \|A\f \|B\f$.
\end{prop}

\begin{defi}[Produit semi-direct]
\index{groupe topologique!produit!semi-direct}
\index{groupe topologique!produit!direct}
 
Soient $N,H$ des groupes et $\theta : H \rightarrow \Aut N$ un morphisme de
groupe. On définit :

\begin{displaymath}
 N\rtimes_\theta H \text{ avec la loi : }
(n,h)(n',h') \mapsto (n\theta(h)(n'), hh')
\end{displaymath}

Si $\theta = \Id$ on dit que le produit est direct.
\end{defi}

\begin{prop}
\index{groupe classique!$\GL_n(\R)$}

$\GL_n(\R)$ n'est pas connexe.
\end{prop}

\begin{proof}
 
En effet $\R^*$ est l'image non connexe de $\det : \GL_n(\R) \rightarrow \R$ qui
est continu.
\end{proof}

\begin{example}[Remarque]
On verra plus loin que $\GL_n(\R)$ possède exactement deux composantes connexes
: $\GL_n^+(\R)$ qui est aussi un sous-groupe distingué et $\GL^-_n(\R)$.
\end{example}

\subsection{Action de groupes topologiques}
\vspace{0.5em}

\begin{defi}[Action de groupes topologiques]\index{action de groupe topologique}

Soient $G$ un groupe topologique et $X$ un espace topologique.

On dit que $G$ agit topologiquement sur $X$ si l'action : $G\times X\rightarrow
X$ est continue.
\end{defi}

\begin{prop}
\index{orbite!connexe}
\index{orbite!compacte}
 
Si $G$ agit topologiquement sur $X$ alors :
\begin{enumerate}
 \item Si $G$ est connexe alors les orbites sont connexes.
 \item Si $G$ est compact alors les orbites sont compactes.
\end{enumerate}
\end{prop}

\begin{proof}
Une orbite est une image par une application continue.
\end{proof}

\begin{prop}
 
Si $G$ agit topologiquement sur $X$ et $x \in X$ alors $\adh (G\cdotp x)$ est
une union d'orbites (ie : $\adh (G\cdotp x)$ est stable sous l'action de $G$).
\end{prop}

\begin{proof}
 On prend une suite $(x_n)_n$ de $G\cdotp x$ qui converge vers $y\in
\adh(G\cdotp x)$. Alors par continuité de l'action, la suite des images est
aussi une suite de $G\cdotp x$ qui converge vers $g\cdotp y \in \adh (G\cdotp
x)$.
\end{proof}

\begin{coro}
 
On peut construire une relation d'ordre partiel $\preccurlyeq$ sur les orbites
de $G$ dans $X$.
\begin{displaymath}
 \Omega \preccurlyeq \Omega' \Longleftrightarrow \Omega \subset \adh(\Omega')
\end{displaymath}
\end{coro}

\begin{defi}\index{orbite}
 
En considérant la relation d'équivalence $\sim$ d'appartenance à la même
orbite, on définit l'espace quotient $X/G$ des orbites.
\end{defi}

\begin{prop}
 
L'ensemble $X/G$ peut être muni d'une topologie naturelle rendant l'application
$\pi$ continue et ouverte.
\begin{displaymath} \begin{array}{rrcl}
 \pi :&X &\longrightarrow& X/G \\
      &x &\longmapsto    & G\cdotp x
\end{array} \end{displaymath}
\end{prop}

\begin{proof}
 On construit la topologie sur $X/G$ en posant que $U\subset X/G$ est ouvert si
$\pi^{-1}(U)$ est ouvert pour la topologie sur $X$. On vérifie ensuite qu'on a
bien une topologie rendant $\pi$kjii ouverte (elle est continue par
construction).
\end{proof}

\begin{prop}
 $f : X/G \rightarrow Y$ est continue si et suelement si $f\circ
\pi:X\rightarrow Y$ est continue.
\end{prop}

\begin{proof}
 C'est une conséquence de la définition de la topologie définie sur $X/G$.
\end{proof}

\begin{defi}[Espace homogène]\index{espace homogène}
 
Un espace homogène est un espace topologique sur lequel un groupe topologique
agit topologiquement et transitivement.

En particulier il n'y a qu'une seule orbite et les stabilisateurs sont tous
conjugués.
\end{defi}

\begin{theo}
 Si $X$ est un espace topologique localement compact à base dénombrable, alors
pour tout $x \in X$ :
\begin{displaymath}
 X \text{ est isomorphe à } G/\Stab_G (x)
\end{displaymath}

\end{theo}
























