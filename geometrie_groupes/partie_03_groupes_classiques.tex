\section{Etude des sous-groupes classiques}
\vspace{0.5em}

On applle groupe classique tout sous-groupe de $\GL_n$ fermé et défini en
famille.

\subsection[$\mathbf{\SL_n(\K)}$]{$\mathbf{\SL_n(\K)}$ avec $\mathbf{\K = \R
\text{ ou }\C}$}
\vspace{0.5em}

\begin{defi}[$\SL_n(\K)$]
 \begin{displaymath} \SL_n(\K) := \{M\in \GL_n(\K) \tq \det M = 1\}
\end{displaymath}

En particulier on a $\SL_n(\R) \subset \GL_n^+(\R)$.
\end{defi}

On définit l'homéomorphisme suivant :
\begin{displaymath}  \begin{array}{rrcl}
    \sigma :&\GL_n(\K) &\longrightarrow&\SL_n(\K) \times \K\* \\
       &     g &\longmapsto    & \left(g \begin{pmatrix}
                                 (\det g)^{-1} &    &        &  \\
                                               & 1  &        &  \\
                                              &    & \ddots &  \\
                                               &    &        & 1
                                \end{pmatrix},\ \det g\right)
   \end{array}\end{displaymath}


\begin{prop}
 \begin{enumerate}
  \item $\SL_n(\C)$ est connexe par arc.
  \item Si $\GL_n^+(\R)$ est connexe par arc, alors $\SL_n(\R)$ l'est aussi.
 \end{enumerate}
\end{prop}

\begin{proof}
On définit la projection : $p : \SL_n(\C)\times \C\* \owns (A,t) \longmapsto A
\in \SL_n(\C)$.

Ainsi on a $p\circ\sigma(\GL_n(\C)) = \SL_n(\C)$.

De même la restriction $\sigma' := \sigma\big|_{\GL_n^+(\R)}$ est aussi un
homéomorphisme et $p\circ \sigma'(\GL_n^+(\R)) = \SL_n(\R)$.
\end{proof}

\subsection{Groupes associés à une forme sesquilinéaire}
\vspace{0.5em}

\begin{defi}[Forme sesquilinéaire]

Soit $E$ un $\C$-espace vectoriel de dimension finie $n$.
Un application $<\cdotp,\cdotp>\ : E\times E \longrightarrow E$ est une forme
sesquilinéaire si :
\begin{enumerate}
 \item $<x,\cdotp>\ : E \longrightarrow E$ est linéaire,
  \item $\forall x,y,z \in E,\ <x+y, z>\ =\ <x,z>\ +\ <y,z>$,
   \item $\forall \lambda \in \C,\ \forall x,y\in E, \ <\lambda x,y>\ =
\overline{\lambda} \ <x,y>$
\end{enumerate}
\end{defi}

\begin{defiprop}[Groupe des invariants $\mathcal{U}(E)$]
 
\begin{displaymath} \mathcal{U}(E) := \{f : E \longrightarrow E\text{ linéaire
bijective } \tq
<f(x),f(y)> \ = \ <x,y>\ \} \end{displaymath}

$\mathcal{U}(E)$ est le sous-groupe des invariants de $<\cdotp,\cdotp>$ dans
$\GL(E)$. C'est un sous-groupe fermé de $\GL(E)$ isomorphe à un sous-groupe de
$\GL_n(\C)$.

\end{defiprop}

\begin{proof}
Si $f,g \in \mathcal{U}(E)$ on vérifie facilement que $f\circ g$, $f^{-1}$, et
$\Id_E$ sont également dans $\mathcal{U}(E)$.

En fixant une base $\mathcal{B} = (e_1, \cdots, e_n)$ on peut identifier
$\GL(E)$ à $\GL_n(\C)$ par $\Mat_{\mathcal{B}} : \GL(E)\longrightarrow
\GL_n(\C)$. Soit alors $\Omega = (<e_i,e_j>) \in \M_n(\C)$, on a $<x,y>\ =
X\*\Omega Y$. Si $M = \Mat_\mathcal{B}(f)$ alors :
\begin{displaymath}\begin{array}{rl}
                    & f\in \mathcal{U}(E) \\
\Longleftrightarrow & X\*\Omega Y = (MX)\*\Omega(MY) \\
\Longleftrightarrow & X\*\Omega Y = X\*M\* \Omega MY \\
\Longleftrightarrow & \Omega = M\*\Omega M \\
  \end{array}
\end{displaymath}

La dernière implication est bien vérifiée car si $A=(a_{kl}), X=e_i, Y=e_j$
alors $X\*AY = a_{ij}$. Ainsi :
\begin{displaymath}\mathcal{U}(E) \simeq \{M\in \GL_n(\C) \tq M\*\Omega M -
\Omega = 0\} \end{displaymath}

C'est en particulier l'image réciproque du fermé $\{0\}$ par un application
continue, c'est donc un sous-groupe fermé.
\end{proof}

\begin{defiprop}[Groupe unitaire $\U(n)$]

Si la forme $<\cdotp,\cdotp>$ est hermitienne (ie : $\forall x,y \in E,\
<y,x>\ =\ \overline{<x,y>}$), on note alors $\U(n,\C)$ ou $\U(n)$.
\begin{displaymath} \U(n) = \{ M \in \GL_n(\C) \tq M\*M = \Id\}
\end{displaymath}
\end{defiprop}

\begin{prop}
 
$\U(n)$ est compact et connexe par arcs.
\end{prop}

\begin{proof}\ 

On a vu que $\U(n)$ est fermé. De plus $\forall M\in\U(n), \|M\f^2 = \Tr
(M\*M) = n$. Ainsi $\U(n)$ est fermé borné, donc compact.

On montre la connexité par arcs en utilisant la théorie de la réduction des
endomorphismes. Pour $M\in\U(n)$, il existe une base orthonormée telle que $M$
est diagonale. On peut alors construire le chemin continu.
\begin{displaymath}\exists P\in\U(n), P\*MP =
\begin{pmatrix}         \lambda_1 & &0\\
                          & \ddots & \\
                      0 & &\lambda_n \end{pmatrix}\end{displaymath}
\end{proof}

\begin{defi}[Groupe spécial unitaire]
 
\begin{displaymath} \SU(n) := \U(n) \cap \SL_n(\C) \end{displaymath} 
\end{defi}


\begin{example}[Remarque]
 En remarquant que $(M\*M=\Id) \Longrightarrow (|\det M| = 1)$, on peut définir
l'homéomorphisme :
\begin{displaymath} \sigma\big|_{\U(n)} : \U(n) \longrightarrow \SU(n)\times
\S^1 \end{displaymath}

Ainsi $\SU(n)$ est également compact et connexe par arcs.
\end{example}


\vspace{0.5em}
\subsection{Groupes associés à une forme bilinéaire symétrique}
\vspace{0.5em}

\begin{defi}[Forme bilinéaire symétrique - antisymétrique]
 
Si $E$ est un $\K$-espace vectoriel de dimension fini, et $<\cdotp,\cdotp>$ un
forme bilinéaire, elle est dite :
\begin{itemize}
 \item Symétrique si : $\forall x,y \in E,\ <y,x>\ =\ <x,y>$
 \item Antisymétrique si : $\forall x,y \in E,\ <y,x>\ = -<x,y>$
\end{itemize}
\end{defi}

Comme dans le paragraphe précédant, en remplaçant $M\*$ par $^tM$ on montre
que $\O(E)$, le sous-groupe de $\GL(E)$ qui laisse une forme bilinéaire
invariante, est isomorphe à un sous-groupe fermé de $\GL_n(\K)$.

On pose $\Omega = (<e_i,e_j>) \in \M_n(\K)$. En particulier si la forme est
symétrique, $\Omega$ est symétrique.

\begin{prop}[$\O(n)$]
\begin{displaymath}\O(n) \simeq \{ M \in \GL_n(\K) \tq \ ^tM\Omega M - \Omega =
0 \} \end{displaymath}
\end{prop}

\begin{defi}[Groupe orthogonal et groupe spécial orthogonal complexe]
 
Si $\K = \C$ et que la forme n'est pas dégénérée, on peut supposer $\Omega =
\Id$. On définit le groupe orthogonal complexe :
\begin{displaymath}\O(n,\C) = \{ M \in \M_n(\C) \tq \ ^tM M = \Id
\}\end{displaymath}

En remarquant que $(^tMM=\Id) \Longrightarrow (|\det M| = 1)$, on définit
également le groupe spécial orthogonal complexe :

\begin{displaymath}\SO(n,\C) = \O(n,\C) \cap \SL(n,\C) \end{displaymath}
\end{defi}

\begin{defi}[Groupe orthogonal et groupe spécial orthogonal réel]
 
Si $\K = \R$ on ne peut plus supposer que $\Omega = \Id$. On définit le groupe
orthogonal réel :
\begin{displaymath}\O(n) := \O(n,\R) = \{ M \in \M_n(\R) \tq \ ^tM M = \Id
\}\end{displaymath}

On définit de même le groupe spécial orthogonal réel :

\begin{displaymath}\SO(n) := \SO(n,\R) = \O^+(\R) = \O(n) \cap \SL(n,\R)
\end{displaymath}
\end{defi}

\begin{example}[Remarque]
 $\O(n) = \O^+(\R) \sqcup \O^-(\R)$ (union disjointe de fermés de $\O(n)$).
\end{example}

\begin{prop}
 \begin{enumerate}
  \item $\O(n)$ est compact,
  \item $\SO(n)$ est connexe par arcs,
  \item $\O^+(\R) \sqcup \O^-(\R)$ est la décomposition en composantes connexes.
 \end{enumerate}
\end{prop}

\begin{proof}\

\begin{enumerate}
 \item $\O(n) = \U(n) \cap \M_n(\R)$, en particulier, c'est un sous-groupe
fermé du compact $\U(n)$, c'est donc un compact.
 \item \begin{displaymath} \ ^tPMP = \begin{pmatrix}
                     I_p &      &             &        & \\
                         & -I_q &             &     0   & \\
                         &      & R_{\theta_1}&        & \\
                         &    0 &             & \ddots & \\
                         &      &             &        & R_{\theta_i} \\
                    \end{pmatrix} = \begin{pmatrix}
                     I_p &             &        & \\
                         &             &     0   & \\
                         & R_{\theta_1}&        & \\
                         &         0    & \ddots & \\
                         &             &        & R_{\theta_r} \\
                    \end{pmatrix} \end{displaymath}
 $M \in \SO(n)$ donc $M$ est diagonalisable par blocs de taille au plus $2$.
De plus $(\det M = 1) \Rightarrow (q\text{ est pair})$, et on peut remplacer
$I_2$ par $R_{\theta_\pi}$.

On peut construire un chemin continu $\tilde{\gamma} : [0,1] \owns t
\longmapsto R_{t\theta}$ tel que $\tilde{\gamma}(0) = \Id$ et
$\tilde{\gamma}(1) = R_\theta$. On en déduit un chemin continu $\gamma$ de $\Id$
à $M$.

\end{enumerate}
\end{proof}


\vspace{0.5em}
\subsection{Groupes associés à une forme bilinéaire antisymétrique}
\vspace{0.5em}

On suppose que la forme est non-dégénérée, ce qui implique que $\dim E = 2n$.
Alors dans une certaine base on a :
\begin{displaymath}\Omega = \begin{pmatrix}
            0 & \Id_n \\ -\Id_n & 0
           \end{pmatrix}
 \quad \text{ ou bien } \quad \begin{pmatrix}
                               &     &  &  &     & -1 \\
                               &    0 &  &  & \cdotp   &    \\
                               &     &  &-1&     &     \\
                               &     &1  &  &     &  \\
                               &\cdotp&  &  &  0  &    \\
                              1&     &  &&     &     \\
                              \end{pmatrix} \end{displaymath}


\begin{defi}[Groupe symplectique]
 
On appelle groupe symplectique :
\begin{displaymath} \Spl_{2n} := \{ M\in \M_n(\K) \tq ^tM\Omega M - \Omega = 0\}
\end{displaymath}
On a toujours $(\M \in \Spl_{2n}) \Rightarrow (|\det M| = 1)$, on définit ainsi
le
groupe spécial symplectique :
\begin{displaymath} \USp_{2n} := \Spl_{2n} \cap
\mathcal{U}(2n)\end{displaymath} 
\end{defi}

\begin{prop}
\begin{enumerate}
 \item $\USp_{2n}$ est compact,
 \item $\USp_{2n} = \left\lbrace z \in \mathcal{U}(2n)\ \Big|\ \exists
A,B\in\M_n{\C}, z = \begin{pmatrix}A&-B\\B&A\end{pmatrix}\right\rbrace.$

\end{enumerate}
\end{prop}

\begin{proof}\ 

\begin{enumerate}
 \item C'est un fermé de $\U(n)$, donc un compact.
 \item
\begin{displaymath}\begin{array}{ccccc} \begin{array}{rl}
                    & z\in \mathcal{U}(2n) \\
\Leftrightarrow & z\*z = \Id \\
\Leftrightarrow & z\* = z^{-1} \\
\Leftrightarrow & ^tz = \overline{z^{-1}}
   \end{array} &  \text{ et }  & 
\begin{array}{rl}
                    & z \in \Spl(2n) \\
\Leftrightarrow & ^tz\Omega z = \Omega \\
                    & z \in \USp(2n) \\
\Leftrightarrow & \overline{z^{-1}}\Omega z = \Omega \\
\Leftrightarrow & z^{-1}\Omega \overline{z} = \Omega \\
\Leftrightarrow & \Omega \overline{z} = z \Omega \\
\end{array}&  \text{ donc }  & 
\end{array} \end{displaymath}
\begin{displaymath}
\begin{array}{rl}
                    & z=\begin{pmatrix}   A&B\\
C&D\end{pmatrix}\in \USp(2n) \\
\Leftrightarrow & \begin{pmatrix}0&I_n\\-I_n&0 \end{pmatrix}
\begin{pmatrix}\overline{A}&\overline{B}\\ \overline{C}&\overline{D}
\end{pmatrix} = \begin{pmatrix}A&B\\C&D \end{pmatrix}
\begin{pmatrix}0&I_n\\-I_n&0 \end{pmatrix}  \\
\Leftrightarrow & \begin{pmatrix}\overline{C}&\overline{D}\\
-\overline{A}&-\overline{B}
\end{pmatrix} = \begin{pmatrix}A&B\\C&D \end{pmatrix} \\
\Leftrightarrow& C = -\overline{B} \text{ et } D = \overline{A} \\
\end{array} \end{displaymath}
\end{enumerate}

\end{proof}


\vspace{0.5em}
\subsection{Exemples en petites dimensions}
\vspace{0.5em}

\begin{itemize}
 \item $\O(1) = \{\pm 1\}$ et $\O^+(1) = \{1\}$,
 \item $\U(1) = \S^1$ et $\S\U(1) = \{1\}$,
 \item $\SO(2) \simeq \U(1)$
\end{itemize}







