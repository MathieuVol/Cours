
\section{Groupes}
\vspace{0.5em}

\subsection{Groupes}
\vspace{0.5em}

\begin{defi}[Groupe]\index{groupe}\index{groupe!abélien}

Un groupe est un ensemble $G$ muni d'une opération interne $\cdotp$ vérifiant :
\begin{description}
 \item [G1] $\cdotp$ est associative,
 \item [G2] Il existe un élément neutre (notation : $1$ ou $e$),
 \item [G3] Tout élément est inversible (notation : $x^{-1}$).
\end{description}

Si de plus la loi est commutative, on dit que le groupe est abélien, et on note
la loi $+$, le neutre $0$, et l'inverse $-x$.
\end{defi}

\begin{defi}[Produit direct]\index{groupe!produit direct}

Si $(G,\cdotp)$ et $(G',*)$ sont des groupes, on définit le groupe produit
direct $(G\times G', \square)$ par : 
\begin{displaymath}\begin{array}{rrcl}
 \square : & (G\times G')\times(G\times G') & \longrightarrow & G\times G' \\
& ((g,g') , (\gamma,\gamma')) &\longmapsto & (g\cdotp \gamma, g'*\gamma')
 \end{array}\end{displaymath}

\end{defi}

\begin{prop}

Soit $G$ un groupe, et $g\in G$.
 \begin{itemize}
  \item L'élément neutre est unique,
  \item L'inverse de $g$ est unique.
 \end{itemize}
\end{prop}

\subsection{Morphisme de groupes}
\vspace{0.5em}

\begin{defi}[Morphisme de groupe]\index{morphisme!de groupe}

 Soient $(G,\cdotp)$ et $(G',*)$ deux groupes, et $\varphi : G \longrightarrow
G'$. On dit que $\varphi$ est un morphisme de groupe, ou un homomorphisme si :
\begin{displaymath} \forall g, \gamma \in G,\ \varphi(g\cdotp \gamma) =
\varphi(g) *
\varphi(\gamma)\end{displaymath}
\end{defi}

\begin{prop}
 
L'image du neutre de $G$ est le neutre de $G'$.

L'image de l'inverse d'un élément $g \in G$ est l'inverse de l'image de $g$.
\end{prop}

\begin{defi}
\index{morphisme!de groupe!isomorphisme}
\index{morphisme!de groupe!endomorphisme}
\index{morphisme!de groupe!automorphisme}
\begin{itemize}
 \item Un isomorphisme est un morphisme bijectif.
 \item Un endomorphisme est un morphisme d'un groupe dans lui-même.
 \item Un automorphisme est un endomorphisme bijectif.
\end{itemize}
\end{defi}

\begin{defi}[Ordre d'un groupe fini]\index{ordre!groupe fini}
 
Si $G$ est un groupe fini, on appelle ordre du groupe et on note $\#G = |G| =
\textnormal{o}(G)$ le nombre de ses éléments.
\end{defi}

\pagebreak

\begin{defi}[Puissances]\index{puissance!élément d'un groupe}

 Si $G$ est un groupe, et $x\in G$, on définit les puissances de $x$ pour $n
\in \mathbb{N}^*$.
\begin{align*}
x^n &:=  \underbrace{x \cdots x}_{n\text{ fois}} &
x^0 &:=  e_G &
x^{-n} &:=  \underbrace{x^{-1} \cdots x^{-1}}_{n\text{ fois}}
\end{align*}
\end{defi}

\begin{defi}[Ordre d'un élément]\index{ordre!élément}

 Soit $x$ un élément d'un groupe $G$. S'il existe $k \in \mathbb{N}^*$ tel que
$x^k = e$, on appelle ordre de $x$ et on note $|x| = \textnormal{o}(x)$, le plus
petit des tels $k$.
\end{defi}


\subsection{Sous-groupes}
\vspace{0.5em}

\begin{defiprop}[Sous-groupe]\index{groupe!sous-groupe}

 Soient $(G,\cdotp)$ un groupe, et $H$ un sous-ensemble non vide $G$.

 Si $\forall x,y \in H,\ x^{-1}y\in H$, alors on dit que $H$ est un sous-groupe
de $G$. On montre facilement que dans ce cas c'est un groupe pour la loi induite
par celle de $G$. On note alors $H < G$.
\end{defiprop}

\begin{example}[Remarque]
 On peut remplacer les conditions précédantes sur $H$ par :
\begin{enumerate}
 \item $e\in H$,
 \item $\forall x,y\in H,\ xy \in H$,
 \item $\forall x \in H,\ x^{-1} \in H$.
\end{enumerate}
\end{example}

\begin{prop}[Intersection de sous-groupes]
 
Un intersection quelconque de sous-groupes est encore un sous-groupe.
\end{prop}

\begin{defitheo}[Sous-groupe engendré]\index{groupe!sous-groupe!engendré}

 Si $X$ est un sous-ensemble d'un groupe $G$, alors il existe un plus petit
sous-groupe de $G$ qui contient $X$. On l'appelle sous-groupe engendré par $X$
et on le note $<X>$.

Si $X = \{g_1,\cdots,g_n\}$ est fini, on note $<X>\ =\ <g_1,\cdots,g_n>$.
\end{defitheo}

\begin{defi}[Mot sur $X$]\index{mot sur un sous-ensemble}
 
Si $X$ est un sous-ensemble d'un groupe $G$, on définit l'ensemble des mots sur
$X$ :
\begin{displaymath} \{e\} \cup \Big\lbrace\omega = x_1^{e_1}\cdots x_n^{e_n} \in
G \tq x_i \in
X, e_i \in \lbrace-1,1\rbrace\Big\rbrace  \end{displaymath}
\end{defi}

\begin{theo}
 
Si $X$ est un sous-ensemble d'un groupe $G$, alors :
\begin{itemize}
 \item Si $X = \emptyset$, alors $<X>\ =\ \{e\}$,
 \item Sinon $<X>$ est l'ensemble des mots sur $X$.
\end{itemize}
\end{theo}

\begin{example}[Cas particulier]\index{groupe!cylcique}
 Si $X = \{a\} \subset G$ avec $a\in G$, alors $<X>\ =\ <a>$ est l'ensemble des
puissances
de $a$. On l'appelle le sous-groupe cyclique engendré par $a$.
\end{example}

\begin{defi}[Classes d'un sous-groupe]\index{groupe!sous-groupe!classe}

 Soient $H < G$. Pour $g \in G$ on définit :
\begin{itemize}
 \item $gH = \{g h \tq h \in H\}$ la classe à gauche de $g$ modulo $H$,
 \item $Hg = \{h g \tq h \in H\}$ la classe à droite de $g$ modulo $H$.
\end{itemize}
\end{defi}

\begin{example}[Remarque]
 En général un classe ne contient pas le neutre et n'est donc pas un
sous-groupe.
\end{example}

\begin{lemm}

 Soient $H < G$ et $a,b \in G$.
\begin{itemize}
 \item $aH = bH \Leftrightarrow a^{-1}b\in H$,
 \item $Ha = Hb \Leftrightarrow ab^{-1}\in H$,
 \item $aH = bH$ ou bien $aH \cap bH = \emptyset$.
\end{itemize}
\end{lemm}

% BUG ?
%\pagebreak

\begin{theo}[Lagrange]\index{Lagrange, théorème de}

 Si $H < G$ avec $G$ fini, alors l'ordre de $H$ divise celui de $G$ :
 $ |H| \ \Big|\ |G| $
\end{theo}


\begin{coro}

 Le nombre de classes à gauche est égal au nombre de classes à droite.
\end{coro}

\begin{defi}[Indice d'un sous-groupe]\index{groupe!sous-groupe!indice}

 L'indice $[G:H]$ d'un sous-groupe $H$ est le nombre de classes modulo $H$ :
\begin{displaymath} |G| = [G:H]\,\cdotp |H| \end{displaymath}
et si $G$ est fini :
\begin{displaymath} [G:H] = \dfrac{|G|}{|H|} \end{displaymath}
\end{defi}

\begin{coro}
\begin{itemize}
 \item Si $G$ est un groupe fini d'ordre premier, ses seuls sous-groupes sont
le groupe trivial et lui même.
 \item Si $G$ est un groupe fini, l'ordre de tout élément divise l'ordre du
groupe.
\end{itemize}
\end{coro}

\begin{example}[Remarque]
Tout groupe fini d'ordre premier est cyclique et
abélien (car engendré par chacun des éléments différents du neutre).
\end{example}

\begin{prop}[Sous-groupes noyau et image]
\index{morphisme!de groupe!noyau et image}
 
Si $\varphi$ est un homomorphisme entre les groupes $G$ et $G'$, alors :
\begin{itemize}
 \item $\Ker \varphi$ est un sous-groupe de $G$,
 \item $\Ima \varphi$ est un sous-groupe de $G'$.
\end{itemize}
\end{prop}


\begin{prop}[Caractérisation de l'injectivité d'un morphisme]
 
Si $\varphi$ est un homomorphisme entre les groupes $G$ et $G'$, alors
$\varphi$ est injective si et seulement si $\Ker \varphi = \{e_G\}$.
\end{prop}

\begin{proof}\ 

 Si $\varphi$ est injective, assez clairement $\Ker \varphi = \{e_G\}$.

 Réciproquement on suppose que $\Ker \varphi = \{e_G\}$ et on considère $x,y\in
G, \varphi(x) = \varphi(y)$. Alors $\varphi(x) \varphi(y)^{-1}= \varphi(x)
\varphi(y^{-1})= \varphi(xy^{-1}) = e_{G'}$ donc $xy^{-1} = e_{G'}$ d'où $x=y$.
\end{proof}


\subsection[Le groupe symétrique]{Le groupe symétrique $\S_X$}
\vspace{0.5em}

\begin{defi}[Groupe symétrique de $X$]
\index{groupe!groupe symétrique}
\index{permutation}
 
Soit $X$ un ensemble non vide. L'ensemble $\mathcal{S}_X$ des permutations de
$X$ (ie : des bijections de $X$ dans lui même) est appelé le groupe symétrique
de $X$.

Si $\#X = n \in \mathbb{N}^*$, on identifie $\mathcal{S}_X$ avec $\mathcal{S}_n
:= \mathcal{S}_{[1\cdots n]}$.
\end{defi}

\begin{defi}[Fixer - déplacer] 

Si $x\in X$ et $\alpha \in \mathcal{S}_X$, on dit que $\alpha$ fixe, ou
stabilise, $x$ si $x = \alpha(x)$. Sinon on dit que $\alpha$ déplace $x$.
\end{defi}

\begin{defi}[Cycle - Transposition]
\index{permutation!transposition}
\index{permutation!cycle}

On dit que $\alpha \in \mathcal{S}_X$ est un $r$-cycle si il existe
$\{i_1,\cdots,i_r\} \subset X$ tels que $\alpha$ fixe $X\-
\{i_1,\cdots,i_r\}$ et $\forall j \in [1,r-1],\ \alpha (i_j) = i_{j+1}$ et
$\alpha(i_r) = i_1$.

Un $2$-cycle est appelé une transposition.
\end{defi}

\begin{defi}[Permutations disjointes]
 
Deux permutations sont dites disjointes si leurs supports sont disjoints.
\end{defi}

\begin{lemm}
 
Deux cycles disjoints commutent.
\end{lemm}

\begin{theo}[Décomposition en produits de cycles disjoints]
\index{permutation!décomposition en cycles disjoints}
 
Toute permutation se décompose en produit de cycles disjoints. Une telle
factorisation est unique à permutation des cycles près.
\end{theo}

\begin{coro}
 
Toute permutation se décompose en produit de transpositions. Cette
décomposition n'est en général pas unique.
\end{coro}

\begin{defitheo}\index{permutation!signature}
 
Il existe une application $\varepsilon : \mathcal{S}_n \longrightarrow
\{-1,1\}$,
appelée signature telle que :
\begin{enumerate}
 \item Si $\tau$ est une transposition, $\varepsilon(\tau) = -1$,
 \item $\forall s,\ \sigma \in \mathcal{S}_n,
\varepsilon(\tau \circ \sigma) = \varepsilon(s)\varepsilon(\sigma)$.
\end{enumerate}

On dit qu'une permutation est paire si sa signature est $1$, et qu'elle est
impaire sinon.
\end{defitheo}


\subsection{Conjugaison}
\vspace{0.5em}

\begin{defi}[Conjugué de $x$]\index{conjugaison}

 Soient $x,y \in G$. On dit que $y$ est un conjugué de $x$ s'il existe $a \in
G$ tel que $y = axa^{-1}$.
\end{defi}

\begin{prop}

 La relation d'être conjugué à un élément fixé est une relation d'equivalence.
\end{prop}

\begin{example}[Remarque]
Si $x$ est seul dans sa classe de conjugaison, alors $\forall a \in G, axa^{-1}
= x$ donc $ax = xa$, ie : $x$ commute avec tous les éléments de $G$.
\end{example}

\begin{defi}[Centre d'un groupe]
\index{groupe!centre}
 
Le centre d'un groupe $G$, noté $\Zen(G)$, est l'ensemble des éléments qui
commutent avec tous les autres :
\begin{displaymath} \Zen(G) := \{g\in G \tq \forall \gamma \in G,\ g\gamma =
\gamma g\} \end{displaymath}
\end{defi}

\begin{defi}[Centralisateur d'un élément]
\index{groupe!centralisateur}
 
Le centralisateur de $x \in G$, noté $\Cen_G(x)$, est l'ensemble des éléments
qui commutent avec $x$.
\begin{displaymath} \Cen_G(x) := \{\gamma \in G \tq \gamma x = x \gamma \}
\end{displaymath}
\end{defi}

\begin{prop}
 
$\Cen_G(x)$ est un sous-groupe de $G$.
\end{prop}

\begin{theo}

 Le nombre des éléments conjugués à $x\in G$ est $[G:\Cen_G(x)]$.

En particulier si $G$ est fini, ce nombre divise $|G|$.
\end{theo}

\begin{proof}
Soit $x$ fixé. Soient $a,b \in G$, alors :
\begin{displaymath}
 (axa^{-1} = bxb^{-1})
\Leftrightarrow (b^{-1}axa^{-1}b = x) \Leftrightarrow (b^{-1}a \in
\Cen_G(x)) \Leftrightarrow (a\Cen_G(x) = b\Cen_G(x))
\end{displaymath}
Il y a donc une
bijection entre les classes à gauche modulo $\Cen_G(x)$ et les conjugués de $x$.
\end{proof}

\begin{defi}[Equation aux classes]\index{equation aux classe@équation aux
classe}
 
Si $G$ est un groupe fini, son équation aux classes est :
\begin{displaymath} |G| = |\Zen(G)| + \sum_{x_i} [G : \Cen_G(x_i)]
\end{displaymath}
où l'on choisit un représentant $x_i$ par classe d'équivalence.
\end{defi}

\begin{defi}[Sous-groupe distingué]\index{groupe!sous-groupe!distingué}
 
On dit qu'un sous-groupe $H$ de $G$ est distingué (ou normal) dans $G$, et on
note $H \< G$, s'il est stable par conjugaison :
\begin{displaymath} \forall g\in G,\ gHg^{-1} \subset H \end{displaymath}
\end{defi}

\begin{prop}
 
Si $H$ est un sous-groupe distingué de $G$, les classes à droite coïncident
avec les classes à gauche, ie : $\forall g\in G,\ gH = Hg$. De plus on a
l'égalité $\forall g\in G,\ gHg^{-1} = H$.
\end{prop}

\begin{prop}[Intersection de sous-groupes distingués]
 
L'intersection de sous-groupes distingués de $G$ est un groupe distingué de $G$.
\end{prop}

\begin{defitheo}[Normalisateur]\index{groupe!normalisateur}
 
Si $H$ est un sous-ensemble d'un groupe $G$, on définit $\Nor_H$ le
normalisateur de $H$ :
\begin{displaymath}\Nor_H := \{x\in G \tq xHx^{-1} = H\} \end{displaymath}

Alors $\Nor_H$ est un sous-groupe de $G$. Si de plus $H$ est un groupe, alors
$\Nor_H$ est distingué.

Le normalisateur est le plus grand sous-groupe tel que $H \< \Nor_H$.
\end{defitheo}

\begin{prop}
 
Si $K < \Nor_H$, alors $H \< KH < G$.
\end{prop}


\subsection[Théorèmes d'isomorphisme]{Les trois théorèmes d'isomorphisme}
\vspace{0.5em}


\begin{defi}[Groupe quotient]
\index{groupe!quotient}
 
Si $H\<G$, on note $G/H$ l'ensemble des classes modulo $H$. Alors $G/H$ hérite
d'une structure de groupe.
\end{defi}

\begin{proof}
 On définit le produit suivant : $(xH)\cdotp (yH) := (x y)H$. On vérifie
alors facilement que les axiomes sont vérifiés.
\end{proof}

\begin{theo}[Noyau et groupe distingué]
\index{groupe!sous-groupe!distingué}
 
$H$ est un sous-groupe distingué de $G$ si et seulement si $H$ est le noyau
d'un morphisme de groupe.
\end{theo}

\begin{proof}
On a déjà vu qu'un noyau est distingué. Et réciproquement, on
construit :\\ $\varphi : G \owns x \mapsto xH \in G/H$.
\end{proof}

\begin{theo}[Premier théorème d'isomorphisme]
\index{isomorphisme, théorème!premier théorème}

 Si $\varphi : G \longrightarrow G'$ est un homomorphisme entre les groupes $G$
et $G'$, alors :
\begin{displaymath} G/\Ker\varphi \simeq \Ima \varphi \end{displaymath}
\end{theo}

\begin{proof}\ 

\setlength{\unitlength}{1em}
\begin{picture}(15,7)
\put(3.5,5.5){$G$}
\put(5,6){\vector(1,0){3}}
\put(6,6.5){$\varphi$}
\put(8.5,5.5){$G'$}

\put(0,0.5){$G/\Ker\varphi$}
\put(5,0.5){\vector(1,0){3}}
\put(6,1){$\tilde{\varphi}$}
\put(8.5,0.5){$\Ima\varphi$}

\put(4,5){\vector(0,-1){3}}
\put(3,3){$\pi$}

\put(9,2){\vector(0,1){3}}
\put(9.5,3){$i$}

\put(12,5){$\pi$ est la projection canonique}
\put(12,3.5){$i$ est l'injection canonique}
\put(12,2){$\tilde{\varphi} : G/\Ker\varphi \owns x\Ker\varphi
\longmapsto \varphi(x) \in \Ima\varphi$}
\end{picture}

On montre que $\tilde{\varphi}$ est un isomorphisme :
\begin{itemize}
 \item $\tilde{\varphi}((x\Ker\varphi)(y\Ker\varphi)) =
\tilde{\varphi}((xy)\Ker\varphi) = \varphi(x)\varphi(y) =
\tilde{\varphi}(x\Ker\varphi)\tilde{\varphi}(y\Ker\varphi)$
\item $\Ker\tilde{\varphi} = \Ker\varphi = e_{G/\Ker\varphi}$
\end{itemize}
\end{proof}

\begin{theo}[Troisième théorème d'isomorphisme]
\index{isomorphisme, théorème!troisième théorème}

 Soient $G$ un groupe et $K\<H\<G$. Alors :
\begin{equation*}
 H/K \< G/K \quad \text{ et } \quad 
(G/K) \big/ (H/K) \simeq G/H
\end{equation*}
\end{theo}

\begin{theo}[Deuxième théorème d'isomorphisme]
\index{isomorphisme, théorème!deuxième théorème}

Soient $G$ un groupe, $H<G$, et $K < G$. On suppose $H\subset \Nor_K$ (ce qui
est en particulier le cas si $H\<G$). Alors :
\begin{enumerate}
 \item \begin{displaymath}H\cap K \< H\quad \text{ et } \quad HK = KH <
G\end{displaymath}
 \item \begin{displaymath}H\big/(H\cap K) \simeq (KH)\big/K\end{displaymath}
\end{enumerate}
\end{theo}

\begin{defi}[Suite exacte]\index{groupe!sous-groupe!suite exacte}
 
Soient $(G_1,\cdots,G_n)$ des groupes et $(f_1,\cdots,f_{n-1})$ des morphismes
tels que $f_i : G_i \longrightarrow G_{i+1}$. On dit que cette suite est exacte
si pour chaque $i$ : 
\begin{displaymath} \Ima f_i = \Ker f_{i+1} \end{displaymath}
\end{defi}


\begin{defi}[Groupe résoluble]
\index{groupe!résoluble}

Un groupe $G$ est dit résoluble s'il existe une suite finie :
\begin{displaymath}G = G_0 \> G_1 \> \cdots \> G_n = \{e\}\end{displaymath}
telle que pour chaque $i$, $G_{i-1}/G_i$ soit abélien.
Dans ce cas une telle suite est appelée suite de résolubilité. 
\end{defi}


\begin{theo}[Théorème de correspondance]
\index{correspondance, théorème de}
 
Soient $K \< G$ et $\pi : G \longrightarrow G/K$. Alors $\pi$ définit une
correspondance entre les sous-groupes de $G$ qui contiennent $K$ et les
sous-groupes de $G/K$. En notant $S^*$ le sous-groupe de $G/K$ correspondant au
sous-groupe $S$ de $G$, on a :
\begin{enumerate}
 \item $S^* \simeq S/K \simeq \pi(S)$,
 \item $T \subset S \subset K \Leftrightarrow T^* \subset S^*$,
 \item $T \< S \Leftrightarrow T^* \< S^*$, et $S/T \simeq S^*/T^*$.
\end{enumerate}

\end{theo}
\begin{center}
\setlength{\unitlength}{1em}
\begin{picture}(10,6)
\put(0.5,0.5){$K$}
\put(2,2.5){$S$}
\put(3.5,4.5){$G$}

\put(7.5,0.5){$\{e\}$}
\put(9,2.5){$S^*\simeq S/K$}
\put(10.5,4.5){$G/K$}

\put(2.5,1){\vector(1,0){4}}
\put(4,3){\vector(1,0){4}}
\put(5.5,5){\vector(1,0){4}}

\put(2.5,1){\line(3,4){3}}
\put(6.5,1){\line(3,4){3}}
\end{picture}
\end{center}

\begin{theo}[Caractérisation d'un groupe résoluble]

Si $H\<G$, alors $G$ est résoluble si et seulement si $H$ et $G/H$ le sont.
\end{theo}


\begin{defi}[Groupe simple]
\index{groupe!simple}
 
Un groupe est dit simple s'il n'est pas trivial, et qu'il ne contient aucun
sous-groupe distingué propre (ie : différent de $G$).
\end{defi}


\begin{theo}
 
Soit $G$ contenant deux sous-groupes distingués $H$ et $K$ avec $H\cap K =
\emptyset$ et $HK = G$. Alors $G \simeq H\times K$.
\end{theo}


\begin{theo}

 Soit $G = H\times K$ avec $H,K < G$. Soient $H_1 < H$ et $K_1 < K$. Alors :
\begin{enumerate}
 \item $H_1\times K_1 \< H/K$
 \item $G / (H_1 \times K_1) \simeq (H/H_1) \times (K/K_1)$
\end{enumerate}
\end{theo}

\begin{proof}\ 

On définit les projections :
\begin{equation*}
\pi_1 : 
\begin{array}{rcl}
G     & \longrightarrow & H/H_1 \\
(h,k) & \longmapsto     & hH_1 =: \overline{h}
\end{array}
\text{ , } \pi_2 : 
\begin{array}{rcl}
G     & \longrightarrow & K/K_1 \\
(h,k) & \longmapsto     & kK_1 =: \overline{k}
\end{array}
\end{equation*}

Et on considère :
\begin{equation*}
\varphi : 
\begin{array}{rcl}
G     & \longrightarrow & (H/H_1) \times (K/K_1) \\
(h,k) & \longmapsto     & (\pi_1(h,k), \pi_2(h,k)) =:
(\overline{h},\overline{k})
\end{array}
\end{equation*}

Comme $\pi_1$ et $\pi_2$ sont des homomorphismes, $\varphi$ est aussi un
homomorphisme. De plus :
\begin{equation*}
\begin{array}{rl}
 \Ima\varphi =& (H/H_1) \times (K/K_1) \\
 \Ker\varphi=& H_1 \times K_1
\end{array}
\end{equation*}

On peut ainsi conclure par le premier théorème d'isomorphisme.

Le premier point du théorème se démontre facilement.
\end{proof}

\begin{example}[Application]
 Si $G = H\times K$, on a $G/(H\times \{e\}) \simeq K$.
\end{example}

\subsection{Actions de groupes}
\vspace{0.5em}

\begin{defi}[Action d'un groupe $G$ sur un ensemble $X$]
\index{action d'un groupe}

 On dit que $G$ agit sur $X$ s'il existe une fonction $\cdotp$ telle que :

\begin{enumerate}
 \item $  \begin{array}{rrcl}
         \cdotp \: :&  G\times X& \longrightarrow & X \\
      &     (g,x)    & \longmapsto     & g\cdotp x
          \end{array}       $
 \item $\forall g,h \in G,\ \forall x \in X,\ g\cdotp(h\cdotp x) = (gh)\cdotp x$
 \item $\forall x \in X,\ e\cdotp x = x$
\end{enumerate}
\end{defi}

\begin{example}[Remarque]
 Il existe toujours une action triviale telle que $\forall g\in G,\ \forall x
\in X,\ g\cdotp x = x$.
\end{example}
\begin{example}[Autre point de vue]
Une action peut également être vue comme un homomorphisme entre $G$ et
$\mathcal{S}_X$.
\begin{displaymath}
 \begin{array}{rrclrcl}
             \varphi :&G &\longrightarrow&\mathcal{S}_X & & & \\
 &  g &\longmapsto & \varphi(g) : & X &\longrightarrow& X \\
 &    &            &              & x &\longmapsto    & \varphi(g)(x) = g\cdotp
x
 \end{array}
\end{displaymath}

\end{example}

\begin{defi}[Action transitive]\index{action!transitive}

 On dit que $G$ agit transitivement sur $X$ si :
\begin{displaymath}\forall x,y \in X,\ \exists g\in
G,\ g\cdotp x = y\end{displaymath}
\end{defi}

\begin{defi}[Action fidèle]\index{action fidèle}

 On dit que $G$ agit fidèlement sur $X$ si :
\begin{displaymath}\forall g\in G, \left[ (\forall x \in X, g\cdotp x = x)
\Rightarrow (g =
e) \right] \end{displaymath}
\end{defi}

\begin{example}[Remarques] \ 
 \begin{itemize}
  \item Avec le second point de vue, une action est fidèle si et seulement si
$\Ker\varphi = \{e\}$, ie : $\varphi$ injective.
 \item Si $G$ n'agit pas fidèlement, on peut quotienter par $\Ker(\varphi)$
pour obtenir une action fidèle.
 \item Si $G$ n'agit pas transitivement, on peut introduire la relation
d'équivalence suivante :
\begin{displaymath} x\mathcal{R}y \Longleftrightarrow \exists g \in G, g\cdotp x
= y \end{displaymath}
Les classes d'équivalences sont appellée les orbites. Pour $x \in X$ on note
$\omega (x) := \{g\cdotp x \tq g\in G\}$ son orbite.\index{orbite}
 \end{itemize}
\end{example}

\begin{example}[Exemples]\ 
\begin{enumerate}
  \item $\GL_n(\R)$ agit transitivement sur $\R^n\-\{0\}$. En
particulier, il n'y a qu'une seule orbite qui est $\R^n\-\{0\}$.
  \item On rappelle que $\O(n) := \{M\in \GL_n (\R) \tq \t MM = \Id_n\}$. Les
orbites de $\O(n)$ sont les sphères centrées en l'origine.
 \end{enumerate}
\end{example}

\begin{defi}[Stabilisateur de $x$ dans $G$]\index{groupe!stabilisateur}

 Soient $G$ agissant sur $X$ et $x \in X$. On appelle stabilisateur de $x$ dans
$G$ l'ensemble :
\begin{displaymath} \Stab_G(x) := \{g\in G \tq g\cdotp x = x\} \end{displaymath}
\end{defi}

\begin{prop}[Stabilisateur]

 Le stabilisateur de tout élément de $X$ est un sous-groupe de $G$.
\end{prop}

\begin{proof}
 Vérifier chaque axiome.
\end{proof}

\begin{prop}[Stabilisateur - Orbite]

 L'application suivante est une bijection :
\begin{displaymath}  \begin{array}{rrcl}
             \varphi :& G/(\Stab_G(x)) &\longrightarrow& \omega (x) \\
           &   \overline{g}   &\longmapsto    &g\cdotp x
             \end{array} \end{displaymath}
\end{prop}

\begin{proof}\ 

 On commence par montrer que $\varphi$ est bien définie.
Soient $g,g' \in G$ tels que $g^{-1}g' \in \Stab_G (x)$. Alors $g\Stab_G(x) =
g'\Stab_G(x)$ donc $(g^{-1}g')\cdotp x = x $ et $ g' \cdotp x = g
\cdotp x$. Ainsi les éléments d'une même classe modulo le stabilisateur
agissent de la même manière sur $X$.

La surjectivité étant évidente, il reste à montrer l'injectivité. Soient $g,g'
\in G$ tels que $g\cdotp x = g' \cdotp x$. Alors $x = (g^{-1}g')\cdotp x$ et
$(g^{-1}g') \in \Stab_G(x)$. Ainsi si deux éléments agissent de la même manière
sur $X$, alors ils sont dans la même classe d'équivalence.
\end{proof}

\begin{example}[Remarque]
 Si $G$ est fini, on a $|\omega(x)| = |G| / |\Stab_G(x)|$.
\end{example}

\begin{example}[Exemples]
\begin{itemize}
\item Soit $H < G$ tel que $H$ agit sur $G$ par multiplication à gauche. Alors
$\omega (x) = Hx$ et $\Stab_H(x) = \{e\}$.
 
\item $G$ agit sur $G$ par conjugaison, ie : $g\cdotp x = gxg^{-1}$. Les orbites
sont alors les classes de conjugaison et $\Stab_G(x) = \Cen_x$ car $(gxg^{-1} =
x) \Leftrightarrow (gx = xg)$. Le nombre de conjugués est $|\omega(x)| =
[G:\Cen_x] = |G| / |\Cen_x|$ si $G$ est fini.
\end{itemize} \end{example}

\begin{theo}[Cayley - 1878]
\index{Cayley, théorème de}

 Soit $G$ un groupe. Alors :
\begin{itemize}
 \item $G$ s'injecte dans $\mathcal{S}_G$.
 \item Si $G$ est fini, $|G| = n$, alors $G$ est isomorphe à un sous-groupe de
$\mathcal{S}_n$.
\end{itemize}
\end{theo}

\begin{prop}

 Pour $n \geq 5$, les cycles de longueur 3 sont conjugués dans $A_n$.
\end{prop}

\begin{lemm}

 Le groupe $A_n$ est $(n-2)$ transitif sur $(1,\cdots,n)$.

 Cela signifie que l'on choisit $n-2$ éléments que l'on envoie transitivement
sur $n$ éléments. Si $a_1, \cdots , a_{n-2}$ sont distincts et $b_1,
\cdots,b_{n-2}$ aussi, alors : 
\begin{displaymath} \exists \sigma \in A_n,\ \forall i \in [ 1, n-2],\ \sigma
(a_i) =
b_i \end{displaymath}
\end{lemm}

\begin{proof}
 Si la permutation $\sigma' \in S_n$ qui envoie chaque $a_i$ sur $b_i$ est dans
$A_n$ il n'y a rien à montrer. Sinon on considère $\tau = (a_{n-1},a_n)$ et
ainsi $\sigma' \circ \tau =: \sigma$ convient.
\end{proof}

\begin{example}[Exemple]
 Soit $H < G$ et $G/H$ les classes à gauche modulo $H$.

$G$ agit sur $G/H$ par $g\cdotp (aH) = (ga)H$. Cette action est transitive mais
elle n'est en général pas fidèle. Si $\varphi : G \longrightarrow
\mathcal{S}_{G/H}$, alors $\Ker (\varphi) = \cup_{a\in G} aHa^{-1}$.

En particulier, $G/\Stab_G(x) \simeq \omega(x)$ par une seule bijection et donc
l'action de $G$ sur $\omega(x)$ ne peut être que l'action naturelle de $G$ sur
$G/\Stab_G(x)$.
\end{example}

\begin{prop}[Application]

 Si $G$ est un groupe infini, $H < G$, $H\neq G$, et $[G:H] = n < \infty$,
alors $G$ n'est pas simple.
\end{prop}

\begin{proof}
 $G$ agit sur $G/H$ d'où un homomorphisme $\varphi : G \longrightarrow
\mathcal{S}_{G/H} \simeq \mathcal{S}_n$ dont le noyau est un sous-groupe
distingué de $G$ qui ne peut être $\{e\}$ car $G$ est infini.
\end{proof}

\begin{example}
 Si $X$ est l'ensemble des sous-groupes d'un groupe $G$, on peut faire agir $G$
sur $X$ par automorphisme intérieur (ie : par conjugaison).

Si $H\in X, g \in G, g\cdotp H = ghg^{-1}$. Le stabilisateur de $H$ est alors
son normalisateur.
\end{example}

\subsection{Groupe abélien de type fini}
\vspace{0.5em}

\begin{defi}[Groupe abélien de type fini]\index{groupe!abélien de type fini}

 Un groupe abélien est dit de type fini s'il existe une famille finie
$(g_1,\cdots,g_n)$ qui l'engendre, ie : $G=\ <g_1, \cdots,g_n >$.
\end{defi}

\begin{example}[Remarque]
 On note $\Z^n := \Z \times \cdots \times \Z =: \Z \oplus \cdots \oplus \Z$
et on définit l'homomorphisme surjectif :
\begin{displaymath} \begin{array}{rrcl}
           \varphi :&  \Z^n &\longrightarrow& G \\
         &   (z_1,\cdots,z_n) &\longmapsto& z_1g_1 + z_2g_2 + \cdots z_ng_n  
           \end{array}\end{displaymath}
\end{example}

\begin{theo}[Classification des groupes abéliens de type fini]

 $G$ est un groupe abélien de type fini si et seulement si :
\begin{displaymath}G \simeq {}^\Z / _{d_1 \Z} \oplus \cdots \oplus
{}^\Z/_{d_k\Z} \oplus \Z \oplus \cdots
\oplus \Z \end{displaymath}

Où pour chaque $i$ on a $d_i > 0$ et $d_i \mid d_{i+1}$
\end{theo}


