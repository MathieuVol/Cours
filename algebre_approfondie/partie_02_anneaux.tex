\section{Anneaux}
\vspace{0.5em}

\subsection{Définitons}
\vspace{0.5em}

\begin{defi}[Anneau]
\index{anneau}

 Soit $A$ un ensemble et soient $+$ et $\cdotp$ deux lois internes. Le triplet
$(A,+,\cdotp)$ est un anneau si :
\begin{description}
 \item [A1] $(A,+)$ est un groupe commutatif (dont le neutre est noté $0$),
 \item [A2] $(A,\cdotp)$ est un monoïde, ie :
       \begin{enumerate}
        \item $\exists 1 \in A,\ \forall a\in A,\ 1\cdotp a = a = a \cdotp 1$,
        \item la loi $\cdotp$ est associative. 
       \end{enumerate}
 \item [A3] La loi $\cdotp$ est distributive par rapport à la loi $+$, ie :
$\forall
a,b,c \in A$ :
       \begin{enumerate}
       \item $a \cdotp (b+c) = (a\cdotp b) + (a\cdotp c)$,
       \item $(b+c) \cdotp a = (b\cdotp a) + (c\cdotp a)$.
       \end{enumerate}
\end{description}

Si la loi $\cdotp$ est commutative, on dit que l'anneau est commutatif.
\index{anneau commutatif}

Si $(A\-\{0\}, \cdotp)$ est un groupe, on dit que c'est un corps.
\index{corps}
\end{defi}

\begin{example}[Exemples]\ 

\begin{enumerate}
 \item L'anneau trivial $(\{0,1\},+,\cdotp)$, qui est même un corps.
\item $\Z, \R, \mathbb{C}$ sont des anneaux munis des lois habituelles.
\item $(\Z / n\Z, +,\cdotp)$ sont des anneaux pour $n > 1$, ce sont des corps
pour $n$ premier.
\item On peut définir l'anneau des fonctions $\mathcal{A}(X,A) =
\{f:X\rightarrow A\}$ où $A$ est un
anneau et $X$ un ensemble.
\item Les suites à valeurs dans $\R, \Z,$ ou $ \mathbb{C}$.
\item Les fonctions de $\R$ à valeurs dans $\R, \Z,$ ou $\mathbb{C}$.
\item Les matrices carrés de taille $n$ à coefficients dans un anneau (ce n'est
pas un anneau commutatif).
\item \begin{displaymath}
       \mathcal{Q} := \left\lbrace \begin{pmatrix}
                               \alpha & -\overline{\beta} \\
                               \beta  & \overline{\alpha}
                              \end{pmatrix} \ \Big|\  \alpha,\beta
\in\mathbb{C} \right\rbrace \text{ est le corps (non commutatif) des
quaternions.}
      \end{displaymath}
\item Si $A$ et $B$ sont des anneaux, on peut considérer l'anneau produit
direct $A\times B$.
\item Si $A$ est un anneau, on définit $A[X]$ :
 \begin{enumerate}
  \item comme l'ensemble des suites à  valeurs dans $A$ qui s'annulent à partir
  d'un certain rang,
  \item  ou bien comme  l'ensemble des polynômes à une indéterminée
  $X$ et à coefficients dans $A$.
  \end{enumerate}
On identifie alors $X^k$ avec $(\underbrace{0,\cdots,0}_{k},1,0,\cdots)$.

La multiplication suivante garantie la distributivité : $aX^k \cdotp bX^l =
(ab)X^{k+l}$.
\item Si $A$ est un anneau, on définit $A\left[\left[X\right]\right]$ :
 \begin{enumerate}
  \item comme l'ensemble des suites à  valeurs dans $A$,
  \item  ou bien comme l'ensemble des séries formelles à coefficients
dans $A$,
  
ie : $A[[X]] := \left\lbrace\sum_{k\in\mathbb{N}}a_kX^k \tq \forall k \in
\mathbb{N},
a_k \in A\right\rbrace$.
  \end{enumerate}
\end{enumerate}
\end{example}

\begin{example}[Remarques]

 On montre facilement que $\forall x,y \in A$ :
\begin{itemize}
 \item $x\cdotp 0 = 0  = 0\cdotp x$
 \item $(-1)\cdotp x = -x = x \cdotp (-1)$
 \item $(-x)\cdotp y  = -(x\cdotp y) = x \cdotp (-y)$
 \item $(-x) \cdotp (-y) = x\cdotp y$
\end{itemize}
\end{example}

\begin{defi}[Inversible - Diviseur de zéro]
\index{inversible}\index{diviseur de zéro}
\begin{itemize}
 \item Un élément d'un anneau sera dit inversible s'il est inversible pour la
multiplication.
\item On appelle diviseur de zéro un élément $x \neq 0$ tel qu'il existe $y\neq
0$ tel que $xy = 0 = yx$.
\item 
On notera $A\x$ l'ensemble des éléments inversibles de $A$.
\end{itemize}
\end{defi}

\begin{example}[Remarque]
 $A\x$ est un groupe pour la multiplication. Ainsi, $A$ est un corps si et
seulement si $A\x = A\-\{0\}$.
\end{example}

\begin{defi}[Anneau intègre]
\index{anneau!intègre}
\index{domaine d'intégrité|see{anneau intègre}}

 Un anneau intègre est un anneau sans diviseur de zéro. Lorsqu'il est de plus
commutatif, il est parfois appelé domaine d'intégrité.
\end{defi}

\begin{example}[Notations]\ 


\begin{align*}
&\text{Pour $n \in \Z$ on définit : }& &
 nx := \left\{
\begin{array}{rl}
\underbrace{x + \cdots + x}_{n\text{ fois}} & \text{si } n > 0,\\
0 & \text{si } n = 0,\\
\underbrace{(-x) + \cdots + (-x)}_{(-n)\text{ fois}} & \text{si } n < 0.
\end{array} \right. \\
&\text{Et pour $n \in \mathbb{N}$ on définit : } & &
x^n := \left\{
\begin{array}{rl}
\underbrace{x \cdots x}_{n\text{ fois}} & \text{si } n > 0,\\
1 & \text{si } n = 0.
\end{array} \right.\\
&\text{Si de plus $x$ est inversible, on définit pour $n\in \mathbb{N}$ :}
& & x^{-n} := \underbrace{x^{-1} \cdots x^{-1}}_{n\text{ fois}} 
\end{align*}
\end{example}

\subsection{Sous-anneaux et idéaux}
\vspace{0.5em}

\begin{defi}[Sous-anneau]
\index{anneau!sous-anneau}

 Soient $A$ un anneau, et $B \subset A$. On dit que $B$ est un sous-anneau de
$A$ si :
\begin{enumerate}
 \item $(B,+)$ est un sous-groupe de $(A,+)$,
 \item $B$ est stable par multiplication,
 \item $1 \in B$.
\end{enumerate}
\end{defi}
 
\begin{defiprop}[Sous-anneau engendré]
\index{anneau!sous-anneau!engendré}
 \begin{enumerate}
  \item Une intersection quelconque de sous-anneaux est un sous-anneau.
  \item Si $X \subset A$, l'intersection de tous les sous-anneaux de $A$ qui
contiennent $X$ est un sous-anneau de $A$ appelé anneau engendré par $X$.
 \end{enumerate}
\end{defiprop}

\begin{example}[Exemples]
\begin{enumerate}            
 \item Le sous-anneau engendré par $\{1,i\} \subset \mathbb{C}$ est $\Z[i] = \{a
+ ib \tq a,b \in \Z \}$.
\item Si $I$ est un intervalle de $\R$, l'ensemble
des fonctions continues de $\mathcal{A}(I,\mathbb{C})$ est un sous-anneau de
$\mathcal{A}(I,\mathbb{C})$.
\end{enumerate}
\end{example}

\begin{defi}[Idéal]
\index{idéal}
 Une partie $I$ d'un anneau $A$ est appelé idéal à gauche si :
\begin{enumerate}
 \item C'est un sous-groupe additif de $A$,
 \item $\forall a \in A, \forall x \in I, a\cdotp x \in I$.
\end{enumerate}
Il est appelé idéal à droite si :
\begin{enumerate}
 \item C'est un sous-groupe additif de $A$,
 \item $\forall a \in A, \forall x \in I, x\cdotp a \in I$.
\end{enumerate}
Un idéal à droite et à gauche est dit bilatère.
\end{defi}

\begin{defiprop}[Idéal engendré]
\index{idéal!engendré}
 \begin{itemize}
  \item Un intersection d'idéaux à gauche (resp. à droite, bilatère) de $A$ est
un idéal à gauche (resp. à droite, bilatère) de $A$.
 \item Si $X \subset A$, l'intersection de tous les idéaux à gauche (resp. à
droite, bilatères) qui contiennent $X$ est un idéal à gauche (resp. à droite,
bilatère) appelé idéal engendré par $X$ et noté $(X)$. Si $X=\{x\}$ on note
simplement $(x)$.
 \end{itemize}
\end{defiprop}

\begin{example}[Exemples]
 \begin{enumerate}
  \item Si $a\in \Z$, les multiples de $a$ forment l'idéal $(a)$ engendré
par $a$.
  \item Si $P\in \mathbb{K}[X]$, alors $(P)$ est un idéal de $\mathbb{K}[X]$.
 \end{enumerate}
\end{example}

\begin{defi}[Idéal principal - de type fini]
\index{idéal!principal} \index{idéal!de type fini} \index{anneau!principal}

 Un idéal est dit principal s'il est engendré par un seul élément.

 Un idéal est dit de type fini (ou de génération finie) s'il est engendré par
une partie finie.

 Un anneau dont tous les idéaux sont principaux est dit principal.
\end{defi}

\begin{prop}[$\Z$ est principal]

 Les idéaux de $\Z$ sont principaux.
\end{prop}

\begin{proof}

 Tout sous-groupe de $(\R,+)$ qui n'est pas dense dans $\R$ est de la forme
$a\Z$.
\end{proof}


\begin{defi}[Idéaux somme et produit]
\index{idéal!somme} \index{idéal!produit}

 Soient $I$ et $J$ deux idéaux. On définit :
\begin{enumerate}
 \item L'idéal somme $I + J := \{i + j \tq i\in I, j
\in J\}$ engendré par $I\cup J$.
 \item L'idéal produit $(IJ) := \left\{\sum_{k=1}^{n}i_kj_k \tq i_k \in I, j_k
\in J, n \in \mathbb{N}^* \right\}  $ engendré par les produits. 
\end{enumerate}
\end{defi}

\begin{defi}[Idéaux maximaux]
\index{idéal!maximal}

 Un idéal $I$ d'un anneau $A$ est dit maximal si :
\begin{enumerate}
 \item $I \neq A$.
 \item Les seuls idéaux contenant $I$ sont $I$ et $A$.
\end{enumerate}
\end{defi}

\begin{example}[Remarque]
On peut donner une définition équivalente :
\begin{enumerate}
 \item $1 \notin I$.
 \item $(B \mbox{ idéal},\ B\supset I) \Rightarrow (B = I \mbox{ ou } B = A)$.
\end{enumerate}
\end{example}

\begin{theo}[Existence d'un idéal maximal]

 Tout idéal est inclu dans un idéal maximal.
\end{theo}

\begin{proof} (C'est une ``Zornette'')

Soit $I$ un idéal d'un anneau $A$ et soit $M$ l'ensemble des idéaux contenant
$I$ et ne contenant pas $1$.
\begin{displaymath}  M = \{B \subset A \tq B \mbox{ idéal},\ B \supset I,\ 1
\notin B\} \end{displaymath}

Soit alors $(I_n)$ une suite croissante (pour l'inclusion) d'idéaux de $M$. On
pose $J := \bigcup_n I_n$. On a bien $J$ est un idéal, $1 \notin J$, $J\supset
I$, donc $J \in M$.

Ansi les hypothèses du lemme de Zorn étant vérifiées, on peut conclure sur
l'existence d'un idéal maximal contenant $I$.
\end{proof}

\begin{example}[Exemple] 
  Dans l'anneau $\Z$ : $7\ \Z$, $3\ \Z$, et $2\ \Z$ sont des idéaux maximaux de
$84\ \Z$ :

\begin{equation*}
  84\ \Z \subset \left\lbrace
      \begin{array}{l}
      21\ \Z \subset \left\lbrace
         \begin{array}{l}
         7\ \Z  \\
         3\ \Z 
         \end{array}\right.\\
      12\ \Z \subset 4\ \Z \subset 2\ \Z 
      \end{array}\right.
\end{equation*}


\end{example}

\begin{defi}[Radical de Jacobson - Anneau local]
\index{radical de Jacobson} \index{anneau local}

 Soit $A$ un idéal commutatif. On appelle radical de Jacobson, et on note $\Rad
(A)$ l'intersection de tous les idéaux maximaux de $A$.

Si $A$ n'a qu'un seul idéal maximal, on dit que $A$ est un anneau local.
\end{defi}

\subsection{Morphismes d'anneaux et quotients}
\vspace{0.5em}


\begin{defi}[Morphisme d'anneaux]
\index{morphisme!d'anneau} \index{morphisme!de corps}

 Soient $A$ et $B$ deux anneaux. Une application $f : A \rightarrow B$ est un
morphismes d'anneaux si :
\begin{enumerate}
 \item $\forall x,y \in A,\ f(x+y) = f(x) + f(y)$
 \item $\forall x,y \in A,\ f(xy) = f(x) f(y)$
 \item $f(1_A) = 1_B$
\end{enumerate}

En particulier, si $K$ et $L$ sont des corps, on parle de morphismes de corps.
\end{defi}

\begin{example}[Remarques]
\begin{enumerate}
 \item Un morphisme d'anneaux est en particulier un morphisme entre les
groupes $(A,+)$ et $(B,+)$.
 \item Si $f : A \rightarrow B$ et $g : B \rightarrow C$ sont des morphismes
d'anneaux, alors $g\circ f : A \rightarrow C$ est un morphisme d'anneaux.
 \item Si $f : A \rightarrow B$ est un morphisme d'anneau bijectif, alors
$f^{-1} : B \rightarrow A$ est aussi un morphisme d'anneau (bijectif).
 \item En notant $A\x$ (resp. $B\x$) l'ensemble des inversibles de
$A$ (resp. $B$), et si $f : A \rightarrow B$ est un morphisme d'anneaux, alors
$f(A\x) \subset B\x$.

Plus précisément, $f$ induit un morphisme de groupes $A\x \rightarrow
f(A\x)$. En effet si $x, x^{-1} \in A\x$ alors $f(x) f(x^{-1}) =
f(xx^{-1}) = f(1_A) = 1_B$ d'où $f(x) \in B\x$
\item Si $f : A \rightarrow B$ est un morphisme d'anneau, et si $A$ est un
corps, alors $f$ est injectif. 

En effet, si $x,y \in 1$ tels que $f(x) = f(y)$, alors $f(x) - f(y) = f(x-y) =
0$ d'où $x-y \notin A\x$ or $A\x = A \ \{0\}$ et donc $x-y = 0
\Rightarrow x = y$.
\item Soit $A$ un anneau et $B \subset A$ un sous-anneau. Soit $i : B
\rightarrow A, x \mapsto x$. Il existe une unique structure d'anneau sur $B$
telle que $i$ soit un morphisme d'anneau.
\end{enumerate}
\end{example}

\begin{defi}[Noyau et Image d'un morphisme d'anneau]
\index{morphisme!d'anneau!noyau et image}

 Soit $f : A \rightarrow B$ un morphisme d'anneau. On définit :
\begin{itemize}
 \item L'image $\Ima f = \{y \in B \tq \exists x \in A,\ y = f(x)\}$
 \item Le noyau $\Ker f = \{x \in A \tq f(x) = 0\}$
\end{itemize}

On précise bien que le noyau est celui du morphisme de groupe $(A,+)
\rightarrow (B,+)$.
\end{defi}


\begin{prop}
 
\begin{enumerate}
 \item L'image d'un morphisme d'anneau est un sous-anneau.
 \item Le noyau d'un morphisme d'anneau est un idéal bilatère.
\end{enumerate}
\end{prop}

\begin{proof}
 On rappelle que ce sont des sous-groupes additifs.
\begin{enumerate}
 \item $1_B = f(1_A) \in \Ima f$, et $\forall x',y' \in \Ima f,\ \exists x,y
\in
A,\ x' = f(x),\ y' = f(y)$ donc $x'y' = f(x)f(y) = f(xy) \in \Ima f$.
 \item $\forall x \in \Ker f,\ \forall a,b \in A,\ f(axb) =
f(a)\underbrace{f(x)}_{= 0} f(b) = 0$.
\end{enumerate}
\end{proof}

\begin{prop}[Anneau quotient]
\index{anneau!quotient}
 
Soit $A$ un anneau, $I$ un sous-groupe de $(A,+)$, et $\pi : x \mapsto x + I$.
Alors $A/I$ possède une structure d'anneau telle que $\pi$ soit un morphisme si
et seulement si $I$ est un idéal bilatère.

Dans ce cas cette structure d'anneau est unique, et on dit que $A/I$ est
l'anneau quotient de $A$ par $I$.
\end{prop}

\begin{proof}
On a déjà vu que $(A/I,+)$ est un sous-groupe abélien, et que $\pi$ est un
morphisme de groupe.

$(\Leftarrow)$ : $\Ker(\pi) = I$ donc si $\pi$ est un morphisme d'anneau, alors
$I$ est un idéal bilatère.

$(\Rightarrow)$ : Soit $I$ un idéal bilatère. Alors on veut que $\pi$ vérifie
les propriétés suivantes :
\begin{itemize}
 \item $\pi (1_A) = 1_{A/I} = 1_A +I,\ \pi(xy) = xy +I,\ \pi(x) = x + I,\
\pi(y) = y+I$
 \item On doit avoir la multiplication suivante :
  $(x + I) (y+I)= (xy+I)$
 \item  Soient alors $i,j\in I$, on a $(x+i)(y+j) = \underbrace{xy + xj + iy +
ij}_{\in I} \in (xy + I)$\\ c'est à dire qu'on a bien $\pi(x)\pi(y) = \pi(xy)$
 \item La distributivité à gauche (et de même à droite) est vérifiée car : \\
$(x+I)((y+I)+(z+I)) =
\pi(xy + xz) = (x+I)(y+i) + (x+I)(z+I) $
\end{itemize}
Ce raisonnement conduit à construire $\pi$ qui convient d'où l'existence, et ne
laisse aucun choix d'où l'unicité.
\end{proof}


\begin{prop}[Factorisation d'un morphisme d'anneau]
\index{morphisme!d'anneau!factorisation}

Soit $f : A \rightarrow B$ un morphisme d'anneau, et $p : A \rightarrow
A'$ un morphisme d'anneau surjectif.

Il existe un morphisme d'anneau $g : A'\rightarrow B$ tel que $f = g \circ
p$ si et seulement si $\Ker p \subset \Ker f$.
\end{prop}

\setlength{\unitlength}{1em}
\begin{center}
 \begin{picture}(10,6)
\put(1,0){$A'$}
\put(1,5){$A$}
\put(6,5){$B$}

\put(2.5,1.5){\vector(1,1){3}}
\put(2.5,5){\vector(1,0){3}}
\put(1.5,4.5){\vector(0,-1){3}}

\put(3.5,5.5){$f$}
\put(0.5,3){$p$}
\put(4.5,2.5){$g$}
 \end{picture}
\end{center}

\begin{proof}\ 

S'il existe un tel $g$, on vérifie que si $x \in \Ker p$ alors $f(x) =
g(p(x)) = g(0) = 0$.

Réciproquement si $\Ker p \subset \Ker f$, alors $\forall a \in A',\ \exists x
\in A,\ p(x) = a$ (car $p$ surjectif). Ainsi on pose $g(a) = f(x)$ et on vérifie
que $g$ est bien définie. En effet si $p(x') = p(x) = a$, on a $p(x') - p(x) =
p(x' - x) = 0$ et $x' - x \in \Ker p \subset \Ker f$. D'où $f(x' - x) =
f(x') - f(x) = 0$ et $f(x') = f(x)$.
\end{proof}

\begin{prop}[Propriété universelle du quotient]
\index{anneau!quotient}
 
Soient $A$ est un anneau, $I$ un idéal bilatère de $A$, et $\pi : A
\longrightarrow A/I$ est la projection canonique. Si $f : A \longrightarrow B$
est un morphisme d'anneau tel que $I \subset \Ker(f)$, alors il existe un
morphisme d'anneau $g : A/I \longrightarrow B$ tel que $g\circ \pi = f$ que
l'on note généralement $\tilde{f}$. 
\end{prop}
\begin{proof}
 C'est une conséquence de la propriété précédante en prenant $p = \pi$.
\end{proof}


\subsection{Algèbre}
\vspace{0.5em}

\begin{defi}[$A$-Algèbre]
\index{algèbre}
 
Soit $A$ un anneau commutatif. On dit qu'un ensemble $B$ est une $A$-algèbre si
:
\begin{enumerate}
 \item $B$ est un anneau,
 \item $\exists \eta : A \rightarrow B$ un morphisme d'anneau tel que
$\forall x \in A,\ \forall y \in B,\ \eta(x)y = y\eta(x)$.
\end{enumerate}

On dit alors que $\eta$ est un morphisme structural.
\end{defi}

\begin{defi}[Morphisme de $A$-algèbres]
\index{morphisme!d'algèbre}
 
Si $B$ et $B'$ sont deux $A$-algèbres, on dit que $f : B \rightarrow B'$ est
un morphisme de $A$-algèbres si :
\begin{enumerate}
 \item $f$ est un morphisme d'anneau,
 \item $g\circ \eta_B = \eta_{B'}$.
\end{enumerate}
\end{defi}

\begin{example}[Exemples]
\begin{itemize}
 \item $A$ est une $A$-algèbre si $A$ est un anneau commutatif ($\eta =
\Id_A$).
 \item Tout anneau $A$ est une $\Z$-algèbre ($\eta : \Z \owns n
\longmapsto n \cdotp 1_A \in A$).
 \item Les suites à valeurs dans un anneau $A$ forment un anneau pour les
opérations terme à terme. Si de plus $A$ est commutatif, c'est une $A$-algèbre
avec $\eta(a) = (a)_{n\in\N}$ (la suite constante).
 \item $\M_n(A)$, les matrices carrées de taille $n$ à coefficients dans un
anneaux $A$ forment un anneau avec les opérations habituelles sur les matrices.
Si de plus $A$ est commutatif, c'est une $A$-algèbre avec $\eta(a) =
a\cdotp\Id$. Le lien entre ces matrices et des applications linéaires est à
définir.
 \item $A[X]$, les polynômes à coefficients dans $A$ est un anneau. Si de plus
$A$ est commutatif, c'est une $A$-algèbre avec $\eta(a) = a\cdotp X^0$.
\end{itemize}
\end{example}

\begin{prop}[Morphisme d'évaluation]
\index{morphisme!d'algèbre!morphisme d'évaluation}
 
Soient $A$ un anneau commutatif, $B$ une $A$-algèbre, et $x\in B$. Il existe un
unique morphisme $e(x) : A[X] \longrightarrow B$ tel que $e(x)(X^1) = x$.

Cela donne un sens aux notions de l'évaluation et de fonction polynômiale.
\end{prop}

\begin{proof}
On note $\eta : A \longrightarrow B$ le morphisme structural de $B$.
\begin{itemize}
 \item [(unicité)]
Soit $e(x)$ un tel morphisme et soit $P(X) = \sum_{k=1}^d a_kX^k$.

En remarquant que $e(x)((a+b)X) = \eta(a+b)e(X) = (\eta(a) + \eta(b))x$. On a
:
\begin{align*}
  e(x)(P(X)) & \quad = \quad e(x)(\sum_{k=1}^d a_kX^k)
             & \quad = \quad & \sum_{k=1}^d e(x)(a_kX^k)\\
             & \quad = \quad \sum_{k=1}^d \eta(a_k) e(x)(X^k)
             & \quad = \quad & \sum_{k=1}^d \eta(a_k) (e(x)(X))^k \\
             & \quad = \quad \sum_{k=1}^d \eta(a_k)x^k
\end{align*}

\item[(existence)]
On a vu qu'en fixant $e(x)(X) = x$ on fixe $e(x)(P(X))$ ($\forall P$). Il reste
à vérifier qu'on a bien un morphisme en posant $e(x)(aX^0) = \eta(a)$.
\end{itemize} \end{proof}

\begin{example}[Remarque]
 En prenant $B = A$ on a l'évaluation classique.
\end{example}

\subsection{Généralité sur les anneaux commutatifs}
\vspace{0.5em}
 \boxed{\textbf{Dans toute cette partie les anneaux sont
commutatifs.}} \hfill

\vspace{0.5em}
\subsubsection{Formule du binôme de Newton}
\vspace{0.5em}
\begin{prop}[Binôme de Newton]
\index{Newton, formule du binôme}

Soit $A$ un anneau et $x,y\in A$ deux éléments qui commutent. Alors :
\begin{displaymath} (x+y)^n = \sum_{k=0}^n\ \Com_n^k\: x^ky^{n-k}
\end{displaymath}

C'est en particulier le cas si $A$ est commutatif ou bien si $y=1$.
\end{prop}

\begin{defi}[Nilpotence]\index{nilpotent}
 
Un élément $a$ d'un anneau $A$ est dit nilpotent s'il existe $n\in\N$ tel que
$a^n=0$. On appelle indice de nilpotence le plus petit de ces entiers, et on
note $\Nil(A)$ l'ensemble des nilpotents.
\end{defi}

\begin{example}[Remarque]\ 

On remarque que si $a$ est nilpotent de degrés $d$, alors $\forall x\in A$ on a
: $(ax)^d = a^dx^d = 0$ d'où $ax \in \Nil(A)$.

De même, si $a$ et $b$ sont nilpotents de degrés $d$ et $d'$, alors $\forall
n\geq d+d', (a+b)=0$ d'où $(a+b) \in \Nil(A)$.
\end{example}

\begin{prop}
 
$\Nil(A)$ est un idéal de $A$.
\end{prop}

\begin{example}[Remarque] Ce n'est pas le cas si $A$ n'est pas commutatif.
\end{example}

\subsubsection{Rappels}
\vspace{0.5em}

\begin{defi}[Diviseur de zéro - Anneau intègre]
\index{diviseur de zéro} \index{anneau intègre}
 
$a\in A$ est dit régulier si la multiplication par $a$ est injective. Dans le
cas contraire, on dit que $a$ est un diviseur de zéro.

Si tous les éléments non nuls de $A$ sont réguliers, alors on dit que c'est un
anneau intègre.
\end{defi}

\begin{prop}
 
Si $A$ est un anneau intègre, alors $A[X]$ est également intègre. De plus
$\deg(PQ) = \deg(P) \deg(Q)$.
\end{prop}

\begin{proof} Il suffit de montrer que $\deg(PQ) = \deg(P) + \deg(Q)$. Ce qui
est trivial si $P$ ou $Q$ est nul. Sinon on considère les coefficients (non
nuls) des termes de plus haut degrès dont le produit sera non nul.
\end{proof}

\subsubsection{Propriétée des idéaux}
\vspace{0.5em}

\begin{prop}
 
Si $X=(x_i)_{i\in I}$ est une famille quelconque d'éléments de $A$, alors :
\begin{displaymath} (X) = \left\lbrace\sum_{i\in F} a_i x_i \tq F \in
\mathcal{P}_{\text{finies}}(I), a_i \in A\right\rbrace
\end{displaymath}

En particulier si $I$ est fini, il de considérer la somme sur $i\in I$.
\end{prop}

\begin{prop}

Soit $f : A \longrightarrow B$ un morphisme d'anneaux.
\begin{enumerate}
 \item Si $J\subset B$ est un idéal, alors $f^{-1}(J)$ est un idéal contenant
$\Ker f$.
 \item Si de plus $f$ est surjectif, l'image d'un idéal $I\subset A$ est un
idéal $f(I) \subset B$. De plus l'application $\varphi$ qui associe à un idéal
$J\subset B$ l'idéal $f^{-1}(J)$ est une bijection :
\begin{displaymath}
 \varphi : \{\text{idéaux de } B\} \longrightarrow \{\text{idéaux de }A\text{ 
contenant }\Ker f\}
\end{displaymath}
\end{enumerate}
\end{prop}

\begin{proof}\ 

 \begin{enumerate}
  \item Soit $J$ un idéal de $B$. $\Ker f \subset f^{-1}(J)$ car $0 \in J$ et
$f^{-1}(J)$ est évidemment un sous-groupe de $(A,+)$. \\Soit alors $x\in
f^{-1}(J)$ et $a\in A$. On a $f(a) \in B$ et $f(x) \in J$ or $J$ est un idéal
de $B$ donc $f(ax) = f(a)f(x) \in J$ d'où $ax\in f^{-1}(J)$.
  \item \begin{enumerate}
  \item Soit $I$ un idéal de $A$ avec $f$ surjective. Alors
$f(I)$ est un sous-groupe de $B$. Il reste à vérifier que pour $\forall x \in
f(I)$ et $ \forall b\in B$ on a bien $bx \in f(I)$. Or il existe $y\in I,\
x=f(y)$ et comme $f$ est surjective, il exsite $a\in A,\ b=f(a)$. Ainsi, $ay \in
I$ car $I$ est un idéal et : $bx = f(a)f(y) = f(ay) \in f(I)$.
 \item Soit $\varphi : \{\text{idéaux de } B\} \longrightarrow \{\text{idéaux de
}A\text { contenant }\Ker (f)$ telle que $\varphi(J) = f^{-1}(J)$.\\
 Par la théorie des ensembles, on a $\varphi$ injective. Soit alors $I$ un
idéal de $A$ contenant $\Ker f$. On pose $J = f(I)$, on a clairement $I \subset
f^{-1}(J)$. Par l'absurde, soit $x\in f^{-1}(J)\- I$ alors il
exsite $y\in I,\ f(x)=f(y)$ donc $x-y \in \Ker f \subset I$ donc $x\in I$ ce
qui est absurde et conclut.
 \end{enumerate} \end{enumerate}
\end{proof}

\begin{example}[Exemple]\

$I=A$ est un idéal de $A$ tel que $1\in f(A)$. Donc $f(A)$ est un idéal si et
seulement si $f$ est surjective. \\ Par exemple, soit $i : \Z \hookrightarrow
\Q$, on a $i(\Z) = \Z $ qui n'est donc pas un idéal de $\Q$.
\end{example}


\subsubsection{Idéaux premiers et maximaux}
\vspace{0.5em}

\begin{prop}[Caractérisation d'un idéal maximal]
\index{idéal!maximal}
 
Un idéal $I$ d'un anneau $A$ est maximal si et seulement si $A/I$ est un corps.
\end{prop}

\begin{proof}
Soit $\pi : A \longrightarrow A/I$ le morphisme surjectif canonique. D'après la
proposition précédante, les idéaux de $A/I$ sont en bijection avec les idéaux
de $A$ contenant $I = \Ker \pi$.
\begin{align*}
 I \text{ est maximal } &\Leftrightarrow A  \text{ et } I  \text{ sont les
seuls idéaux contenant }I\\
&\Leftrightarrow A/I  \text{ contient exactement deux idéaux (qui sont donc }
\{0\}  \text{ et } A/I  \text{)}\\
&\Leftrightarrow A/I  \text{ est un corps} 
\\& \quad\ \: \text{(tout idéal engendré par } x\neq
0 \text{ contient alors }1  \text{ et }x  \text{ est donc inversible)}
\end{align*}

\end{proof}

\begin{defi}[Idéal premier]
\index{idéal!premier}
 
Un idéal $I$ d'un anneau $A$ est dit premier si $A/I$ est intègre.
\end{defi}

\begin{example}[Remarque] 
  C'est en
particulier le cas si $I$ est maximal.
\end{example}

\begin{prop}[Caractérisation d'un idéal premier]

Un idéal $I$ d'un anneau $A$ est premier si et seulement si :
\begin{displaymath}
 I\neq A \ \text{ et } \ \forall x,y \in A,(xy \in I) \Longrightarrow (x\in I
\text{ ou } y\in I)
\end{displaymath}
\end{prop}

\begin{proof}\

 Soit $I$ un idéal premier différent de $A$. Soient $x,y\in A$ tels que $ xy\in
I$. Alors
$\pi(xy) = 0 = \pi(x)\pi(y)$ or $A/I$ intègre donc $\pi(x) = 0$ ou $\pi(y) = 0$
(ie : $x \in I$ ou $y\in Y$).

Réciproquement, soit $\pi(x)\pi(y) = 0$ alors $xy \in I$ donc par hypothèse
$x\in I$ ou $y\in I$. Ainsi $\pi(x) = 0$ ou $\pi(y) = 0$ et $A/I$ est intègre.
\end{proof}

\begin{prop}
 
L'ensemble $A\-\{A\x\}$ des éléments non inversibles de $A$
est la réunion de tous les idéaux maximaux de $A$.
\end{prop}

\begin{proof}\
 
Un idéal maximal ne contient pas d'inversible sinon $1$ serait dans l'idéal, et
il ne serait pas maximal. Donc une première inclusion est satisfaite.

Réciproquement, $x\in A\x \Longleftrightarrow xA=A$, donc si $x$ n'est pas
inversible, $xA\neq A$ et l'idéal $xA$ est inclu dans un idéal maximal.
\end{proof}

\begin{prop}[Radical de Jacobson]
\index{radical de Jacobson}
 
L'intersection des idéaux maximaux de $A$ est l'ensemble des éléments $x\in A$
tels que $\forall y\in A, 1+xy$ est inversible.
\end{prop}

\begin{proof} (exo)
\end{proof}

\begin{prop}
 
L'intersection de tous les idéaux premiers de $A$ est l'ensemble $\Nil(A)$ des
nilpotents de $A$.
\end{prop}

\begin{proof} (exo)
\end{proof}


\subsubsection{Anneaux de fractions}
\vspace{0.5em}

Il s'agit ici de généraliser la construction de $\Q$ à partir de $\Z$.

\vspace{0.5em}

\begin{defi}[Partie multiplicative d'un anneau]
\index{anneau!partie multiplicative}

Une partie $S\in A$ est une partie multiplicative si : $1\in S$ et $\forall
x,y\in S, xy\in S$.

On définit une relation $\Rel$ sur $S\times A$ par : 
\begin{displaymath} (s,x)\Rel(s',x') \Longleftrightarrow \exists \sigma \in
S,\ \sigma s'x = \sigma s x' \end{displaymath}
\end{defi}

\begin{example}[Remarque]
La relation $\Rel$ est une relation d'équivalence. La réflexivité est la
symétrie sont évidentes. Si $(s,x)\Rel(s'x')$ et $(s',x')\Rel(s'',x'')$ alors
il existe $\sigma, \rho \in S,\ \sigma s x' = \sigma s' x$ and $\rho s' x'' =
\rho s''x'$.
Donc : \begin{displaymath}
\begin{array}{ccccc}
\sigma \rho s' s'' x &=& \rho s'' \sigma s' &=& \rho s''\sigma s x'\\
 &=& \sigma s \rho s'' x' &=& \sigma s \rho s' x'' \\
(\sigma \rho s') s'' x &=& (\sigma \rho s') s x'' & & \\
       \end{array}  \end{displaymath}
\end{example}

\begin{defi}[Fractions]
\index{anneau!fraction}
 
On définit les fractions à dénominateur dans une partie multiplicative $S$ par
$S^{-1}A = S\times A/\Rel$ et on note $x/s$ une classe d'équivalence.

On définit les opétions suivantes :
\begin{displaymath}
 \dfrac{x}{s} + \frac{x'}{s'} := \dfrac{s'x + sx'}{ss'} \quad \text{ et } \quad
 \dfrac{x}{s} \times \frac{x'}{s'} := \dfrac{xx'}{ss'}
\end{displaymath}
On pose $\eta : A \longrightarrow S^{-1}A$ tel que $\eta(a) = {a/1}$.
\end{defi}

\begin{defiprop}[Algèbre des fractions]
\index{anneau!algèbre des fractions}

L'ensemble $(S^{-1}A,+,\times)$ est un anneau commutatif et $\eta$ est un
morphisme d'anneau de sorte que $S^{-1}A$ soit une $A$-algèbre.

On l'appelle
l'aglèbre des fractions de $A$ à dénominateur dans $S$.
\end{defiprop}

\begin{example}[Remarque]
 Si $s\in S$, alors $\eta(s)$ est inversible d'inverse $1/s$.
\end{example}

\begin{prop}
 
Soient $A$ un anneau, $S$ une partie multiplicative de $A$, $B$ une $A$-algèbre
de morphisme structural $\eta' : A\longrightarrow B$ telle que l'image de tout
élément de $S$ est inversible dans $B$. 

Alors il exsite un unique isomorphisme de $A$-algèbres $f : S^{-1}A
\longrightarrow B$.
\end{prop}

\begin{proof}
On veut que $f : S^{-1}A \longrightarrow B$ soit un isomorphisme :
\begin{itemize}
 \item $f(\eta(a)) = f(1/a) = \eta'(a)$
 \item $\forall s \in S,\ f(1/s) = f(\eta(s)^{-1}) = f(\eta(s))^{-1} =
(\eta'(s))^{-1}$
\end{itemize}
Donc $\forall a\in A,\ \forall s \in S, f(a/s) = (f(a/1\times 1/s)) =
\eta'(a)\times(\eta'(s))^{-1}$. Ceci prouve l'unicité d'un tel morphisme s'il
existe. Il reste à vérifier qu'on a bien un ismorphisme $S^{-1}A \longrightarrow
B$.
\end{proof}

\begin{example}[Remarque]
 Si $A$ est un anneau intègre, alors $A\-\{0\}$ est une partie
multiplicative de $A$.
\end{example}

\begin{defiprop}[Corps des fractions d'un anneau intègre]
\index{anneau!corps des fractions}

On suppose que $A$ est un anneau intègre.
\begin{enumerate}
 \item L'anneau $\Frac A := (A\-\{0\})^{-1}A$ est un corps appelé
corps des fractions. De plus, $A$ s'identifie à sous-anneau de
$A\-\{0\}A$ par $a \mapsto a/1$.
 \item Si $S$ est une partie multiplicative ne contenant pas $0$, alors :\\
$S^{-1}A \simeq \{a/s \tq a\in A,\ s\in S\} \subset \Frac A$.
 \item Pour tout corps $K$ et $f : A  \hookrightarrow K$ injectif, il existe un
unique morphisme d'anneau $\Frac A \longrightarrow K$ qui prolonge $f$.
\end{enumerate}
\end{defiprop}

\begin{proof}\

 \begin{enumerate}
  \item Par construction, tous les éléments de $A$ sont inversibles.
  \item (exo, évident)
  \item Par la proposition précédante avec $S = A\-\{0\}A$, et en
remarquant que $K$ et $\Frac A$ sont des $A$-algèbres ($f(a/b) =
f(a)f(b)^{-1}$).
 \end{enumerate}
\end{proof}

\begin{example}{Exemples}
 $\Frac \Z = \Q$.\\
Si $K$ est un corps (ou un anneau intègre), alors $K[X]$ est également un
anneau intègre. On définit $K(X) := \Frac K[X]$ l'ensemble des fractions
rationnelles à coefficients dans $K$. 
\end{example}


\subsection{Anneaux euclidiens, principaux, et factoriels}
\vspace{0.5em}

\subsubsection{Divisibilité}
\vspace{0.5em}

\begin{defi}[Divisibilité]
\index{divisibilité}

 Soient $x,y \in A\-\{0\}$. On dit que $x$ divise (ou est un
diviseur de) $y$ s'il exsite $z\in A\-\{0\}$ tel que $y=xz$. On
note alors $x\mid y$.
 \begin{displaymath}
  x \mid y\ \Longleftrightarrow\ (y) \subset (x)
 \end{displaymath}
\end{defi}

\begin{example}[Remarque]
 La relation de divisibilité est un préordre (réflexivité et transitivité).
\end{example}

\begin{prop}
 
Soient $x,y \in A\-\{0\}$. Alors pour tout $a\in
A\-\{0\}$ : 
\begin{displaymath}
 ax \mid ay\ \Longleftrightarrow\ x \mid y
\end{displaymath}
\end{prop}

\begin{proof} $(\Rightarrow)$ est évident.

$(\Leftarrow)$ : $\exists t \in A\-\{0\} ay = tax$ donc $a(y-tx) =
0$ or $A$ est intègre et $a \neq 0$ donc $y = tx$.
\end{proof}

\begin{defi}

On définit la relation d'équivalence $\sim$ par :
\begin{displaymath}
 x\sim y\ \Longleftrightarrow\ x\mid y \text{ et } y\mid x
\end{displaymath}
\end{defi}

\begin{prop}

\begin{displaymath}
 x\sim y\ \Longleftrightarrow\ (x) = (y)\ \Longleftrightarrow\ \exists u\in
A\x,\ x = uy
\end{displaymath}

En particulier, comme pour tout $x \in A\-\{0\},\ 1\mid x$, on a
: \begin{displaymath}
 x \mid 1\ \Longleftrightarrow\ x \sim 1 \ \Longleftrightarrow\ x \in
A\x
\end{displaymath} 
\end{prop}

\begin{defi}[Irréductibilité]
  
Un élément $p \in A\-\{0\}$ est dit irréductible si :
\begin{enumerate}
 \item $p\notin A\x$,
 \item $\forall x,y\in A\-\{0\},\ (p = xy)\ \Longrightarrow\ (x\in
A\x$ ou $y\in A\x)$.
\end{enumerate}
\end{defi}

\begin{example}[Example]
 
$\Z$, $\Z\x = \{-1,1\}$, les irréductibles sont les premiers $\Prem \cup
-\Prem$.
\end{example}

\subsubsection{Anneaux Noethériens}
Commentaire : Noether est une femme.
\vspace{0.5em}

\begin{defiprop}[Anneau Noethérien]
\index{anneau!Noethérien}

Soit $A$ un anneau commutatif. Les assertions suivantes sont équivalentes :
\begin{enumerate}
 \item Toute suite croissante d'idéaux est stationnaire.
 \item Tout idéal de $A$ est de type fini (ie : engendré par une partie finie).
\end{enumerate}

Dans ce cas on dira que $A$ est Noethérien
\end{defiprop}

\begin{proof}\

$(1\Rightarrow2)$ : Soit $I$ un idéal de $A$.
\begin{itemize}
 \item $I_1 := (x_1)$ avec $x_1\in I$, si $I = I_1$ on a fini, sinon :
 \item $I_2 := (x_1,x_2)$ avec $x_2 \in I\- I_1$, si $I = I_1$ on a
fini, sinon :
 \item $\cdots$
\end{itemize}
On cosntruit ainsi une suite croissante d'idéaux $I_1\subsetneq I_2 \subsetneq
\cdots \subsetneq I_n \subset I$. Donc $(I_n)_n$ est stationnaire ainsi il
existe $N\in\N,\ \forall n \geq N,\ I_n = I_N$. On vérifie aisément que $I_N =
(x_1,\cdots,x_N) = I$.

$(2\Rightarrow1)$ : Soit $(I_n)_{n\in\N}$ une suite croissante d'idéaux. On
pose $I := \cup_{n\in\N} I_n$ qui est donc un idéal de type fini par hypothèse.
Donc $I = (x_1,\cdots,x_k)$ et $\forall i \in[1,k],\ \exists j_i \in\N,\ x_i
\in I_{j_i}$. On montre ainsi que la suite est stationnaire à partir du rang $N
:= \Sup \{j_i\tq i\in[1,k]\}$.
\end{proof}

\begin{defi}[Anneau principal (rappel)]

Un anneau est dit principal s'il est intègre et que tous ses idéaux sont
principaux (ie : engendré par un seul élément). En particulier un anneau
principal est Noethérien.
\end{defi}

\begin{defi}[Elément décomposable]\index{element decomposable@élément
décomposable}

Un élément d'un anneau est dit décomposable s'il est associé à un produit fini
d'irréductible.
\end{defi}


\begin{defi}[Anneau factoriel]\index{anneau!factoriel}
 
Soit $A$ un anneau intègre. On dit que $A$ est factoriel si :
\begin{description}
 \item[F1] Tout élément non nul de $A$ est décomposable.
 \item[F2] $\forall m,n\in \N,\ \forall p_1, \cdots, p_n$ et $\forall
q_1,\ \cdots q_m$ irréductibles de $A$ tels que les produits des $p_i$ et des
$q_j$ sont associés. Alors $m=n$ et il existe $\sigma$ bijective telle
que : $\forall i,\ p_i\sim q_{\sigma(j)}$.
\end{description}
\end{defi}

\begin{defiprop}[Valuation p-adique]
\index{valuation p-adique}
 
Soit $A$ un anneau factoriel. 
\begin{enumerate}
 \item Soit $p$ un irréductible de $A$ et $a\in A\-\{0\}$, alors il
existe un unique entier $n$ tel que $a=\alpha p^n$ avec $\alpha \in
A\-(p)$. On l'appelle la valuation p-adique de $a$ et on le note
$\nu_p(a)$ (on pose $\nu_p(0) := -\infty$).
 \item Soit $p$ irréductible, $a,b \in A$, alors $\nu_p(ab) = \nu_p(a) +
\nu_p(b)$.
 \item Soit $\mathcal{P}$ un système de représentants d'irréductibles de $A$
(ie : tout irréductible de $A$ est associé à un élément de $\mathcal{P}$).\\
Alors pour tout $a \in A\-\{0\}$, la famille
$(\nu_p(a))_{p\in\mathcal{P}}$ est presque nulle (ie : sauf sur un nombre fini
d'éléments). De plus $a$ s'écrit :
\begin{displaymath}
 a = u\prod_{p\in\mathcal{P}}p^{\nu_p(a)} \text{ (pour } u\in A\x\text{ )}
\end{displaymath}
\end{enumerate}
\end{defiprop}

\begin{proof}\
 \begin{enumerate}
  \item Supposons que $a = \alpha p^n = \beta p^m$ avec $\alpha,\beta\in
A\-(p)$ et $n < m$. Alors :
\begin{displaymath}
 \alpha p^n - \beta p^m = 0 \Longleftrightarrow (\alpha + \beta p^{m-n})
\Longleftrightarrow \alpha \in (p) \text{ par intégrité de } A
\end{displaymath}
\item Soient $a=\alpha p^n$ et $b = \beta p^m$ avec $\alpha,\beta\in
A\-(p)$. Alors $ab = \alpha \beta p^{m+n}$ et on a toujours $\alpha
\beta \in A \- (p)$.
\item Soit $a$ un élément non nul de l'anneau factoriel $A$. On sait que $a$
est associé à un produit fini d'iiréductibles de $A$. Ainsi il existe
$p_1,\cdots,p_r$ irréductibles et $u$ inversible tels que : 
\begin{displaymath}
 a = u\prod_{i=1}^r p_i^{n_i} \quad \text{ on pose alors } x_r =
u\prod_{i=1}^{r-1} p_i^{n_i} \quad \text{ (ie : } a = x p_r^{n_r}\text{)}
\end{displaymath}
Comme $\nu_{p_r}(p_r^{n_r}) = p^{n_r}$ et $\nu_{p_r}(x) = 0$ d'après la
propriété précédante on a $\nu_{p_r}(a) = n_r$. On obtient le résultat souhaité
par récurrence.
 \end{enumerate}
\end{proof}

\begin{defiprop}[Elément premier]\index{premier!élément}
 
Soit $A$ un anneau intègre et $p$ un élément non nul. On dit que $p$ est
premier si l'idéal engendré par $p$ est premier (ie : $A/(p)$ est intègre).

Cette définition est équivalente aux deux conditions suivantes :
\begin{enumerate}
 \item $p$ n'est pas inversible.
 \item Pour tous éléments $a,b \in A$, si $p$ divise le produit alors $p$
divise $a$ ou $b$ :
\begin{displaymath}
 p\mid ab \Longrightarrow p\mid a \text{ ou } p\mid b
\end{displaymath}
\end{enumerate}
\end{defiprop}

\begin{prop}
 
Dans un anneau $A$ intègre, tout élément premier est irréductible.
\end{prop}

\begin{proof}
Soit $p$ premier. En particulier $(p)\neq A$ donc $p\notin A\x$.

Supposons $p=xy$. Par la caractérisation des idéeaux premiers, on en déduit que
$x\in (p)$ ou $y\in (p)$. Par exemple $x\in(p)$ donc $p\mid x$ et $x\mid
p$ d'où $x = up$ avec $u \in A\x$. Donc $y$ inversible.
\end{proof}

\begin{prop}[Définition équivalente d'un anneau factoriel]
\index{anneau!factoriel}
\begin{description}
 \item [F'1] Toute suite croissante d'idéaux principaux est stationnaire.
 \item [F'2] Tout élément irréductible est premier.
\end{description}
\end{prop}

\begin{proof}\
 \begin{itemize}
  \item [(F'1 $\Rightarrow$ F1)] Soit $p_0$ un élément non décomposable (ie :
n'est pas associé à un produits fini d'éléments irréductibles). $p_0 = xy$ avec
$x$ et $y$ non inversibles et $x$ ou $y$ non décomposable. Supposons que $x$
n'est pas décomposable alors :    
$I_0 := (p_0) \subsetneq (x) =: I_1 \subsetneq \cdots$ (récurrence)

  \item [(F'2 $\Rightarrow$ F2)] Soit $\Pi_{i=1}^n p_i \sim \Pi_{j=1}^m q_j$
deux produits d'irréductibles associé. Par hypothèse tous les irréductibles
sont premiers. On va montrer par récurrence sur $n$ que $n=m$ et $p_i \sim
q_{\sigma(j)}$ :
\begin{itemize}
 \item [($n=0)$] Tous les $q_j$ sont inversibles donc $m = 0$.
 \item [(hérédité)] On suppose la propriété vraie pour $k = n-1 \geq 0$. Si 
 $\Pi_{i=1}^n p_i \sim \Pi_{j=1}^m q_j$, comme $p_n$ est premier il divise
l'un des $q_j$, par exemple $q_m$ et $p_n\mid q_m \Rightarrow p_n\sim q_m$.
On a encore $\Pi_{i=1}^{n-1} p_i \sim \Pi_{j=1}^{m-1} q_j$ et par hypothèse
de récurrence $m-1 = n-1$ et il existe $\sigma$ tel que $\forall (1\leq i\leq
n-1),\ p_i \sim q_{\sigma (j)}$. \\ Il suffit alors de poser $\sigma(n) = m$
pour obtenir le résultat souhaité.
\end{itemize}
\item[(Factoriel $\Rightarrow$ F'1)] Soit $\mathcal{P}$ un système de
représentants d'irréductibles de $A$ et $a, b$ deux éléments non nuls de $A$.
On a toujours : $b\mid a \Rightarrow \forall p \in \mathcal{P},\ \nu_p(b)
\leq \nu_p(a)$ avec les $\nu_p(a)$ presque tous nuls.\\ Donc en particulier il
exsite un nombre fini de diviseurs de $a$, pour tout $a$ non nul.\\ Ainsi $a$
appartient à un nombre fini d'idéaux principaux et toute suite croissante
d'idéaux principaux contenant $a$ est stationnaire.
\item[(Factoriel $\Rightarrow$ F'2)] Soit $p$ un irréductible et $x,y \in
A\-{0}$. Alors si $p\nmid x$ et $p\nmid y$ on a $\nu_p(x) = \nu_p(y) = 0$ et
donc $\nu_p(xy) = 0$ ainsi $p\nmid xy$.\\
On a montré par contraposition la caractérisation d'un éléments premier.
\end{itemize}
\end{proof}

\begin{coro}\index{anneau!principal}\index{anneau!factoriel}
 
Tout anneau principal est factoriel.
\end{coro}

\begin{proof}\
\begin{itemize}
 \item [(F'1)] Principal $\Rightarrow$ Noethérien $\Rightarrow$ F'1.
 \item [(F'2)] Soient $p$ un irréductible et $x,y\in A$ tels que $p\mid xy$. On
a $(p) \neq A$.\\ On pose $I := \{u\in A \tq p \mid uy\}$ et on vérifie que
c'est un idéal de $A$ donc il est principal par hypothèse et il existe $a\in A$
tel que $I = (a)$. Mais $p\in I$ donc $p\mid a$ or $p$ irréductible. Deux cas
sont possibles :
\begin{itemize}
 \item [($a\sim 1$)] $I=A$ et $p\mid y$.
 \item [($a\sim p$)] $I=(p)$ or $x\in I$ donc $p\mid x$.
\end{itemize}
\end{itemize}
\end{proof}

\begin{coro}\index{anneau!principal}
 
Soit $A$ un anneau principal.
\begin{enumerate}
 \item Les idéaux de $A$ sont l'idéal nul et ceux engendrés par les
irréductibles de $A$.
 \item Si $A$ n'est pas un corps, les idéaux maximaux de $A$ sont ceux
engendrés par les éléments irréductibles.
\end{enumerate}
\end{coro}

\subsubsection{Pgcd, ppcm, élements premiers entre eux}
\vspace{0.5em}

\begin{defi}[Eléments premiers entre eux]
\index{premier!éléments premiers entre eux}

Soient $x_1,\cdots,x_n \in A$. On dit que les $x_i$ sont premiers entre eux si
$A$ est le plus petit idéal qui les contient tous.
\end{defi}

\begin{defi}[pgcd - ppcm]\index{pgcd}\index{ppcm}

On appelle $\pgcd$ de $x,y\in A$ tout élément $c \in A$ tel que $c$ soit le
plus petit idéal principal contenant $(x)+(y) = (x,y)$. On note $c =
\pgcd(x,y)$ ou $c=x\vee y$.

On appelle $\ppcm$ de $x,y\in A$ tout élément $d\in A$ tel que $(d) =
(x)\cap(y)$. On note $d=\ppcm(x,y)$ ou $d = x\wedge y$.
\end{defi}

\begin{example}[Remarques]
Il n'y a pas toujours existence, mais dans ce cas il y a unicité à association
près.

On a : $1=x\vee y \Leftrightarrow x$ et $y$ sont premiers entre eux.
\end{example}

\begin{theo}[Bézout]\index{Bézout, théorème de}
 
Soient $A$ un anneau principal et $x,y\in A$. Alors il existe $u,v \in A$ tels
que : $ux + vy = x\vee y$.

En particulier si $x$ et $y$ sont premiers entre eux il existe $u,v \in A$ tels
que : $ux + vy = 1$.
\end{theo}

\begin{proof}
 
$(x)+(y) = (x,y)$ est principal donc engendré par l'élément $(x\vee y)$. \\  
Par définition $x\vee y \in (x,y) \Rightarrow \exists u,v \in A,\ ux + vy =
x\vee y$.
\end{proof}


\begin{prop}\index{anneau!factoriel}

Soit $A$ un anneau factoriel.
\begin{enumerate}
 \item $\pgcd$ et $\ppcm$ existent toujours.
 \item Soient $x,y\in A\-\{0\}$ et $\delta = x\vee y$ alors :\\
       $x = \delta x_0,\ y = \delta y_0$ avec $x_0\vee y_0 = 1$. De plus
$x\wedge y = \delta x_0 y_0$.
 \item Soient $x,y,z\in A \-\{0\}$ tels que $x\mid yz$ et $x\wedge z = 1$ alors
$x\mid y$.
\end{enumerate}
\end{prop}

\begin{lemm}[Lemme chinois]
\index{lemme chinois}
 
Soit $A$ un anneau et soient $I,J$ deux idéaux de $A$ tels que $I+J = A$. Alors
: \begin{displaymath}
   A/(I\cap J) \simeq A/I \times A/J
  \end{displaymath}
En particulier si $A$ est principal, $p,q \in A$ premiers entre eux, alors :
\begin{displaymath}
 A/(pq) \simeq A/(p) \times A/(q)
\end{displaymath}
\end{lemm}

\begin{proof} On considère les projections canoniques $\pi : A \rightarrow A/I$
et $\pi' : A \rightarrow A/J$, et on pose :
\begin{displaymath} \begin{array}{rrcl}
 \varphi : &A &\longrightarrow & A/I \times A/J \\
           &x &\longmapsto     & (\pi(x),\pi'(x))
\end{array} \end{displaymath}

En remarquant que $\Ker \varphi = I\cap J$, on applique le premier théorème
d'isomorphisme :
\begin{displaymath}
 A/(I\cap J) \simeq \Ima \varphi
\end{displaymath}

Il reste à montrer que $\Ima \varphi = A/I\times A/J$. Or $I+J = A$ donc il
existe $u\in I$ et $v\in J$ tels que $u+v = 1$.
Soit alors $x, y\in A$ et $z=ux+vy$ on a $ux\in I$ et $vy\in J$
\begin{displaymath}u+v=1 \Rightarrow 
\left\lbrace\begin{array}{rcl}
u \equiv 1 (J) &\Rightarrow& \pi(z) = \pi(vy) = \pi(y) \\
v \equiv 1 (I) &\Rightarrow& \pi'(z) = \pi'(ux) = \pi'(x)\\
\end{array}\right. \end{displaymath}

Ainsi $(\pi(y),\pi(x)) = \varphi(z) \in \Ima \varphi$ d'où $\Ima \varphi \simeq
 A/I\times A/J$.
\end{proof}

\begin{defi}[Anneau euclidien]\index{anneau!euclidien}
 
On dit qu'un anneau intègre $A$ est euclidien s'il existe une application $N :
A\rightarrow \N$ telle que :
\begin{description}
 \item [E1] $\forall x \in A,\ (N(x) = 0 \Leftrightarrow x=0)$,
 \item [E2] $\forall x \in A,\ \forall y \in A\-\{0\},\ \exists q,r\in A,\ x =
qy+r$ avec $N(r) < N(y)$.
\end{description}
\end{defi}

\begin{example}[Exemples]\
 \begin{enumerate}
  \item Soit $\Z[i] = \{a +ib \tq a,b\in\Z\}$ et $N : (a+ib) \mapsto
|a+ib|^2=a^2+b^2$.
\begin{itemize}
 \item [(E1)] Assez clair,
 \item [(E2)] Soit $\xi = x+iy \in \C$, alors il existe $z = a+ib\in \Z[i]$ tel
que $N(\xi-z) < \frac{1}{2}$. En effet, il suffit de prendre
$a=E(x-\frac{1}{2})$ et $b=E(y-\frac{1}{2})$.\\
Soient alors $x,y\in\Z[i]$ avec $y\neq 0$. Soit $\xi = \frac{x}{y}$ et $q$ tel
que $N(\xi-q) < \frac{1}{2}$. On a $x = qy +r$ avec :\\
$N(r) = |x-qy|^2=|y|^2|\xi-q|^2 < \frac{1}{2}|y|^2 < N(y)$.
\end{itemize}
\item Soit $K$ un corps et $N : K[X] \rightarrow \N$ définit comme suit :
\begin{displaymath}
 N(P) = \left\lbrace\begin{array}{ll}
\deg P & \text{ si } P \neq 0\\
0      &\text{ si } P = 0   \end{array}\right.
\end{displaymath}
\begin{itemize}
 \item [(E1)] Assez clair,
 \item [(E2)] On vérifie que la division des polynômes convient.
\end{itemize}
\end{enumerate}
\end{example}

\begin{prop}
 
Soit $A$ un anneau non nul et soient $P,T \in K[X]$ avec $T\neq 0$ et tels que
leurs coefficients directeurs soient inversibles.

Alors il existe un unique couple $(Q,R)\in K[X]\times K[X]$ tel que $P = QT+R$
avec $N(R) < N(T$.
\end{prop}

\begin{theo}
 
Tout anneau eclidien est principal.
\end{theo}

\begin{example}[Remarque] Ainsi $K[X]$ est principal mais en général $A[X]$ ne
l'est pas.
\end{example}

\begin{proof}
 Soit $I$ un idéal non nul de $A$. Il existe $y\in I\-\{0\}$ qui minimise $N$.\\
$\forall x \in I,\ x=qy +r \Rightarrow N(r) < N(y) \Rightarrow r=0$. Ainsi
$x\in (y)$ donc $I = (y)$.
\end{proof}

























































