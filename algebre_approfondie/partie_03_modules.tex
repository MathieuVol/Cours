\section{Modules}
\vspace{0.5em}
\hfill \boxed{\boxed{\textbf{Dans toute cette partie les anneaux sont
commutatifs.}}} \hfill
\vspace{0.5em}

\subsection{Définitions}
\vspace{0.5em}

\subsubsection{Modules}
\vspace{0.5em}

\begin{defi}[Module]\index{module}
 
 Soit $A$ un anneau. Un ensemble $M$ est un $A$-module s'il est muni d'une loi
de composition interne ``$\,+$'' et d'une loi externe ``$\:\cdotp$'' telles que
:
\begin{description}
 \item [M1\:] $(M,+)$ est un groupe abélien,
 \item [M2\:] $\forall a\in A,\ \forall x,y \in M,\ a(x+y) = (ax)+(ay)$, 
 \item [M2']  $\forall x\in M,\ \forall a,b\in A,\ (a+b)x = (ax)+(bx)$,
 \item [M3\:] $\forall x\in M,\ \forall a,b\in A,\ (ab)x = a(bx)$,
 \item [M3']   $\forall x \in M,\ 1_A x= x$.
\end{description}
\end{defi}

\begin{example}[Attention]\

 Ne pas confondre les lois additive de l'anneau et du module.\\ Ne pas confondre
la loi multiplication de la l'anneau et la loi externe.\\ Ne pas inventer de loi
multiplicative interne dans le module.
\end{example}

\begin{example}[Remarque]
 Si $K$ est un corps, tout $K$-module est un $K$-espace vectoriel.
\end{example}

\begin{example}[Autre point de vue]
 On remarque que comme $(M,+)$ est un groupe abélien, l'ensemble $\End M$ est
un anneau (non commutatif en général). On montre alors que la donnée d'une loi
externe de module revient à se donner un morphisme d'anneau $\rho : A\rightarrow
\End M$.\\
En effet, si $M$ est un module, on définit $\rho(a)(x) = ax$ qui est bien un
morphisme d'anneau.\\
Réciproquement, si $\rho : A \rightarrow \End M$ est un morphisme d'anneau
alors on vérifie que $M$ est bien un module.
\end{example}

\begin{example}[Exemples]\
\begin{enumerate}
 \item Un $\Z$-module est simplement un groupe abélien, et tout groupe abélien
peut être vu comme un $\Z$-module.
 \item $A$ est un $A$-module.
 \item $(0)$ est un $A$-module.
 \item Tout idéal d'un anneau $A$ est un $A$-module.
 \item Si $B$ est une $A$-algèbre, et $M$ un $A$-module, alors en considérant
les morphismes :\\
$\eta : A \rightarrow B$ et $\rho : B \rightarrow \End M$ on construit $\rho'
=\rho\circ \eta : A\rightarrow \End M$ qui est aussi un morphisme.\\ Donc $M$
est un $A$-module et en particulier, $B$ et ses idéaux sont des $A$-modules.
\item Si $I$ est un ensemble quelconque on note $A^I$ l'ensemble des fonctions
de $I$ dans $A$ que l'on appelle aussi suites indexées par $I$. On a déjà vu
que $A^I$ est une $A$-algèbre.\\
On définit également $A^{(I)}$ l'ensemble des suites presque nulles. On vérifie
aisémént que $A^{(I)}$ est un idéal bilatère de $A^I$. Donc $A^I$ et $A^{(I)}$
sont des $A$-modules.\\
Pour $i\in I$ on définit $e_i\in A^{(I)}$ tel que $(e_i)j = \delta_{ij}$. Ainsi
pour tout $a = (a_i)_{i\in I} \in A^{(I)}$ on a :
\begin{displaymath}
 a = \sum_{i\in I} a_ie_i \text{ et cette somme est finie. }
\end{displaymath}
On appellera $(e_i)_{i\in I}$ la base canonique.
\end{enumerate}
\end{example}

\subsection{Sous-modules et applications linéaires}
\vspace{0.5em}

\begin{defi}[Sous-module]\index{module!sous-module}
 
 Soient $M$ un $A$-module et $N \subset M$. On dit que $N$ est un sous-module
de $M$ si :
\begin{enumerate}
 \item $N$ est un sous-groupe de $(M,+)$,
 \item $N$ est stable pour la loi externe $A\times N \rightarrow N\subset M$.
\end{enumerate}
\end{defi}
\begin{example}[Remarque]
  Cette notion généralise celle d'idéal. En effet $A$ est un $A$-module dont les
sous-modules sont ses idéaux.
\end{example}

\begin{defiprop}[Sous-module engendré]\index{module!sous-module!sous-module
engendré}
 
 Toute intersection de sous-modules est un sous-module. En particulier, pour
toute partie $X\subset M$ il existe un plus petit sous-module de $M$ qui
contient $X$.

On le définit comme le sous-module engendré par $X$ et on le note $<X>$.
\end{defiprop}

\begin{defi}[Combinaison linéaire]\index{combinaison linéaire}
 
 On appelle combinaison linéaire d'éléments de $X\subset M$ à coefficients dans
$A$ tout élément $x$ de la forme :
\begin{displaymath}
 x = \sum_{i\in F} a_ix_i \text{ où } F \text{ est fini, et pour tout }
i\in F \text{ et } a_i\in A,\ x_i\in X
\end{displaymath}
\end{defi}

\begin{prop}[Description du sous-module engendré]
 
 $<X>$ est l'ensemble des combinaisons linéaires d'éléments de $X$.
\end{prop}

\begin{defi}[Module de type fini et cyclique]\index{module!de type fini}
 \index{module!cyclique}
 
 On dit qu'une partie $X$ d'un module $M$ l'engendre si $<X>\ = M$.
 \begin{enumerate}
  \item On dit qu'un module est de type fini s'il est engendré par une partie
finie,
\item On dit qu'il est cyclique s'il est engendré par un seul élément.
 \end{enumerate}
\end{defi}

\begin{defi}[Application linéaire]\index{application linéaire}
\index{module!morphisme|see{application linéaire}}

Soient $M$ et $N$ deux $A$-modules. Un application $f : M \rightarrow N$ est
dite $A$-linéaire si :
\begin{enumerate}
 \item $f$ est un morphisme de groupe $(M,+)\rightarrow(N,+)$,
 \item $\forall a \in A,\ \forall x \in M,\ f(ax) = af(x)$.
\end{enumerate}

On dit aussi que $f$ est alors un morphisme de $A$-modules.
\end{defi}

\begin{example}[Remarque]
 \begin{enumerate}
  \item La composée de deux applications $A$-linéaires est encore une
application $A$-linéaire.
\item Si $f$ est linéaire bijective alors $f^{-1}$ est aussi linéaire. On dit
alors que $f$ est un ismorphisme de $A$-modules.
\item Soient $M$ un module et $M'$ un sous-module. Soit $i : M'\hookrightarrow
M$. Alors il existe une unique structure de $A$-module sur $M'$ telle que $i$
soit linéaire.
 \end{enumerate}
\end{example}

\begin{prop}[Noyau et image d'une application linéaire]
\index{application linéaire!noyau}
\index{application linéaire!image}
 
 Soit $f : M \rightarrow N$ une application linéaire. Alors :
 \begin{enumerate}
  \item $\Ker f$ est un sous-module de $M$,
  \item $\Ima f$ est un sous-module de $N$.
 \end{enumerate}
\end{prop}

\begin{proof}
 Ce sont des sous-groupes car $f$ est en particulier un morphisme de groupe.\\
 De plus, si $x\in \Ker f$ et $a\in A$ alors $f(ax)=f(a)f(x)=0$ donc $ax\in
\Ker f$.\\ De même si $y=f(x) \in\Ima f$ alors $ay = af(x) = f(ax)\in\Ima f$.
\end{proof}

\subsection{Produit et somme directe de modules}
\vspace{0.5em}

\begin{defi}[Module produit]\index{module!produit}
 
 Soient $A$ un anneau, $I$ un ensemble, et $(M_i)_{i\in i}$ une famille de
$A$-modules indexée par $I$. On définit le produit $\prod_{i\in I}M_i$
(éventuellement via l'axiome du choix) que l'on muni des lois suivantes :
\begin{enumerate}
 \item Pour $X = (x_i)_{i\in I},\ Y = (y_i)_{i\in I},\ \in \prod_{i\in I}M_i$
on pose $X+Y = (x_i + y_i)_{i\in I} \in \prod_{i\in I}M_i$.
\item Et pour $a\in a$ on pose $aX = (ax_i)_{i\in I} \in \prod_{i\in I}M_i$.
\end{enumerate}

On appelle ce $A$-module le $A$-module produit des $M_i$.
\end{defi}

\begin{defi}[Projection canonique]
\index{module!produit!projection canonique}

Pour chaque $j\in I$ on définit l'application linéaire $j^{\text{ème}}$
projection canonique :
\begin{displaymath} \begin{array}{rrcl}
p_j : &\Prod_{i\in I}M_i &\longrightarrow& M_j \\
      &(x_i)_{i\in I} & \longmapsto & x_j
\end{array} \end{displaymath}
\end{defi}


\begin{prop}
 
 Soit $M'$ un module. On se donne pour tout $i\in I$ une application linéaire
$f_i : M' \rightarrow M_i$.

Alors il existe une unique application linéaire $f : M' \rightarrow \prod_{i\in
I}M_i$ telle que :
\begin{displaymath}
 f_i = p_i \circ f \text{ pour tout }i
\end{displaymath}
\end{prop}

\begin{proof} 
 On prend $f(x) = (f_i(x))_{i\in I}$ qui est linéaire. Il reste à
vérifier l'unicité.
\end{proof}

\begin{example}[Remarque]
 $A^I = \prod_{i\in I} A_i$ en prenant $A_i = A$ pour chaque $i$.
\end{example}

\begin{defi}[Somme directe]\index{module!some directe}
 
 Le sous ensemble du produit constitué des familles presques nulles est un
sous-module appelé somme directe des $M_i$ et noté $\bigoplus_{i\in I} M_i$.
\end{defi}

\begin{defi}[Injection canonique]
 \index{module!somme directe!injection canonique}

Pour chaque $j\in I$ on définit l'application linéaire
$j^{\text{ème}}$ injection canonique :
\begin{displaymath} \begin{array}{rrcl}
i_j : &M_j &\longrightarrow& \Bigoplus_{i\in I} M_i \\
      & x  &\longmapsto    & (y_i)_{i\in I} \text{ avec }y_j = x \text{ et }
y_i = 0 \text{ pour } i\neq j
\end{array} \end{displaymath}
\end{defi}

\begin{prop}
 
 Soit $M'$ un module. On se donne pour tout $i\in I$ une application linéaire
$f_i : M_i \rightarrow M'$.

Alors il existe une unique application linéaire $f : \bigoplus_{i\in I} M_i
\rightarrow M'$ telle que :
\begin{displaymath}
 f\circ i_j = f_j \text{ pour tout }j
\end{displaymath}
\end{prop}

\begin{proof}
 Soit $(x_i)_{i\in I} \in \bigoplus_{i\in I}$ alors $f((x_i)_{i\in I}) =
\sum_{i\in I} f_i (x_i)$.
\end{proof}

\begin{example}[Remarque]
 $A^{(I)} = \bigoplus_{i\in I} A_i$ en prenant $A_i = A$ pour chaque $i$.
\end{example}


\begin{prop}[Caractérisation de la somme directe]
 \index{module!somme directe}

Soient $M$ un module et $I$ un ensemble. Soit $(M_i)_{i\in I}$ une famille de
sous-modules telle que :
\begin{enumerate}
 \item $\sum_{i\in I} M_i = M$ en tant que sous-module engendré par l'union,
 \item $\forall j\in I,\ M_j\cap \sum_{i\neq j} M_i = \{0\}$.
 
 Alors $M$ est isomorphe à la somme directe $\bigoplus_{i\in I} M_i$.
 On dit dans ce cas que $M$ est somme directe des sous-modules $M_i$ et on note
: \begin{displaymath}
   M=\Bigoplus_{i\in I} M_i
  \end{displaymath}
\end{enumerate}
\end{prop}

\begin{defi}[Supplémentaire]\index{module!supplémentaire}
 
 Soit $M$ un module et soient $M',M''$ deux sous-modules. On dit que $M'$ et
$M''$ sont supplémentaires si $M=M'\oplus M''$.
\end{defi}

\subsection{Module quotient}
\vspace{0.5em}

\begin{defiprop}[Module quotient]
\index{module!quotient}\index{module!sous-module}
 
 Soient $M$ un module et $M'$ un sous-groupe de $(M,+)$. $M'$ est un
sous-module de $M$ si et seulement si il existe sur le sous-groupe quotient
$M/M'$ une structure de module telle que la projection canonique $\pi : M
\rightarrow M/M'$ soit linéaire.

Dans ce cas, cette structure est unique et $M/M'$ est alors le module quotient.
\end{defiprop}

\begin{proof} \
\begin{description}\item[(rappel)] On considère la congruence modulo $M'$ telle
que $x\equiv y \Leftrightarrow (x-y)\in M'$.

$M/M'$ sont les classes d'équivalence, et $\pi : x\mapsto \overline{x}$ associe
la classe d'un élément. On sait qu'il existe sur $M/M'$ une unique loi de
composition telle que : $M/M'$ soit un groupe abélien, $\pi$ soit un morphisme
de groupe et $\Ker \pi = M'$.

\item[(sous-module)] On suppose l'existe de la
projection canonique linéaire sur le module $M/M'$. Alors soient $a\in A,\ x\in
M'$ on a $\pi(ax) = a\pi(x) = 0$ donc $ax\in M'$ et ainsi $M'$ est un
sous-module. 

\item[(module quotient)] Réciproquement, on suppose que $M'$ est un sous-module
de $M$.
\begin{description}
 \item [($\pi$ linéaire)] Soient $\xi \in M/M'$ et $a\in A$. Il existe $x\in M$
tel que $\pi(x) = \xi$.  TODO
 \item [(M1)] $M/M'$ est en particulier un groupe abélien.
 \item [(M2)] TODO ...
\end{description}
\end{description}
\end{proof}

\begin{prop}[Factorisation]

\begin{tabular}{rl}
 Soient &$f : M\rightarrow N\ $ un morphisme de modules \\
  et &         $p : M\rightarrow M'$ un morphisme de modules surjectif.
\end{tabular}

\setlength{\unitlength}{1em}
\begin{picture}(10,7.5)
\put(3.5,5.5){$M$}
\put(5,6){\vector(1,0){3}}
\put(6,6.5){$f$}
\put(8.5,5.5){$N$}
\put(3.5,0.5){$M''$}
\put(5,2){\vector(1,1){3}}
\put(4,5){\vector(0,-1){3}}
\put(3,3.5){$p$}
\put(7.3,3.5){$g$}
\put(11,5.5){Alors les deux assertions suivantes sont équivalentes :}
\put(11,4){1. $\Ker p \subset \Ker f$}
\put(11,2.5){2. Il existe un morphisme de modules $g : M''\rightarrow N$ tel que
$f =g\circ p$}
\end{picture}

Dans ce cas, un tel $g$ est unique.
\end{prop}
\begin{proof} Soit $M' = \ker p$.
\begin{description}
 \item [(2 $\Rightarrow$ 1)] On suppose que $g$ existe. Soit alors $x\in
\ker p$. On a bien $g\circ p (x) = 0 = f(x)$ d'où l'inclusion demandée.
\item [(1 $\Rightarrow$ 2)] On suppose que $\Ker p \subset \Ker f$. 
\begin{itemize}
 \item [(définition de $g$)] 
 Soit $y \in M''$ alors il existe $x\in M$ tel que $p(x) = y$ par surjectivité
de $p$. On pose $g(y) = f(x)$ et on vérifie que pour tout $x'\in M''$ tel que
$p(x) = p(x')$ on a bien $f(x) = f(x')$. En effet $p(x-x')=0$ donc $f(x-x')=0$.
 \item [(linéarité)] Avec les mêmes notations $g(ay) = f(ax) = af(x)=ag(y)$, et
si $p(x) = y$ alors $p(ax) = ap(x) = ay$.
 \item [(additivité)] Soit $y'=p(y)$ et $x' = p(x)$. Alors $p(x+y)=x'+y'$ donc
$g(x'+y') = f(x+y) = f(x) + f(y) = g(x') + g(y')$.
 \end{itemize}
\end{description}
\end{proof}

\begin{theo}[Propriété universelle du quotient]

Soient $N$ un $A$-module, $M$ un $A$-module, $M'$ un sous module de $M$ et $\pi
: M\rightarrow M/M'$ la projection canonique. Alors il y a une équivalence
entre :
\begin{enumerate}
 \item La donnée d'un morphisme de $A$-modules $g:M/M'\rightarrow N$.
 \item La donnée d'un morphisme de $A$-modules $f : M\rightarrow N$ tel que
$M'\subset \Ker f$.
\end{enumerate}

De plus, étant donné un tel morphisme $f$ il existe une unique application
$A$-linéaire $\tilde{f} : M/M'\rightarrow N$ tel que $\tilde{f}\circ\pi = f$ et
$\Ima \tilde{f} = \Ima f$ et $\Ker\tilde{f} = \pi(\Ker  f)$.

On dit alors qu'on obtient $\tilde{f}$ par passage au quotient.
\end{theo}

\begin{proof}
 C'est un corollaire de la proposition précédante en prenant $M'' = M/M'$ et $p
= \pi$.
\end{proof}

\begin{prop}[Factorisation canonique d'un morphisme de A-module]
 
Soit $f:M\rightarrow N$ une application linéaire. Alors il existe un unique
isomorphisme de $A$-module $\tilde{f}: M/\Ker f \rightarrow \Ima f$ tel que le
diagramme suivant commute :
\begin{center}
\setlength{\unitlength}{1em}
\begin{picture}(15,7)
\put(3.5,5.5){$M$}
\put(5,6){\vector(1,0){3}}
\put(6,6.5){$f$}
\put(8.5,5.5){$N$}

\put(0,0.5){$M/\Ker f$}
\put(5,0.5){\vector(1,0){3}}
\put(6,1){$\tilde{f}$}
\put(8.5,0.5){$\Ima f$}

\put(4,5){\vector(0,-1){3}}
\put(3,3){$\pi$}

\put(9,2){\vector(0,1){3}}
\put(9.5,3){$i$}
\end{picture}
\end{center}
En particulier on a :
\begin{displaymath}
 M/\Ker f \simeq \Ima f
\end{displaymath}

\end{prop}

\begin{proof}
 La propriété universelle du quotient, en prenant $M'=\Ker f$, nous assure
l'existence d'un unique $\tilde{f} : M/\Ker f \rightarrow N$ tel que
$\tilde{f}\circ \pi = f$. De plus on a $\Ima\tilde{f} = \Ima f$ et
$\Ker\tilde{f} = \pi(\Ker f) = \{0\}$. Donc $\tilde{f}$ est un isomorphisme.
\end{proof}

\subsection{Suites exactes}
\vspace{0.5em}

\begin{defi}
 
 Soient $M,M',M''$ des $A$-modules, $f : M' \rightarrow M$ et $g : M\rightarrow
M''$ deux applications $A$-linéaires. On dit que la suite suivante est exacte
si $\ker g = \Ima f$ :
\begin{displaymath}\begin{array}{ccccc}
   & f             &   & g              &    \\
 M'&\longrightarrow& M &\longrightarrow &M''
 \end{array} \end{displaymath}

Plus généralement, si $I$ est un intervalle de $\Z$ et pour tout $i\in I,\ M_i$
est un $A$-module et $f_i : M_i \rightarrow M_{i+1}$ est une application
linéaire, on dit que la suite suivante est exacte si $\Ker f_{i+1} = \Ima
f_i$ :
\begin{displaymath}\begin{array}{ccccccc}
                &       & f_{i-1}       &     & f_i            &       & \\
 \dashrightarrow&M_{i-1}&\longrightarrow& M_i &\longrightarrow &M_{i+1}&
\dashrightarrow
 \end{array} \end{displaymath}
\end{defi}

\begin{example}[Exemples] \
 
 \begin{enumerate}
  \item Les suites suivantes sont exactes si et seulement si $f$ est surjective,
$g$ est injective, et $h$ est bijective :
  \vspace{-0.5em}\begin{displaymath}\begin{array}{ccccc}
    &               &   & f              &    \\
 (0)&\longrightarrow& M &\longrightarrow &N
 \end{array} \end{displaymath}
   \begin{displaymath}\begin{array}{ccccc}
    & g              &   &               &    \\
 M&\longrightarrow& N &\longrightarrow &(0)
 \end{array} \end{displaymath}
   \begin{displaymath}\begin{array}{ccccccc}
    &               &   & h              & &               & \\
 (0)&\longrightarrow& M &\longrightarrow &N&\longrightarrow&(0)
 \end{array} \end{displaymath} \vspace{0.5em}
 
  \item La suite suivante est une suite exacte courte si $g$ est surjective,
$f$ injective et $\Ker g = \Ima f$ : 
\vspace{-0.5em}\begin{displaymath}\begin{array}{ccccccccc}
    &               &   & f          &               & g &   & &\\
 (0)&\longrightarrow&M'&\longrightarrow& M &\longrightarrow
&M''& \longrightarrow &(0)
 \end{array} \end{displaymath} \vspace{0.5em}
 
  \item Soient $M_1$ et $M_2$ deux $A$-modules. $i_1 : M_1 \rightarrow
M_1\oplus M_2$ l'injection canonique et $p_2 : M_1\oplus M_2 \rightarrow M_2$
la projection canonique. Alors on a la suite exacte courte :
\vspace{-0.5em}\begin{displaymath}\begin{array}{ccccccccc}
    &               &   & i_1           &               & p_2                  
&   & &\\
 (0)&\longrightarrow&M_1&\longrightarrow& M_1\oplus M_2 &\longrightarrow
&M_2& \longrightarrow &(0)
 \end{array} \end{displaymath}
 \end{enumerate}
\end{example}

\subsection{Modules de type fini}
\vspace{0.5em}

\begin{prop}[Caractérisation des modules de type fini]
 Un $A$-module est de type fini si et seulement si il existe $n\in\N$ et une
application linéaire $\sigma : A^n \rightarrow M$ surjective.
\end{prop}

\begin{proof}\
\begin{description}
 \item [($\Rightarrow$)] Soit $M$ un $A$-module de type fini. Alors il est
engendré par une famille finie $\{x_1,\cdots,x_n\} \subset M$. Ainsi $M =
\left\lbrace\sum_{i=1}^n a_ix_i \tq a_i\in A\right\rbrace$, il suffit donc de
prendre $\sigma (a_1,\cdots,a_n) = \sum_{i=1}^n a_ix_i$.
\item [($\Leftarrow$)] En considérant la famille des $x_i =
(0,\cdots,0,1,0,\cdots,0)$ avec un $1$ en $i^\text{ème}$ position, par
linéarité et surjectivité de $\sigma$ on obtient que $M =
\left\lbrace\sum_{i=1}^n a_ix_i \tq a_i\in A\right\rbrace$ est de type fini.
\end{description}
\end{proof}

\begin{theo}[Sur les modules de type fini]

\begin{enumerate}
 \item Soient $M$ et $M'$ deux $A$-modules et $f:M\rightarrow M'$ une
application linéaire. Si $M$ est de type fini, alors $M'$ est aussi de type
fini.
 \item Soit la suite exacte :
 \begin{displaymath}\begin{array}{ccccccc}
   & f          &               & g &   & &\\
M'&\longrightarrow& M &\longrightarrow
&M''& \longrightarrow &(0)
 \end{array} \end{displaymath} \vspace{0.5em}
 Si $M'$ et $M''$ sont de type fini alors $M$ est de type fini.
 \item Soient $M$ et $M'$ deux $A$-modules et $f : M'\rightarrow M$ une
application linéaire injective. Si de plus $A$ est N\oe{}thérien et $M$ est de
type fini, alors $M'$ est de type fini.
\end{enumerate}
\end{theo}

\begin{proof}\ 
 \begin{enumerate}
  \item On a $\sigma : A^n\rightarrow M$ et $f : M\rightarrow M'$ avec $\sigma$
et $f$ deux applications linéaires surjectives. On peut donc considérer
l'application linéaire surjective $\sigma'= f\circ \sigma : A^n\rightarrow M'$
donc $M'$ est de type fini.
\item TODO
\item TODO
 \end{enumerate}
\end{proof}

\begin{theo}[Théorème de Hilbert]
 
Si $A$ est un anneau N\oe{}thérien, alors l'anneau $A[X]$ est N\oe{}thérien.
\end{theo}

\begin{proof}
 TODO
\end{proof}

\subsection{Algèbre de type finie}
\begin{center}
-- Dans cette partie $A$ est un anneau et $B$ une $A$-algèbre -- 
\end{center}

\begin{defi}[Algèbre de type finie]
\begin{enumerate}
\item Une partie $X\subset B$ engendre $B$ comme $A$-algèbre si le sous-anneau
de $B$ engendré par $\eta_B(A) \cup X$ est $B$.
\item On dit que $B$ est une $A$-algèbre de type finie si $B$ est engendré
comme $A$-algèbre par une partie finie de $B$.
\end{enumerate}
\end{defi}

\begin{example}[Remarque]
 Soient $x_1,\cdots,x_n \in B$ et $\phi : A[X_1,\cdots,X_n]\rightarrow B$ le
morphisme de $A$-algèbre défini par $\phi(X_i) = x_i$. Alors l'image de $\phi$
est le sous-anneau engendré par $\eta_B(A)\cup\{c_1,\cdots,x_n\}$. Ainsi $B$
est engendré comme $A$-algèbre par les $x_i$ si et seulement si $\phi$ est
surjective.

En particulier pour toute $A$-algèbre $B$ de type finie, il existe $n\in\N$ un
idéal $I$ de $A[X_1,\cdots,x_n]$ tel que :
\begin{displaymath}
 B \simeq A[X_1,\cdots,x_n]
\end{displaymath}
\end{example}

\begin{example}[Rappel] Si $B$ est une $A$-algèbre alors $B$ est naturellement
muni d'une structure de $A$-module.
\end{example}
\begin{example}[Proposition]
Si $B$ est de type fini comme $A$-module, alors c'est une $A$-algèbre de type
fini. Mais il n'y a pas de réciproque.
\end{example}
\begin{example}[Contre exemple]
 $\Q[X]$ est une $\Q$-algèbre de type finie (car engendré par $X$). Mais c'est
un $\Q$-espace vectoriel de dimension infinie (donc un $\Q$-module qui n'est
pas de type fini).
\end{example}

\begin{lemm}
 
 Soit $f : A \rightarrow B$ un morphisme d'anneaux. Alors :
 \begin{enumerate}
  \item L'application qui à un idéal $J$ de $B$ associe l'idéal $f^{-1}(J)$ de
$A$ est injective.
\item Si $A$ est N\oe{}thérien, alors $B$ aussi.
 \end{enumerate}
\end{lemm}

\begin{proof}\
\begin{itemize}
 \item C'est un rappel du chapitre sur les anneaux.
 \item Soit $(J_n)_{n\in\N}$ une suite croissante d'idéaux de $B$. On pose
$(I_n = f^{-1}(J_n))_{n\in\N}$ qui est une suite croissante d'idéaux de $A$
donc stationnaire. Il existe donc $N\in\N$ tel que pour tout $n\geq N$ on a
$I_n=I_N$. Par injectivité de $J\mapsto f^{-1}(J)$ on obtient également que
$J_n = J_N$ ce qui montre que $B$ est N\oe{}thérien.
\end{itemize}
\end{proof}


\begin{prop}[Quotient d'anneau N\oe{}thérien]
 
 Si $A$ est un anneau N\oe{}thérien, alors pour tout idéal $I$ de $A$ le
quotient $A/I$ est N\oe{}thérien.
\end{prop}
\begin{proof}
 C'est une conséquence du lemme en prenant $f = \pi : A \rightarrow A/I$.
\end{proof}


\begin{prop}[Algèbre de type finie et anneau N\oe{}thérien]

 Soit $A$ un anneau N\oe{}thérien et $B$ une $A$-algèbre de type fini. Alors $B$
est un anneau N\oe{}thérien.
\end{prop}

\begin{proof} Le théorème de Hilbert nous assure que $A[X_1,\cdots,X_n]$ est
N\oe{}thérien et que \\$B\simeq A[X_1,\cdots,X_n]/I$ qui est N\oe{}thérien
d'après la propriété précédante.
\end{proof}

\subsection{Modules libres et modules de torsion}
\subsubsection{Modules libres}
\vspace{0.5em}

\begin{defi}[Famille libre, génératrice, et base]
 
 Soient $A$ un anneau, $M$ un $A$-module et $(x_i)_{i\in I}$ une famille
d'éléments de $M$.

Soit $\phi : A^{(I)}\owns e_i \longrightarrow x_i \in M$ une application
linéaire. On rappele que les $e_i$ forment la base canonique de $A^{(I)}$.

\begin{enumerate}
 \item On dit que la famille des $x_i$ est libre si $\phi$ est injective. On
dit aussi que les $x_i$ sont linéairement indépendants.
\item On dit qu'elle est génératrice si $\phi$ est surjective.
\item On dit que c'est une base si $\phi$ est bijective. C'est à dire qu'elle
est libre et génératrice.
\end{enumerate}
\end{defi}

\begin{example}[Remarque]\ 
\begin{enumerate}
 \item Une famille est génératrice si et seulement si elle engendre $M$ comme
module.\\ En particulier tout module admet une famille génératrice : $(x)_{x\in
M}$.
\item Si $K$ est un corps alors un $K$ module est un espace vectoriel et admet
une base, mais ce n'est pas le cas pour un anneau en général.
\end{enumerate}
\end{example}
\vspace{0.5em}

\begin{defi}[Module libre]
 
Un $A$ module $M$ est dit libre s'il admet un base.

Cela revient à dire qu'il
existe un ensemble $I$ tel que $M\simeq A^{(I)}$.
\end{defi}

\begin{example}[Exemples]
 $A$, $A^{(I)}$, $A[X]$.
\end{example}

\vspace{0.5em}
\begin{defi}[Annulateur d'un module]

L'annulateur d'un module $M$ est :
\begin{displaymath}
 \Ann M = \{a\in A \mid \forall x\in M,\ ax = 0\}
\end{displaymath}
\end{defi}

\begin{example}[Remarques]\

\begin{enumerate}
 \item En considérant le morphisme $\rho : A \rightarrow \End(M,+)$ définissant
la structure de $A$-module de $M$ on remarque que $\Ann(M) = \Ker\rho$ est un
idéal de $A$.
\item Si $I$ est un idéal de $M$ contenu dans $\Ker\rho = \Ann M$ alors par
passage au quotient, il existe un morphisme d'anneau $\tilde{\rho} :
\quo{A}{I} \rightarrow \End(M,+)$. Donc $M$ hérite d'une structure de
$\quo{A}{I}$ module.
\item Si $M$ est un $A$-module et $I$ un idéal de $A$ alors $IM$ est un
sous-module de $M$. On a alors $\Ann \quo{M}{IM} \subset I$ donc le $A$-module
$\quo{M}{IM}$ hérite d'une structure de $\quo{A}{I}$-module.
\end{enumerate}
\end{example}

\begin{defiprop}[Rang d'un module libre]

Soit $A$ un anneau non nul et soit $L$ un $A$-module libre.

Alors toutes les
bases de $L$ ont le même cardinal appelé rang de $L$ et noté $\rang L$.
\end{defiprop}

\begin{proof} Comme $L$ est libre on a $L\simeq A^{(I)}$ pour un certain
ensemble $I$.

Soit $m$ un idéal maximal de $A$. Par définition $\quo{A}{m}$ est un corps donc
d'après la troisième remarque $\quo{L}{mL}$ est un $\quo{A}{m}$-module donc un
$K$-espace vectoriel admettant une base de cardinal $C$ (on rappelle qu'un
cardinal est une classe d'équivalence sur les ensembles définie par la relation
d'être en bijection).

Comme $\quo{L}{mL} \simeq 
\quo{A^{(I)}}{mA^{(I)}} $ on déduit que $I\in C$.
\end{proof}

\subsubsection{Torsion}
\vspace{0.5em}

\begin{defi}[Elément de torsion]
 
Un élément $x$ d'un $A$-module $M$ est un élément de torsion s'il existe un
$a\in A$ non nul tel que $ax = 0$.

Un module est dit de torsion si tous ses éléments sont de torsion.

Un module est dit sans torsion si le seul élément de torsion est $0$.
\end{defi}

\begin{example}[Remarques]\
 
 \begin{enumerate}
  \item Un module libre est sans torsion car isomorphe à $A^{(I)}$ avec $A$
intègre.
\item Si $\Ann M \neq 0$ alors $M$ est de torsion. Réciproquement, si $M$ est
de type fini et de torsion alors $\Ann M \neq 0$.
\item Si $I$ est un idéal de $A$ alors l'annulateur de $\quo{A}{I}$ est $I$.
Donc mour $I$ non nul, $\quo{A}{I}$ est un module de torision. Anisi si $A$
n'est pas un corps, il existe des modules de torsion non nuls, et ils ne sont
pas libres.
 \end{enumerate}
\end{example}

\begin{defiprop}[Sous-module de torsion]
 
 Soit $M$ un $A$-module. L'ensemble des éléments de torsion de $M$ est un
sous-module de $M$ noté $\textnormal {T}(M)$ ou $M\tors$.
\end{defiprop}

\begin{example}[Remarque] On a la suite exacte courte :
\vspace{-1em}
 \begin{displaymath}
\begin{array}{ccccccccc}
  &                &        & i             & & \pi & & & \\
0 &\longrightarrow & M\tors &\longrightarrow&M&\longrightarrow
&\quo{M}{M\tors}&\longrightarrow& 0
 \end{array} \end{displaymath}

Donc $\quo{M}{M\tors}$ est sans torsion.
\end{example}

\subsubsection{Module de fraction}
\vspace{0.5em}

Soit $A$ un anneau et $S$ une partie multiplicative de $A$. On construit, comme
dans la section précédante l'algèbre $S^{-1}A$. Alors un $S^{-1}A$-module est
naturellement muni d'une structure de $A$-module.

Réciproquement, étant donnée un $A$-module $M$ on va construire un $S^{-1}A$
module noté $S^{-1}M$.

\begin{enumerate}
 \item On muni $S\times M$ de la relation d'équivalence $\Rel$ définie par :
\begin{displaymath}
 (s,x) \Rel (s',x') \Longleftrightarrow \exists \sigma \in S,\ 
 \sigma s' x = \sigma s x'
\end{displaymath}
\item On pose $S^{-1}M = \quo{S\times M}{\Rel}$ dont les éléments seront notés
$\quo{x}{s}$.
\end{enumerate}











 















