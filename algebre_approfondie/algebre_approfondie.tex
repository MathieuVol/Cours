\documentclass[a4paper,10pt,makeidx]{article}

\usepackage[utf8x]{inputenc}
\usepackage{fullpage}
\usepackage[cut=false]{thmbox}
\usepackage{amsmath}
\usepackage{amssymb}
\usepackage{amsfonts}
\usepackage[french]{babel}
\usepackage{makeidx}
\usepackage[pdftex,colorlinks,linkcolor=black]{hyperref}

%\newtheorem{defi}{Définition}[section]
%\newtheorem{prop}{Proposition}[section]
%\newtheorem{coro}[prop]{Corollaire}
%\newtheorem{lemm}[prop]{Lemme}
%\newtheorem{theo}[prop]{Théorème}
%\newtheorem{defiprop}[defi]{Définition - Proposition}
%\newtheorem{defitheo}[defi]{Définition - Théorème}


%\newtheorem{resdefi}{Définition}[section]
%\newtheorem{resprop}{Proposition}[section]
%\newtheorem{rescoro}[resprop]{Corollaire}
%\newtheorem{reslemm}[resprop]{Lemme}
%\newtheorem{restheo}[resprop]{Théorème}
%\newtheorem{resdefiprop}[resdefi]{Définition - Proposition}
%\newtheorem{resdefitheo}[resdefi]{Définition - Théorème}

\newtheorem[style=S]{defi}{Définition}[section]
\newtheorem[style=M]{prop}{Proposition}[section]
\newtheorem[style=M]{coro}[prop]{Corollaire}
\newtheorem[style=M]{lemm}[prop]{Lemme}
\newtheorem[style=L]{theo}[prop]{Théorème}
\newtheorem[style=S]{defiprop}[defi]{Définition - Proposition}
\newtheorem[style=S]{defitheo}[defi]{Définition - Théorème}


\newtheorem[style=S]{resdefi}{Définition}[section]
\newtheorem[style=S]{resprop}{Proposition}[section]
\newtheorem[style=S]{rescoro}[resprop]{Corollaire}
\newtheorem[style=S]{reslemm}[resprop]{Lemme}
\newtheorem[style=S]{restheo}[resprop]{Théorème}
\newtheorem[style=S]{resdefiprop}[resdefi]{Définition - Proposition}
\newtheorem[style=S]{resdefitheo}[resdefi]{Définition - Théorème}



\author{Notes de Cours}
\title{Algèbre Approfondie}
\date{2011}




% OPERATEURS :
% ------------

% Boules
\DeclareMathOperator{\Boule}{B}
\DeclareMathOperator{\adh}{adh}

% Image / Noyau / ...
\DeclareMathOperator{\Ima}{Im\,}
\DeclareMathOperator{\Ker}{Ker\,}
\DeclareMathOperator{\Nil}{Nil\,}

% Fonctions classiques
\DeclareMathOperator{\Id}{Id}
\DeclareMathOperator{\e}{e}
\DeclareMathOperator{\Frac}{Frac}
\DeclareMathOperator{\Log}{Log}
\DeclareMathOperator{\Exp}{Exp}
\DeclareMathOperator*{\Sup}{Sup}
\DeclareMathOperator*{\Inf}{Inf}
\DeclareMathOperator*{\Max}{Max}
\DeclareMathOperator*{\Min}{Min}
\DeclareMathOperator{\Zen}{Z} % Zentrum (centre)
\DeclareMathOperator{\Cen}{C} % Centralisateur
\DeclareMathOperator{\Com}{C} % Combinatoire
\DeclareMathOperator{\Nor}{N} % Normalisateur
\DeclareMathOperator{\Rad}{Rad} % Radical (de Jacobson)
\DeclareMathOperator{\pgcd}{pgcd}
\DeclareMathOperator{\ppcm}{ppcm}
\DeclareMathOperator{\Spectre}{Sp}
\DeclareMathOperator{\Aut}{Aut}
\DeclareMathOperator{\End}{End}
\DeclareMathOperator{\Ann}{Ann}
\DeclareMathOperator{\rang}{rg}


% Fonction classiques (Actions)
\DeclareMathOperator{\Stab}{Stab\,}

% Fonctions classiques (Matrices)
\DeclareMathOperator{\Mat}{Mat\,}
\DeclareMathOperator{\Tr}{Tr\,}
\DeclareMathOperator{\Diag}{Diag\,}

\makeindex

\begin{document}

\newcommand{\quo}[2]{\raisebox{0.25em}{\ensuremath{#1}}\hspace{0.05em}/\hspace{
0.1em}
\raisebox{-0.1em}{\ensuremath{#2} } }

\newcommand{\Quo}[2]{\raisebox{0.25em}{\ensuremath{#1}}\hspace{0.05em}/\hspace{
0.em}
\raisebox{-0.25em}{\ensuremath{#2} } }

% Topologie 

\newcommand{\XO}{$(X,\mathcal{O})$}
\newcommand{\XOX}{$(X,\mathcal{O}_X)$}
\newcommand{\YOY}{$(Y,\mathcal{O}_Y)$}

% ``Tel que`` dans une definition ensembliste {x \tq p(x)}

\newcommand{\tq}{ \: | \: }
\newcommand{\btq}{ \: \big| \: }
\newcommand{\Btq}{ \: \Big| \: }
\renewcommand{\|}{|\negthinspace\hspace{0.07em}|} % norme ||.||
\newcommand{\f}{\|_{{}_F}} % norme ||.||
\renewcommand{\*}{^*\!{}} % conjuguée
\renewcommand{\t}{{}^t\!} % transposée

% Ensembles classiques de nombres (mathbb)

\newcommand{\Z}{\mathbb{Z}}
\newcommand{\N}{\mathbb{N}}
\newcommand{\R}{\mathbb{R}}
\newcommand{\C}{\mathbb{C}}
\newcommand{\K}{\mathbb{K}}
\newcommand{\Prem}{\mathbb{P}}
\newcommand{\Q}{\mathbb{Q}}
\newcommand{\UU}{\mathbb{U}}

% Ensembles classiques (Matrices)
% --------- ---------- ----------

% hermitienne (positive, def positive)

\renewcommand{\H}{\mathcal{H}}
\newcommand{\Hp}{\mathcal{H}^{+}}
\newcommand{\Hpp}{\mathcal{H}^{++}}

% symétrique (positive, def positive)

\renewcommand{\S}{\mathcal{S}}
\newcommand{\Sp}{\mathcal{S}^{+}}
\newcommand{\Spp}{\mathcal{S}^{++}}

% sous-groupes

\newcommand{\GL}{\mathcal{GL}}
\newcommand{\SO}{\mathcal{SO}}
\newcommand{\U}{\mathcal{U}}
\newcommand{\SU}{\S\U}
\newcommand{\Spl}{\mathcal{S}\textnormal{p}}
\newcommand{\USp}{\mathcal{US}\textnormal{p}}
\newcommand{\M}{\mathcal{M}}
\renewcommand{\O}{\mathcal{O}}
\newcommand{\SL}{\mathcal{SL}}

% Autres
% ------


\newcommand{\tors}{_{\hspace{-0.05em}\textnormal{tors}}}

% < et > triangle

\newcommand{\<}{\vartriangleleft}
\renewcommand{\>}{\vartriangleright}

% Divise : ''a | b``
\newcommand{\divise}{\: |\: }

\newcommand{\x}{^{\hspace{-0.1em}\times}}
\renewcommand{\-}{\!\smallsetminus\!{}}

\newcommand{\Rel}{\mathcal{R}}
\newcommand{\Bigoplus}{\displaystyle\bigoplus}
\newcommand{\Prod}{\displaystyle\prod}



%%%%%%%%%%%%%%%%%%%%%%%%%%%%%%%%%%%%%%%%%%%%%%%%%%%%%%%%%%%%%%%%%%%%%%%%%%%%%%%%
%% DEBUT DOCUMENT %%%%%%%%%%%%%%%%%%%%%%%%%%%%%%%%%%%%%%%%%%%%%%%%%%%%%%%%%%%%%%


% Page de titre

\maketitle
\tableofcontents

\pagebreak
\include{partieGroupes}

\pagebreak
\include{partieAnneaux}

\pagebreak
\include{partieModules}


\pagebreak
\printindex

\pagebreak
\include{coursResume}


%% FIN DOCUMENT %%%%%%%%%%%%%%%%%%%%%%%%%%%%%%%%%%%%%%%%%%%%%%%%%%%%%%%%%%%%%%
%%%%%%%%%%%%%%%%%%%%%%%%%%%%%%%%%%%%%%%%%%%%%%%%%%%%%%%%%%%%%%%%%%%%%%%%%%%%%%%%


\end{document}

